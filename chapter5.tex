\chapter{A counterexample of the boundary coarse Baum-Connes conjecture.}
In this chapter we develop the ideas of Higson, Lafforgue and Skandalis concerning the counterexamples to the coarse Baum-Connes conjecture further, to construct a metric space that has the properties constructed in \cite{} at infinity. The main idea is to decompose the boundary groupoid further, giving a new short exact sequence at infinity similar to the sequences considered in Chapter 4. From this, we then construct an operator that is not a ghost operator, but is ghostly on certain parts of the boundary. A tracelike argument, similar to those of \cite{higsonpreprint, explg1, }, then completes the proof.

\section{The space and its non-ghosts.}

The space we are going to consider first appeared in \cite{MR2363697}.

Let $\lbrace X_{i} \rbrace_{i \in \mathbb{N}}$ be a sequence of finite graphs. Then we construct a space similar to a space of graphs in the following manner: Let $Y_{i,j} = X_{i}$ for all $j \in \mathbb{N}$ and consider $Y:= \sqcup_{i,j} Y_{i,j}$. We metrize this space using a box metric - that is with the property that $d(Y_{i,j},Y_{k,l}) \rightarrow \infty$ as $i+j+k+l \rightarrow \infty$. 

Now let $\lbrace X_{i} \rbrace_{i}$ be an expander sequence. As discussed in section \ref{Sect:GO}, we can construct a ghost operator $p= \prod_{i} p_{i}$ on $X$, the space of graphs of $\lbrace X_{i} \rbrace_{i}$. Similarly, we can construct this operator on $Y$. In this situation we get a projection $q:=\prod_{i,j}p_{i} \in C^{*}_{u}Y$, which is a constant operator in the $j$ direction. This was precisely the operator of interest in \cite{MR2363697}, as it can be seen that $q$ is not a ghost operator, as its matrix entries do not vanish in the $j$ direction - a fact proved below in Lemma \ref{Lem:nag}.

Recall that associated to $Y$ we have a short exact sequence of $C^{*}$-algebras:
\begin{equation*}
\xymatrix{
0 \ar[r] & ker(\pi) \ar[r]& C_{r}^{*}(G(Y)) \ar[r]^{\pi} & C_{r}^{*}(G(Y)|_{\partial\beta Y}) \ar[r] & 0.
}
\end{equation*}

We remark the kernel, $ker(\pi)$ consists of all the ghost operators in $C^{*}_{u}(Y)$, that is those operators with matrix coefficients that tend to $0$ in all directions on the boundary. 

\begin{lemma}\label{Lem:nag}
The projection $\pi(q):= \prod_{i,j}p_{i} \not = 0 \in C^{*}_{r}(G(Y)|_{\partial\beta Y})$. That is $q \not\in ker(\pi)$.
\end{lemma}
\begin{proof}
We first observe that every bounded subset $B$ of $Y$ is contained in some rectangle of the form $R_{i_{B},j_{B}}:=\sqcup_{i\leq i_{B},j\leq j_{B}}Y_{i,j}$. So to prove that $q$ is not a ghost operator it suffices to show that there exists an epsilon such that for all rectangles $R_{i,j}$ there is a pair of points $x,y$ in the compliment of the rectangle such that  the norm $\Vert q_{x,y} \Vert \geq \epsilon$. To prove this, recall that the projection $q$ is a product of projections $p_{i}$ on each $X_{i}$ and fixing $j$, these projections form a ghost operator. 

Fix $\epsilon = \frac{1}{2}$. Then there exists an $i_{\epsilon}$ with the property that $\forall i>i_{\epsilon}$ and for every $x,y \in \sqcup_{i}X_{i}$ we know $\Vert p_{i,x,y} \Vert < \epsilon$. We remark that this $i_{\epsilon}$ can be taken to be the smallest such. So for $i \leq i_{\epsilon}-1$, we have that $\Vert p_{i,x,y} \Vert \geq \frac{1}{2}$. Now let $R_{i_{\epsilon}-1,\infty}$ be the vertical rectangle $\sqcup_{i\leq i_{\epsilon}-1,j} Y_{i,j}$. 

To finish the proof, consider an arbitrary finite rectangle $R_{i,j}$. This intersects the infinite rectangle $R_{i_{\epsilon}-1,\infty}$ in a bounded piece. Now pick any pair of points in a fixed box $x,y \in Y_{k,l} \subset R_{i_{\epsilon}-1,\infty} \setminus R_{i,j}$. Then for those points $x,y$ it is clear that $\Vert q_{x,y} \Vert = \Vert p_{k,x,y}\Vert \geq \frac{1}{2}$.
\end{proof}

We now describe the boundary $\partial\beta Y$. We are aiming at a decomposition into saturated pieces and with that in mind we construct a map to $\beta X$.

Consider the map $\beta Y \twoheadrightarrow \beta X \times \beta \mathbb{N}$ induced by the bijection of $Y$ with $X \times \mathbb{N}$ and the universal property of $\beta Y$. Now define:
\begin{equation*}
f: \beta Y \rightarrow \beta X \times \beta \mathbb{N} \rightarrow \beta X
\end{equation*}
The map $f$ is continuous, hence the preimage of $X$ under projection onto the first factor is an open subset of $\beta Y$, which intersects the boundary $\partial \beta Y$. In fact, what we can see is that $f^{-1}(X)= \sqcup f^{-1}(X_{i})$, where each $f^{-1}(X_{i})$ are closed, and therefore homeomorphic to $X_{i} \times \beta \mathbb{N}$. We can define $U = f^{-1}(X)\cap \partial\beta Y$.

\section{The boundary groupoid associated to a certain $\Gamma$.}

To continue, we pass from the general case to a specific example; let $\Gamma$ be a finitely generated residually finite discrete group with the Haagerup property, and let $\lbrace N_{i}\rbrace$ be a family of nested finite index subgroups with trivial intersection. Let $X_{i}:=Cay(\frac{\Gamma)}{N_{i}})$. In this context, the boundary groupoid is generated by the action of the group $\Gamma$ extended to the boundary (see Proposition \ref{Prop:Crit}). In this context we can show that $U$ defined above is saturated:

\begin{lemma}
$U$ is an open, saturated subset of the boundary $\partial\beta Y$. 
\end{lemma}
\begin{proof}
We showed above that $U$ was open. To see it is saturated we prove that $U^{c}$ is saturated, observe that the following diagram commutes:
\begin{equation*}
\xymatrix{
\overline{g}_{Y}:\beta Y\ar[r]\ar[d]^{p} & \beta Y\ar[d]^{p}\\
\overline{g_{X} \times 1}:  \beta X \times \beta \mathbb{N} \ar[r] & \beta X \times \beta \mathbb{N}
}
\end{equation*}
where these maps extend the group action on $Y=X \times \mathbb{N}$. The projection onto $\beta X$ is also equivariant under this action. Assume for a contradiction that $U^{c}$ is not saturated; there exists $\gamma$ in $U^{c}$ such that $\overline{g}_{Y}(\gamma) \in U$. It follows that $\overline{g_{X} \times 1}(p(\gamma))$ is in $p(U)$, whilst $p(\gamma) \in p(U^{c})$, hence $\overline{g_{X}}(f(\gamma))\in U$ whilst $f(\gamma) \in U^{c}$. This is a contradiction as $f(U^{c}) = \partial\beta X$ is saturated.
\end{proof}

This gives us two natural complimentary restrictions of $G(Y)|_{\partial\beta Y}$ and a short exact sequence of function algebras as in section \ref{Sect:CE}:
\begin{equation*}
\xymatrix{
0 \ar[r] & C_{c}(G(Y)|_{U}) \ar[r]& C_{c}(G(Y)|_{\partial\beta Y}) \ar[r] & C_{c}(G(Y)|_{U^{c}}) \ar[r] & 0.
}
\end{equation*}

We will now show that the corresponding sequence:
\begin{equation*}
\xymatrix{
0 \ar[r] & C^{*}_{r}(G(Y)|_{U}) \ar[r]& C_{r}^{*}(G(Y)|_{\partial\beta Y}) \ar[r]^{h} & C_{r}^{*}(G(Y)|_{U^{c}}) \ar[r] & 0
}
\end{equation*}
fails to be exact in the middle. We proceed as in \cite{explg1,MR1911663} by using the element $\pi(q)$, which certainly vanishes under the quotient map from $C^{*}_{r}(G(Y)|_{\partial\beta Y}) \rightarrow C^{*}_{r}(G(Y)|_{U^{c}})$. To show the failure we will show this sequence fails to be exact in the middle at the level of K-theory and for this we will require a firm understanding of the structure of $G(Y)|_{U}$.

We observe the following facts: 
\begin{enumerate}
\item $\Gamma$ acts on the space $Y:=\sqcup_{i,j}Y_{i,j}$ built from $\lbrace X_{i} \rbrace$.
\item This action becomes free on piece of the boundary that arises as $i \rightarrow \infty$, that is on $U^{c}$.
\item The group action generates the metric coarse structure on the boundary; The finite sets associated to each $R>0$ in the decomposition are now finite rectangles - because of the metric prescribed on the space $Y$.
\end{enumerate} 

It follows from Proposition \ref{Prop:Crit} that the groupoid $G(Y)|_{U^{c}}$ is isomorphic to $U^{c}\rtimes \Gamma$ and under the assumption that $\Gamma$ has the Haagerup property we can conclude that the Baum-Connes assembly map for the groupoid $G(Y)|_{U^{c}}$ is an isomorphism (with any coefficients). We now concern ourselves with $G(Y)|_{U}$.

\begin{lemma}\label{Lem:CE3}
The groupoid $G(Y)|_{U}$ is isomorphic to $\sqcup_{i}(X_{i}\times X_{i})\times G(\mathbb{N})|_{\partial\beta \mathbb{N}}$.
\end{lemma}
\begin{proof}
The reductions to the inclusions of the preimages $f^{-1}(X_{i})$ restricted to the boundary are isomorphic to the closed subgroupoids $G(X_{i}\times \mathbb{N})|_{\partial\beta \mathbb{N}}$ of $G(Y)|_{U}$. These groupoids are disjoint by construction and therefore the inclusion $\sqcup_{i}G(X_{i}\times \mathbb{N})|_{\partial\beta \mathbb{N}}$ is an open subgroupoid of $G(Y)|_{U}$. We now prove that:
\begin{enumerate}
\item each $G(X_{i} \times \mathbb{N})|_{\partial\beta \mathbb{N}}$ is isomorphic to $(X_{i}\times X_{i})\times G(\mathbb{N})|_{\partial\beta \mathbb{N}}$;
\item the union $\sqcup_{i}G(X_{i}\times \mathbb{N})|_{\partial\beta \mathbb{N}}$ is the entire of $G(Y)|_{U}$.
\end{enumerate}
To prove (1), observe that the groupoid decomposes as $G(X_{i}\times \mathbb{N}) = \bigcup_{R>0}\overline{\Delta_{R}(X_{i}\times \mathbb{N})}$. For each $R>0$ we can find a $j_{R}$ such that $\Delta_{R}(X_{i}\times \mathbb{N}) = F_{R} \cup \bigcup_{j>j_{R}}\Delta_{R}^{j}(X_{i}\times \mathbb{N})$, hence for the boundary part of this groupoid it is enough to understand what happens in each piece $Y_{i,j}$, which is constant for each $j$. Secondly, observe that in the induced metric on a column, the pieces $Y_{i,j}$ separate as $j\rightarrow \infty$. This, coupled with the fact that for large enough $R$, we know that $ \Delta^{j}_{R}(X_{i}\times \mathbb{N}) = X_{i} \times X_{i}$ allow us to deduce that any behavour at infinity of this groupoid is a product of $X_{i} \times X_{i}$ and the boundary groupoid $G(\mathbb{N})|_{\partial\beta \mathbb{N}$ where $\mathbb{N}$ has the coarsely disconnected metric. This groupoid is isomorphic to $\partial\beta \mathbb{N}$, from which we can deduce that $G(X_{i}\times \mathbb{N})|_{\partial\beta \mathbb{N}} = (X_{i}\times X_{i})\times \partial\beta \mathbb{N}$ for each $i$.

To prove (2) we assume for a contradiciton that there is a partial translation $t$, such that $\overline{t}$ is not an element of the disjoint union. Such an element maps some $(x_{i}, \omega)$ to $(x_{k},\omega)$, where $i\not =k$. Without loss of generality assume also $t$ has translation length at most $R$. Then the domain and range of $t$ are both infinite (as the closure is defined in $G(Y)|_{U}$), and must be contained within a strip of width at most $R>0$. From the definition of the metric, there are only finitely many $Y_{i,j}$ within such a rectangle, hence $t \in F_{R}$ and hence $\overline{t}$ is not defined in $G(Y)|_{U}$, which yields a contradiction.
\end{proof}

\begin{remark}
Lemma \ref{Lem:CE3} allows us to conclude that $C^{*}_{r}(G(Y)|_{U}) \cong \bigoplus_{i}M_{\vert X_{i}\vert} \otimes C(\partial\beta Y)$
\end{remark}

To conclude that $[\pi(q)]$ is not an element of $K_{0}(C^{*}_{r}(G(Y)|_{U}))$ we construct a trace-like map.

\begin{theorem}
The element $h_{*}[\pi(q)]=0$, but does not belong to $K_{0}( C^{*}_{r}(G(Y)|_{U}))$.
\end{theorem}
\begin{proof}
We begin the proof by remarking that each $U_{i}:=f^{-1}(X_{i}) \cap U$ is a closed saturated subset of $\partial\beta Y$, hence we can consider the reduction to $U_{i}$ for each $i$. We consider the product, and the following map:
\begin{eqnarray*}
\phi : C^{*}_{r}(G(Y)|_{\partial\beta Y})& \rightarrow & \prod_{i} C^{*}_{r}(G(Y)|_{U_{i}})= \prod_{i}C^{*}_{r}(G(X_{i} \times \mathbb{N})\\
 T  &\mapsto & \prod_{i}T|_{U_{i}}
\end{eqnarray*}
Under the map $\phi$, the subalgebra $C^{*}_{r}(G(Y)|_{U})= \bigoplus_{i}M_{\vert X_{i}\vert} \otimes C(\partial\beta Y)$ maps to the ideal $\bigoplus_{i}C^{*}_{r}(G(X_{i} \times \mathbb{N}))$. So, we can define a tracelike map, in analogy to \cite[Section 6]{explg1}, by composing with the quotient map $\tau$ onto $\frac{\prod_{i}C^{*}_{r}(G(X_{i} \times \mathbb{N}))}{\bigoplus_{i}C^{*}_{r}(G(X_{i} \times \mathbb{N}))}$. This gives us a map at the level of K-theory:
\begin{equation*}
Tr_{*}=\phi \circ \tau : K_{0}(C^{*}_{r}(G(Y)|_{\partial\beta Y})) \rightarrow \frac{\prod_{i}K_{0}(C^{*}_{r}(G(X_{i} \times \mathbb{N})))}{\bigoplus_{i}K_{0}(C^{*}_{r}(G(X_{i} \times \mathbb{N})))}= \frac{\prod_{i}K_{0}(C(\partial\beta \mathbb{N}))}{\bigoplus_{i}K_{0}(C(\partial\beta \mathbb{N}))}
\end{equation*}

We now consider $[\pi(q)]$ under $Tr_{*}$. Recall that $q=\prod_{i,j}p_{i}$. We define $q_{i}=\prod_{j}p_{i}$ and observe that the operation of reducing to $G(Y)|_{U_{i}}$ can be performed in two commuting ways: restricting to $U$ then $f^{-1}(X_{i})$ or by restricting to $f^{-1}(X_{i})$ then $U$. The second tells us that $q_{i}=p_{i} \otimes 1_{\beta\mathbb{N}$ is constant in the $j$ direction and when restricted to the boundary is $\pi(q_{i})=p_{i}\otimes 1_{\partial\beta \mathbb{N}}$. Hence, $Tr_{*}([\pi(q)])=[1_{\partial\beta \mathbb{N}},1_{\partial\beta\mathbb{N}},...] \not = 0$ and so $[\pi(q)] \not \in K_{0}(C^{*}_{r}(G(Y)|_{U})$.
\end{proof}

So in this case we have the following diagram:
\begin{equation*}
\xymatrix@=0.7em{
 K_{1}(C(U^{c})\rtimes \Gamma) \ar[r] & K_{0}(\ker (\pi)) \ar[r]& K_{0}(C^{*}_{r}(G(Y)|_{\partial\beta Y})) \ar[r]& K_{0}(C(U^{c})\rtimes \Gamma)\ar[r] & K_{1}(\ker (\pi))  \\
 K_{1}^{top}(U^{c}\rtimes \Gamma) \ar[r] \ar[u]^{\ucong}& K_{0}^{top}(Y \times Y) \ar[r]\ar@{^{(}->}[u]\ar@{^{(}->}[ru]& K_{0}^{top}(G(Y)|_{\partial\beta Y} \ar[r]\ar[u]^{\mu_{bdry}}& K_{0}^{top}(U^{c}\rtimes \Gamma) \ar[r]\ar[u]^{\ucong}& K_{1}^{top}(X \times X)\ar[u]
}
\end{equation*}

\begin{remark}
We justify the diagonal inclusion of $K_{0}^{top}(Y \times Y)$ into $K_{0}(C^{*}_{r}(G(Y)|_{\partial\beta Y}$. This follows as the groupoid $G(Y)_{U}$ is nuclear, and hence the assembly map is an isomorphism.   The algebra $C^{*}_{r}(G(Y)_{U})= \bigoplus_{i}M_{\vert X_{i}\vert} \otimes C(\partial\beta Y)$ injects into the product $\prod_{i} M_{\vert X_{i}\vert} \otimes C(\partial\beta Y)$ at the level of K-theory and this inclusion factors through in the inclusion into the kernel of $\pi$ and into $C^{*}_{r}(G(Y)|_{\partial\beta Y}$. These maps provide enough information to conclude injectivity of the assembly map $\mu_{bdry}$.
\end{remark}

A diagram chase under the assumption that the map $\mu_{bdry}$ is surjective quickly yields a contradiction, whence we have:

\begin{corollary}
The assembly map $\mu_{bdry}$ associated to $Y=\sqcup_{i,j}Cay(\frac{\Gamma}{N_{i}})$ is not surjective but is injective.\qed
\end{corollary}

\chapter{An application to the Exactness of Gromov Monster groups.}

It is well known \cite{MR1911663,explg1} that any group that contains a coarsely embedded expander does not have Yu's property A and admits coefficients for which the Baum-Connes conjecture fails to be a surjection, but is an injection:

\begin{theorem}\label{Thm:IT1}
Let $G$ be a Gromov monster group. Then there exists a (commutative) $G$-$C^{*}$-algebra $A$ such that the Baum-Connes assembly map:
\begin{equation*}
\mu_{r,A,*}: KK^{G}_{*}(\underline{E}G, A) \rightarrow K_{*}(A\rtimes_{r}G)
\end{equation*}
is not surjective, but is injective.
\end{theorem}

We explore this result from the point of view of the geometry that can be associated to the expander graph that it inherits from the group.

From the point of view of \cite{MR1905840}, the coarse conjecture for any uniformly discrete bounded geometry space $X$ can be phrased as a conjecture with coefficients in a certain groupoid $G(X)$ associated to $X$. This groupoid admits a transformation groupoid decomposition \cite[Lemma 3.3b)]{MR1905840}, giving an easy description of $G(X)$ when it is possible to get a handle on the generators of the metric coarse structure on $X$. When $X$ is coarsely embedded into a group, this is certainly the case; the concept of a partial translation structure \cite{MR2363428} gives any space coarsely embedded into a group a nice collection of generators, as well as a locally compact, Hausdorff, second countable \etale groupoid that impliments the transformation decomposition.

On the other hand, the question of when a groupoid admits a transformation groupoid decompostion, up to Morita equivalence, has been well studied for the class of groupoids that are constructed from suitable inverse semigroups \cite{MR1900993,Milan-Steinberg}. This is related to the problem of globalisation of a partial action of $\Gamma$ on a space $X$. The result would be a space $Y$, with a true action of $\Gamma$ such that $X \hookrightarrow Y$ is a topological embedding and the enlargement of the $\Gamma$ action by restricting the maps to $X$ induces the original partial action.

The problem of globalisation of partial actions of groups has been considered in a variety of settings \cite{MR0160848, MR1798993, MR2041539, MR2419858, MR1900993, Milan-Steinberg}, each using the same central theme.

\begin{definition}
Let $X$ be a topological space and let $G$ be a group acting partially on $X$. Then we denote by $\Omega$ the \textit{Morita evelope} of the action of $G$ on $X$, which is constructed as follows:

Consider the space $X\times \Gamma$, equipped with the product topology. Then define $\sim$ on $X\times \Gamma$ by $(x,g)\sim (y,h)$ if there exists $\gamma \in \G$ with $x(h^{-1}g)=y$. We define $\Omega$ as the quotient of $X\times \Gamma$ by $\sim$ with the quotient topology. 
\end{definition}

The main idea that was presented first by Khoskham and Skandalis \cite{MR1900993} was the concept of a \textit{group valued cocycle}, which we outline below. Milan and Steinberg then developed this idea much further, giving an almost full answer to the problem of which groupoids are Morita equivalent to a transformation groupoid \cite{Milan-Steinberg}.

\section{Groupoid Valued Cocycles and a Theorem of Milan and Steinberg.}

In this section we consider the question of when a groupoid admits a transformation groupoid decompostion up to Morita equivalence. This has been well studied for the class of groupoids that are constructed from suitable inverse semigroups \cite{MR1900993,Milan-Steinberg} that admit a certain type of map onto an inverse semigroup:

\begin{definition}
Let $G$ be a locally compact groupoid. Then we call a continuous homomorphism from $G$ to a locally compact group $\Gamma$ a group valued cocycle (or just cocycle).
\end{definition}

\begin{definition}
Let $\rho: G \rightarrow \Gamma$ be a cocycle. We say it is:
\begin{enumerate}
\item \textit{transverse} if the map $\Gamma \times G \rightarrow \Gamma \times X$, $(g, \gamma) \mapsto (g\rho(\gamma),s(\gamma))$ is open.
\item \textit{closed} if the map $\gamma \mapsto ((r(\gamma),\rho(\gamma),s(\gamma))$ is closed.
\item \textit{faithful} if the map $\gamma \mapsto ((r(\gamma),\rho(\gamma),s(\gamma))$ is injective.
\end{enumerate}
We call a cocycle $\rho$ with all these properties a \textit{(T,C,F)-cocycle}.
\end{definition}

Below is the main result of \cite{Milan-Steinberg}, a generalisation of the main results of \cite{MR1900993}:

\begin{theorem}\label{Thm:IT2}
Let $\rho: \G \rightarrow S$ be a continuous, faithful closed transverse cocycle where $\G$ is a locally compact groupoid and $S$ is a countable inverse semigroup. Then there is a locally compact Hausdorff space $X$ equipped with an action of $S$ so that $\G$ is Morita equivalent to the groupoid of germs $X \rtimes S$. Consequently $C^{*}_{max}\G$ is strongly Morita equivalent to $C_{0}(X)\rtimes S$. If $S$ is a group, then the analogous result holds for reduced $C^{*}$-algebras.
\end{theorem}

From an F-inverse monoid $S$ it is possible to construct a (T,C,F)-cocycle onto the maximal group homomorphic image of $G$ \cite{MR1900993}. To prove Theorem \ref{Thm:IT2} in the case that the monoid is $F$-inverse then makes use of the Morita envelope of the partial action that the maximal group homomorphic image $G$ has on the unit space of the universal groupoid $\G_{\E}$. In this case the space is a quotient of  $\G^{(0)}\times \Gamma$ equipped with the product topology.  The closed condition on the cocycle makes this space Hausdorff.

What follows from here can be found as a corollary to Theorem \ref{Thm:IT2} from \cite{Milan-Steinberg}. We provide a direct proof of a special case using the original methods of \cite{MR1900993}. This is possible by considering the construction of the groupoid $G(S)$ for a strongly 0-E-unitary inverse monoid $S$. It is clear that the only danger is mapping elements to $0$ in $\Gamma^{0}$; this is overcome by the observation that the element $[0,f]$ would be defined if and only if $f \in D_{0}$. However, $f \in D_{0}$ implies that $f(0)=1$ and hence $f \not\in \E$, so the $0$ element of $S$ contributes nothing to the groupoid $G(S)$, either in objects or arrows.

We are interested in proving that if $S$ is a strongly $0$-F-inverse monoid that it is possible to apply Theorem \ref{Thm:IT2}. This is Corollary 6.17 from a \cite{Milan-Steinberg}, however we give a direct proof here for completeness just in the special case in which we are interested, by adapting the original techniques of \cite{MR1900993}. The main idea relies on carefully considering the universal groupoid, making sure that the cocycle does not interact with the zero element in the semigroup in the wrong way.

\begin{theorem}\label{Thm:IT2-a}
Let $S$ be an inverse monoid. If $S$ is strongly 0-E-unitary with universal group $U(S)=\Gamma$ such that the prehomomorphism has the finite cover property. Then the groupoid $G(S)$ admits a transverse and faithful cocycle to a group.
\end{theorem}
\begin{proof}
Let $\Phi$ be the 0-restricted, idempotent pure prehomomorphism onto $\Gamma^{0}$. We build an induced map on the groupoid $G(S)$ by considering a new map $\Psi:$
\begin{equation*}
\Psi([s,x])=\Phi(s)
\end{equation*}
This map is well-defined as any non-zero idempotent in $S$ is mapped to the identity in $\Gamma$, and so for any pair $(s,f) \sim (t,f)$ there is an $e \in E \cap D_{f}$, in particular not $0$, such that $es=et$ and hence $\Phi(s)=\Phi(es)=\Phi(et)=\Phi(t)$. This is clearly a groupoid homomorphism to $\Gamma$. To check it is continuous observe that as $\Gamma$ is a discrete group so all subsets are open. The preimage of a singleton is given by the union:
\begin{equation*}
\Psi^{-1}(\lbrace g \rbrace)=\bigcup_{\Phi(u)=g}[u,D_{u^{*}u}] 
\end{equation*}
which is certainly open in $G(S)$. The map is proper, because the preimage of any finite set in $\Gamma$ is given by a finite union of $[u,D_{u^{*}u}]$ that are compact by construction.

It remains to check it is a (T,C,F)-cocycle, and from the remarks prior to the Theorem the proof of this follows exactly from the proof \cite[Proposition 3.6]{MR1900993} modified suitably. We provide this proof below:

To prove this is transverse, it is enough to prove that $\lbrace (\Psi(\gamma),s(\gamma)):\gamma \in G(S)\rbrace$ is open in $\Gamma \times G(S)^{(0)}$, and this in turn reduces to studying this problem for all $g \in G$, that is if $\lbrace s(\gamma) :\Psi(\gamma)=g \rbrace$. is open in $G(S)^{(0)}$. This set is equal to $\bigcup_{\Psi(\gamma)=g}D_{s(\gamma)}$, which is certainly open in $G(S)^{(0)}$ as each piece is.

To see that this is faithful, let $[u,f], [v,f^{'}] \in G(S)$ such that $(f,\Phi(u),\theta_{u}(f))=(f^{'},\Phi(v),\theta_{v}(f^{'}))$. Then it is clear that $f=s([u,f])=s([v,f^{'}])=f^{'}$, so it is enough to prove now that $\Phi(v)=\Phi(u)$ implies $[u,f]=[v,f]$. Observe that $\Phi(u)\Phi(v)^{-1}=1$ in $\Gamma$ and $\Phi(v)^{-1}=\Phi(v^{*})$, so $\Phi(uv^{*})=1$. This map is idempotent pure, so $uv^{*} \in E(S)$. So $[u,f][v,f]^{-1}=[uv^{*},\theta_{v}(f)]$ is a unit in $G(S)$. From here it is clear that $[u,f]$ is an inverse to $[v^{*},\theta_{v}(f)]$ and so $[u,f]=[v,f]$.\end{proof}

This gets us a little closer to applying Theorem \ref{thm:1.8}. We still need to check the fact that the cocycles are closed. We proceed as in \cite{MR1900993}.

\begin{definition}
We say that S satisfies the finite cover property with respect to $\phi$,, if for every $p,q \in S$, $p,q\not = 0$ there exists a finite set $U \subset S_{g}$ such that:
\begin{equation*}
pS_{g}q=\lbrace s \in S | \exists u \in U | s \leq u \rbrace.
\end{equation*}
Where $S_{g}$ is the preimage $\phi^{-1}(g)$.
\end{definition}

Again, this follows from the work of \cite{Milan-Steinberg} or \cite{MR1900993}, but we give the proof in this setting:

\begin{lemma}
If $S$ is an inverse monoid and has the finite cover property with respect to $\phi$, then the induced cocycle $\rho$ is closed.
\end{lemma}
\begin{proof}
As $\Gamma$ is discrete, it is enough to prove that the graph $Gr(g)$ over $g$ in $G(S)^{(0)}\times G(S)^{(0)}$ is closed. We remark also that this product space is covered by the set of $D_{e} \times D_{f}$, where $e,f$ run though the idempotents $E(S)$, and is compact; thus only finitely many pairs $D_{e}\times D_{f}$ are necessary. The intersection $Gr(g) \cap D_{e}\times D_{f}$ is covered by $\bigcup_{u \in eS_{g}f} [u,D_{u^{*}u}]$ and so are compact if and only if:
\begin{equation*}
Gr(g) \cap D_{e}\times D_{f} = \bigcup_{u \in U}[u,D_{u^{*}u}]
\end{equation*}
for some finite $U \subset S_{g}$. However, this is precisely implied by the finite cover property.
\end{proof}

\begin{corollary}\label{Thm:Trick}
If $S$ is a 0-E-unitary monoid with the finite cover property then the groupoid $G(S)$ is Morita equivalent a transformation groupoid $Y \rtimes G$.\qed
\end{corollary}
\begin{proof}
This follows from Theorem 1.8 from \cite{MR1900993}.
\end{proof}

\begin{remark}
If, in addition the inverse monoid $S$ is 0-F-inverse, then it satisfies the finite cover property with $\vert U \vert=1$ as each $S_{g}$ will contain a unique maximal element.
\end{remark}

\subsection{Some remarks about the coarse groupoid.}

From earlier work in Chapter \ref{} any space that coarsely embeds into a group admits a grouplike partial translation structure. We equip the expander sequence coarsely embedded in our Gromov monster group with this translation structure. The results of Chapter \ref{} then tell us that there is a groupoid $\G(\mathcal{T})$, such that the translation algebra is isomorphic to the reduced groupoid  $C^{*}$-algebra $C^{*}_{r}(\G(\mathcal{T})$. 

Using Claim \ref{} the translations of $\mathcal{T}$ are the only elements we need to be understand when working with $\G(\mathcal{T})$. From the definition of $\mathcal{T}$ we can think of a translation structure as providing us an excellent generating set for the metric coarse structure on the space $X$; the groupoid $\G(\mathcal{T})$ acts freely on $\beta X$, and we can generate now the coarse groupoid using this data:

\begin{lemma}\label{Lem:CG}
The coarse groupoid $G(X) \cong \beta X \rtimes \G(\mathcal{T}).$
\end{lemma}
\begin{proof}(Short)
We observe that the set of $[t_{g},\widehat{D}_{t_{g}^{*}t_{g}}]$ covers $G(X)$; hence the collection $\mathcal{T}$ forms an admissible psuedogroup in the terminology of \cite{MR1905840}. The groupoid it generates is $\G(\mathcal{T})$. Then the result follows from \cite[Lemma 3.3b)]{MR1905840}.
\end{proof}


\begin{proposition}\label{Prop:Cocycle}
Let $X$ be a metric space and $G$ be a group such that $X$ is coarsely embedded into $G$. Then the translation groupoid $\G(\mathcal{T})$ admits a (T,C,F)-cocycle onto $G$.
\end{proposition}
\begin{proof}
This follows immediately from Proposition \ref{Thm:Trick} and Corollary \ref{Cor:Trick}.
\end{proof}

\section{Non-exactness of a Gromov Monster.}
We begin with a definition:
\begin{definition}
A finitely generated discrete group $\Gamma$ is a \textit{Gromov monster group} if there exists a large girth expander with vertex degree uniformly bounded above $X$ and a coarse embedding $f: X \hookrightarrow \Gamma$. 
\end{definition}

These groups were shown to exist by Gromov \cite{MR1978492}, with a detailed proof by Arzhantseva, Delzant \cite{exrangrps}. The construction is technical and we require no details beyond those presented in the definition.

The rest of this chapter is deadicated to proving the following theorem:

\begin{thm}\label{Thm:GM1}
Let $G$ be a Gromov monster group. Then there are locally compact Hausdorff topological $G$-spaces $Y_{i}$, $i \in \lbrace 1,2,3 \rbrace$ and a short exact sequence:
\begin{equation*}
0 \rightarrow C_{0}(Y_{1}) \rightarrow C_{0}(Y_{2}) \rightarrow C_{0}(Y_{3}) \rightarrow 0
\end{equation*}
such that
\begin{equation*}
0 \rightarrow C_{0}(Y_{1})\rtimes_{r} G \rightarrow C_{0}(Y_{2})\rtimes_{r} G \rightarrow C_{0}(Y_{3})\rtimes_{r} G \rightarrow 0
\end{equation*}
fails to be exact in the middle.
\end{thm}

We proceed first by anaylzing the situation from \cite[Section 8]{explg1}. The authors produce a commutative coefficient algebra for any Gromov monster group $G$ for which Baum-Connes conjecture $BC(G,A)$ fails to be surjective but is injective. 

We make precise the definition of a Gromov monster group that we will use for the remainder of the paper.

\begin{definition}
A finitely generated discrete group $\Gamma$ is a \textit{Gromov monster group} if there exists a large girth expander with vertex degree uniformly bounded above $X$ and a coarse embedding $f: X \hookrightarrow \Gamma$. 
\end{definition}

These groups were shown to exist by Gromov \cite{MR1978492}, with a detailed proof by Arzhantseva, Delzant \cite{exrangrps}. The construction is technical and we require no details beyond those presented in the definition.

In this section we connect the globalisation of the coarse groupoid with the ideas of Higson, Willett and Yu from \cite{higsonpreprint},explg1} concerning the $C^{*}$-algebraic construction of the coefficients for which a Gromov monster group fails to have the Baum-Connes assembly map an isomorphism.

We proceed first by anaylzing the situation from \cite[Section 8]{explg1}. The primary idea is to globalise  $C^{*}X$ in $C^{*}G$. Fix a left invariant proper metric on $G$.

Let $X_{n}:=N_{n}(X)\subset G$. Then we can form the $C^{*}$-algebras $\ell^{\infty}(X_{n}) \subseteq \ell^{\infty}(G)$. Being commutative algebras in this case, we could consider the dual picture by taking spectra, getting $C_{0}(\widehat{X_{n}}) \subset C(\beta G)$. It is clear that $X_{n} \subset X_{n+1}$, so the algebras $\ell^{\infty}(X_{n}) \subset \ell^{\infty}(X_{n+1})$. The remark here is that the inclusion of $X_{n} \subset G$ is not $G$-equivariant, but the system is $G$-equivariant; the action of $G$ on $X_{n}$ on the right by translations will send points in $X_{n}$ into a most $X_{n+l(g)}$ for each $g \in G$. Hence, the limit of the $\ell^{\infty}(X_{n})$ over $n$ is a $G$-algebra, and so we can form the semidirect product algebra $(\lim_{n}\ell^{\infty}(X_{n}))\rtimes_{r} G$. Lemma 8.4 from \cite{explg1} provides us the following isomorphisms:

\begin{lemma}\label{lem:GMG}
Let $X_{n}$ as above. Then $(\lim_{n}\ell^{\infty}(X_{n}))\rtimes G \cong \lim_{n} C^{*}_{u}(X_{n})$ and $(\lim_{n}\ell^{\infty}(X_{n},\mathcal{K}))\rtimes G \cong \lim_{n} C^{*}(X_{n})$.\qed
\end{lemma}

Let the coefficients $\lim_{n}\ell^{\infty}(X_{n},\mathcal{K})$ be denoted by $A$. We appeal to the fact that each $X_{n}$ is coarsely equivalent to $X$. As these limits are functorial in coarse maps, we conclude:

\begin{proposition}
Let $G$ be a Gromov monster group and $X$ the coarsely embedded large girth expander. Then we have $A\rtimes_{r} G \cong C^{*}X$.\qed
\end{proposition}

The procedure we will follow will be a geometric analogue of this argument using translation structures and Theorem \ref{Thm:IT2} of Milan and Steinberg, which relies on the information about the coarse groupoid given above as well as the fact the the inverse semigroups associated to the coarse groupoid are strongly 0-F-inverse.

The approach is via Theorem \ref{thm:PV2}:

\begin{theorem}
Let $G$ be a Gromov monster group. Then there are locally compact Hausdorff topological $G$-spaces $Y_{i}$, $i \in \lbrace 1,2,3 \rbrace$ and a short exact sequence:
\begin{equation*}
0 \rightarrow C_{0}(Y_{1}) \rightarrow C_{0}(Y_{2}) \rightarrow C_{0}(Y_{3}) \rightarrow 0
\end{equation*}
such that
\begin{equation*}
0 \rightarrow C_{0}(Y_{1})\rtimes_{r} G \rightarrow C_{0}(Y_{2})\rtimes_{r} G \rightarrow C_{0}(Y_{3})\rtimes_{r} G \rightarrow 0
\end{equation*}
fails to be exact in the middle.
\end{theorem}
\begin{proof}
To construct the complete sequence we use Lemma \ref{Lem:Cut} to get $Y_{1}:= (X \times G)/\sim$ and $Y_{3}:= (\partial\beta X \times G)/\sim$. We then get the short exact sequence of $G$-algebras:
\begin{equation*}
0 \rightarrow C_{0}(Y_{1}) \rightarrow C_{0}(Y_{2}) \rightarrow C_{0}(Y_{3}) \rightarrow 0.
\end{equation*}

Now we consider the crossed product algebras $C_{0}(Y_{i})\rtimes G$. Then the sequence above gives us some terms on K-theory: 
\begin{equation*}
\xymatrix@=1em{...\ar[r] & K_{0}(C_{0}(U)\rtimes G) \ar[r]& K_{0}(C_{0}(Z)\rtimes G) \ar[r]& K_{0}(C_{0}(Y)\rtimes G)\ar[r] & ...\\
...\ar[r] & K_{0}(\mathcal{K}) \ar[r]\ar[u]^{\ucong}& K_{0}(C^{*}_{r}(G(X))) \ar[r]\ar[u]^{\ucong}& K_{0}(C^{*}_{r}(G(X)|_{\partial\beta X})) \ar[r]\ar[u]^{\ucong}& ...}
\end{equation*}
And the bottom line is not exact on K-theory by either \cite{MR1911663}, \cite{explg1} or \cite{mypub1}. It follows therefore that the sequence:
\begin{equation*}
0 \rightarrow C_{0}(Y_{1})\rtimes_{r} G \rightarrow C_{0}(Y_{2})\rtimes_{r} G \rightarrow C_{0}(Y_{3})\rtimes_{r} G \rightarrow 0
\end{equation*}
is not exact in the middle term.
\end{proof}

This idea can be extended to connect this proof of failure to be exact to the geometric one that is outlined above from \cite{higsonpreprint,explg1}. 

We connect this geometric approach using groupoids to the analytic approach outlined in the previous section. 

\begin{proposition}\label{prop:GMG}
Let $X=X_{0}$ and $X_{n}$ as above. Then the globalisations of $G(X_{n})$ given by $B_{n}\rtimes G$ that come from the translation groupoid action of Lemma \ref{Lem:CG} are all Morita equivalent.
\end{proposition}
\begin{proof}
As $X_{0}$ is coarsely equivalent to $X_{n}$ for all $n$, it follows that $G(X)$ is Morita equivalent to $G(X_{n})$ for all $n$. Using Lemma \ref{Lem:CG} we can see that each of the groupoids $G(X_{n})$ is isomorphic to a transformation groupoid $\beta(X_{n}) \rtimes \G_{\X_{n})$ and therefore admits a (T,C,F)-cocycle onto the monster group $G$. Using Theorem \ref{Thm:IT2-a} (or Theorem \ref{Thm:IT2}) we can conclude that each $G(X_{n})$ is also Morita equivalent to $B_{n}\rtimes G$. Subsequently $B_{n}\rtimes G$ are Morita equivalent for each $n$, induced by the natural inclusions that extend $B_{n} \rightarrow B_{n+1}$.  
\end{proof}

Lemma \ref{lem:GMG} is naturally a corollary to Proposition \ref{prop:GMG}.

\subsection{Boundary Coefficients for a Gromov Monster.}

We extend the ideas in the previous section using the results from Chapter 4. In that chapter we proved that the boundary groupoid $G(X)|_{\partial\beta X}$ of a large girth sequence with uniformly bounded vertex degree decomposes as $\partial\beta X \rtimes \G_{\widehat{X}}$, where $\G_{\widehat{X}}$ has the Haagerup property. We extend these ideas by considering the impact this has on a Gromov monster group that contains such a large girth expander. To this end we prove:

\begin{theorem}\label{Thm:GM2}
There exists a locally compact Hausdorff space $Z$ such that the groupoid $Y_{3}\rtimes G$ is Morita equivalent to $Z \rtimes F_{k}$.
\end{theorem}

We recall the information from Chapter 4 that is required to prove this Theorem: (List of results, probably via bullet points)


\begin{lemma}\label{Lem:MEFree}
The boundary groupoid $G(X)|_{\partial\beta X}$ admits a (T,C,F)-cocycle onto $F_{k}$. 
\end{lemma}
\begin{proof}
We remark that this follows directly from the fact that the coarse boundary groupoid $G(X)|_{\partial\beta X}$ has a decompostion as $\partial\beta X \rtimes \G_{\widehat{X}}$, and that $S_{inf}$ is strongly $0$-F-inverse.
\end{proof}

\begin{proof}(Of Theorem \ref{Thm:MT2}).
Recall from the proof of Theorem \ref{Thm:MT1} that the groupoid $G(X)|_{\partial\beta X}$ is Morita equivalent to $Y_{3}\rtimes G$. Using Lemma \ref{Lem:MEFree} we know also that $Z:=(\partial\beta X \times F_{k} )/\sim$ is a locally compact Hausdorff space, arising from a (T,C,F)-cocycle onto $F_{k}$. This enables us to again appeal to either \cite[Theorem 6.14]{Milan-Steinberg} (Theorem \ref{Thm:IT2}) or \cite[Theorem 1.8]{MR1900993} (Theorem \ref{Thm:1.8}) to conclude that $G(X)|_{\partial\beta X}$ is Morita equivalent to $Z\rtimes F_{k}$. The Theorem then follows from transitivity of Morita equivalence.
\end{proof}

Theorem \ref{Thm:GM2} has an important corollary, as the Baum-Connes conjecture with coefficients is a Morita invariant:

\begin{corollary}
The Baum-Connes conjecture for $G$ with coefficients in any $(Y_{3}\rtimes G)$-$C^{*}$-algebra is an isomorphism.\qed
\end{corollary}