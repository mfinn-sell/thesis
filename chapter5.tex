%\setstretch{1.6}
\chapter{Applications and Connections.}
The focus of this chapter is developing ideas that appeared in Chapters 3 and 4 further as well as connecting these ideas together. Firstly, we consider some examples that arise from the short exact sequence of Chapter 3. Secondly, we outline a construction of a counterexample to the boundary conjecture; this space and its construction were first introduced in \cite{MR2363697} and its properties are developed further in this Chapter. Lastly, we connect the ideas of Chapter 3 and 4 together by outlining how the concept of a partial translation structure and associated groupoid can be used to describe why a Gromov monster group, a group that coarsely contains an expander, fails to be $C^{*}$-exact. 

\section{K-theory examples.}\label{sect:K-theory}

In this section we construct some examples, some well known in the literature, of inverse monoids associated to subspaces of groups. We then consider applications of the results outlined in the previous sections of the paper combined with a result of Norling \cite{Nor-2012} concerning the K-theory of $C^{*}_{r}(S)$, when $S$ is strongly 0-F-inverse. We first begin with a seemingly disconnected topological notion:

\begin{definition}
Let $X$ be a totally disconnected space. A set $\mathcal{V}$ is said to be a \textit{regular basis} for the topology of $X$ if:
\begin{enumerate}
\item $\mathcal{V}\cup \lbrace \emptyset \rbrace$ is closed under finite intersections; 
\item $\mathcal{V}$ generates the compact open sets of $X$;
\item $\mathcal{V}$ is independent, that is for every finite family $X,X_{1},...,X_{n} \in \mathcal{V}$ such that $X = \cup_{i=1}^{n} X_{i}$ there exists an $i \in \lbrace 1,..,n \rbrace$ such that $X=X_{i}$.
\end{enumerate}
\end{definition}

In \cite{CEL-2} the authors compute the K-theory for transformation groupoid $C^{*}$-algebras associated to actions of discrete groups $G$ on totally disconnected spaces $\Omega$ that carry a regular $G$-invariant basis. They rely on the Baum-Connes conjecture with coefficients for the group $G$ to deal with certain coefficient algebras via KK-theory. Furthermore, Norling \cite{Nor-2012} gave a proof that the basis $\lbrace \widehat{D}_{e} | e \in E \rbrace$ associated to a strongly $0$-F-inverse monoid $S$ is a regular basis of $\E$, and that the induced basis of $Y_{\E}$ is also regular and $G$-invariant. His proof requires an understanding of the following equivalence relation:

\begin{definition}
Let $e,f \in E(S)$. $e \approx f$ if $\exists s \in S$ such that $e \leq s^{*}s$ and $ses^{*}=f$.
\end{definition}

This equivalence relation captures information about the action of $S$ on its idempotents $E$, which is naturally used to construct the basis elements $D_{e}$ and the groupoid $\G_{\E}$, hence is intimately connected to the structure of $C^{*}_{r}(S)$. The main result of \cite{Nor-2012}, which utilises this relation, is stated below:

\begin{theorem}\label{Thm:Norling}
Let $S$ be a strongly 0-F-inverse monoid with universal group $G$, where $G$ has the Haagerup property. Then there is an isomorphism:
\begin{equation*}
K_{*}(C^{*}_{r}(S)) \cong \bigoplus_{[e]\in \frac{E^{\times}}{\approx}} K_{*}(C^{*}_{r}(G_{e})
\end{equation*}
Where $G_{e}$ is the stabiliser, in the conjugation action of $S$ on $E$, of the idempotent $e$
\end{theorem}

This Theorem gives a method for computing the K-theory for certain reduced semigroup $C^{*}$-algebras, in particular those constructed from partial translation structures arising from groups. We consider some general natural inverse monoids and compute both the K-theory groups and the associated long exact sequence that arises from the corresponding Pimnser-Voiculescu type sequence.

\subsection{The Examples.}
We consider the examples outlined in the introduction as well as other inverse monoids that we have introduced throughout the document.
\begin{example}\label{Ex:Toe}(Toeplitz extension)
Let $X=\mathbb{N} \subset \mathbb{Z}=G$. We arive at a partial translation structure for $X$ by considering the maps:
\begin{eqnarray*}
& t_{n}: \mathbb{N} \rightarrow \mathbb{N}, x \mapsto x+n \\
&t_{-n}: \mathbb{N} \rightarrow \mathbb{N}\setminus \lbrace 0,1,...,n-1 \rbrace , x \mapsto x-n
\end{eqnarray*} 
These partial bijections generate an inverse monoid, given by the presentation:
\begin{equation*}
S=\langle t_{n},t_{n}^{*}=t_{-n} | t_{n}^{*}t_{n}=1 \rangle
\end{equation*}
This is a well known example from semigroup theory called the \textit{Bicyclic Monoid}. It is well known also that the translation algebra is the universal algebra generated by a unilateral shift; the Toeplitz algebra. To understand the translation algebra, it is enough to compute $\X$. We know that $\X$ is the quotient of $\beta \mathbb{N}$ using the family of domains of the maps $t_{n}$ as $n$ runs though $\mathbb{Z}$, which are in particular cofinite.

\begin{claim}\label{prop:example}
If $E(S) \subseteq $ Cofin($X$) then $\X = (H \backslash X)^{+}$   
\end{claim}
\begin{proof}
It is enough to remark that in general we have:
\begin{equation*}
\X = H\backslash X \cup {1_{E}} \cup \lbrace \mbox{Filters arising from nonprincipal ultrafilters in } \beta X \rbrace
\end{equation*} If we assume $E(S) \subseteq $ Cofin($X$) then \textit{all} nonprincipal ultrafilters will agree in the quotient as they only fight over and subsequently differ on infinite subsets with infinite compliments.
\end{proof}

In this example the stabilizer $H$ is trivial, so $\X = \mathbb{N}^{+} = \E$. So we have that the quotient map $C^{*}_{r}(\G_{\E}) \rightarrow C^{*}_{r}(\G_{\X})$ is the identity and the translation algebra is also the Toeplitz algebra.

We remark that Theorem \ref{Thm:Norling} of Norling now gives a direct computation of the K-groups in this instance, as each idempotent is related to $1$ via the translation that sends $1$ to $n$ for each $n$. So it is enough to understand the stabiliser group $G_{1}$, which in this instance is trivial. Hence we get that the K-theory groups are those of a point, which is well-known although computed in a different manner \cite{MR587369,MR2457037}.
\end{example}

\begin{example}(Bridget-Rhodes expansion of a group)
In this instance there is much more interesting K-theory arising from Theorem \ref{Thm:Norling}. The Bridget-Rhodes expansion of a group \cite{MR745358,MR2221438} was first outlined in Section \ref{Sect:S3} as an example. We recall the construction again for clarity.

In the context of a group $G$ we define an a set $S(G)$, the elements are given by pairs: $(X,g)$ for $\lbrace 1,g\rbrace \subset X$, where $X$ is a finite subset of $G$. The set of such $(X,g)$ is then equipped with a product and inverse:
\begin{equation*}
(X,g)(Y,h) = (X\cup gY,gh)\mbox{ , } (X,g)^{-1}=(g^{-1}X,g^{-1})
\end{equation*}
This turns $S(G)$ into a inverse monoid with maximal group homomorphic image $G$, satisfying a universal covering property for partial $G$-actions. The partial order on $S(G)$ can be described by reverse inclusion, induced from reverse inclusion on finite subsets of $G$. It is F-inverse, with maximal elements: $\lbrace(\lbrace 1,g \rbrace, g):g \in G \rbrace$. 

Using Theorem \ref{Thm:Norling}, provided the group has the Haagerup property, we can again compute the K-theory, each finite subset $F$ of $G$ containing $1$ admits the partial action by the group elements that arise within the finite subset; any finite subgroup occurs this way and this gives us all the possible stabilisers. Denote the subset of finite subgroups by $FSG(G)\subset Fin(G)_{1}$. 

The main outcome of this is the following calculation:

\begin{equation*}
K_{*}(C^{*}_{r}(S)) \cong \bigoplus_{\substack{[F] \\ F \in Fin(G)_{1}}} K_{*}(C^{*}_{r}(G_{F})) \cong \bigoplus_{\substack{[F] \\F \in FSG(G)^{c}} }\mathbb{Z} \oplus \bigoplus_{\substack{[F] \\F \in FSG(G) }}K_{*}(C^{*}_{r}(F))
\end{equation*}

In the light of Theorem \ref{thm:PV1} this suggests it should be possible, using the long exact sequence, to compute the K-theory of $C^{*}_{r}(G)$ from information about its finite subgroups.

If $G$ is finite then it appears as an element in $FSG(G) \subset Fin(G)_{1}$ and so the sequence provided by Theorem \ref{thm:PV1} will split at the level of K-theory. Additionally, if we assume $G \cong C_{p}$, the cyclic group of order $p$ for some prime $p$ it is possible to compute the size of the index sets that occur. For each $n$, the number of elements of $Fin(C_{p})_{1}$ of cardinality $n$ is ${p-1 \choose n}$ and so the cardinality of $E(S)$ will be $\sum_{n\geq 1} {p-1 \choose n-1} = F_{p-1}$ the $p-1$ term in the Fibonacci sequence. 

To compute the index we need to know how many of these are related via the partial action of $C_{p}$ and each subset of cardinality $n$ can be related to $n$ other elements: hence the number of orbits of subsets of size $n$ is precisely $\frac{1}{n} {p-1 \choose n-1}$. Let $k_{p}:= 1+ \sum_{n>0}\frac{1}{n} {p-1 \choose n-1}$ whence we get:

\begin{equation*}
K_{*}(C^{*}_{r}(S(C_{p})) \cong \bigoplus_{k_{p}}\mathbb{Z} \oplus K_{*}(C^{*}_{r}(C_{p}))
\end{equation*}

We remark that it is possible to arrive at this inverse monoid in a natural way via a subset of a group known as a universal deep set \cite{BNW-KTA}. This subset is universal for partial translation structures, and that it generates this inverse monoid is immediate. However to compute the K-theory of this translation algebra is more complicated than for the reduced $C^{*}$-algebra.

\end{example}

We now shift our considerations to the free group on two generators. In this setting, it is possible to get inverse monoids coming from translation structures that are richer in interesting behaviour. We outline some of their natural properties:

\begin{claim}
Let $X \subset F_{2}$ be connected and let $g$ and $h$ be words in $F_{2}$ such that $g$ does not end in $a_{i}^{\pm 1}$ and $h$ does not start with $a_{i}^{\mp 1}$. Then the the translations $\lbrace t_{g} | g \in F_{2} \rbrace$ satisfy $t_{g}t_{h}=t_{gh}$. 
\end{claim}
\begin{proof}
These translations differ only by the fact that the product $t_{g}t_{h}$ may contain a relation that is unreduced, whereas $t_{gh}$ acts by the reduced form of $gh$. As there are no relations other than $a_{i}a_{i}^{-1}$ or $a_{i}^{-1}a_{i}$ in $F_{2}$ whence $t_{g}t_{h}$ and $t_{gh}$ agree everywhere they are defined.
\end{proof}

This property makes working with translation algebras arising from $F_{2}$ easier.

\begin{example}\label{ex:PV}(A free group via the Pimsner-Voiculescu method)
Let $X$ be the subset of the free group $F_{n}=\langle a_{1},...,a_{n} \rangle$ consisting of all the words that do not start with an $a_{1}^{-1}$. This subset was considered in \cite{MR670181} and gave rise to a short exact sequence:
\begin{equation*}
0 \rightarrow \mathcal{K}(\ell^{2}(X)) \rightarrow C^{*}\mathcal{T}_{n} \rightarrow C^{*}_{r}F_{n} \rightarrow 0.
\end{equation*}
This sequence is the translation algebra sequence that arises from Theorem \ref{thm:T4}. In addition to this sequence the authors of \cite{MR670181} gave a computation of the K-theory groups associated to $C^{*}\mathcal{T}_{n}$ as those for $C^{*}_{r}F_{n-1}$, giving an inductive method for computing the K-theory of a free group $C^{*}$-algebra. We give a new proof using the generalised short exact sequence from Theorem \ref{thm:PV1} and Theorem \ref{Thm:Norling}.

For this subspace the behavour is exceptionally like the Toeplitz shift in Example \ref{Ex:Toe}, however it is decidedly more complex than in that instance.

\begin{claim}
Let $w \in F_{n}$. Then $t_{w}^{*}t_{w}\approx 1$.
\end{claim}
\begin{proof}
We will prove this by induction on the length of $w$. This clearly holds for the case when the length of $w$ is $1$. So now, assume this holds for length equal to $n$ and let $w$ have length $n+1$. Consider $t_{w}^{*}t_{w}$; this will be a word of length $2n+2$ containing a relation of the form in the centre $t_{a_{i}^{\pm 1}}^{*}t_{a_{i}^{\pm 1}}$ for some $i$. If $i$ is not 1, and this word is not $t_{a_{1}}t_{a_{1}}^{*}$ then we can reduce the idempotent in length to $2n$, represented by the word $w_{1}$ with left end removed. So, we can suppose that the centre of $t_{w}^{*}t_{w}$ is $t_{a_{1}}t_{a_{1}}^{*}$. Now, if the outer translation is a $t_{a_{i}^{\pm 1}}$ for $i$ not $1$, then it is a bijection, and we can apply this bijections inverse without disrupting the equivalence relation $\approx$. This again reduces the length of the word by $1$. Finally, suppose that $w$ starts and ends with $a_{1}^{-1}$. Then $t_{w}^{*}t_{w}$ is certainly less than $t_{a_{1}}t_{a_{1}}^{*}$, at which point we can conjugate by $t_{a}$, preserving the relation $\approx$ and reducing our idempotent in length by $2$. 
\end{proof}

This however does not let us compute for a general $e$, which will be a product of conjugates of the $t_{w}^{*}t_{w}$.

Putting this together with the long exact sequence in K-theory connecting the short exact sequences from Theorem \ref{thm:PV1} and Theorem \ref{thm:T4} gives us the following diagram of K-theory groups:

$$
\xymatrix@=0.7em{\ar[r] & 0 \ar[r]\ar[d] & K_{0}(C^{*}_{r}(\G_{U\cap \widehat{X}^{c}}))\ar[d] \ar[r]^{\cong}& K_{0}(\ker p) \ar[r]\ar[d]& 0 \ar[r]\ar[d] & K_{1}(C^{*}_{r}(\G_{U\cap \widehat{X}^{c}})) \ar[r]\ar[d]^{\ucong} &\\
\ar[r] & K_{1}(C^{*}_{r}(F_{n})) \ar[r]\ar[d]^{\ucong} & K_{0}(C^{*}_{r}(\G_{U}))\ar[r]\ar[d]& K_{0}(C^{*}_{r}(F_{n-1})) \oplus \bigoplus_{[ef]_{\approx}} K_{*}(\mathbb{C})\ar[r]\ar[d]^{p}& K_{0}(C^{*}_{r}(F_{n}))\ar[r]\ar[d]^{\ucong} & K_{1}(C^{*}_{r}(\G_{U}))\ar[d] \ar[r] & \\
\ar[r] & K_{1}(C^{*}_{r}(F_{n})) \ar[r]& K_{0}(\mathcal{K}(\ell^{2}(X))) \ar[r]& K_{0}(C^{*}\mathcal{T}_{n}) \ar[r]& K_{0}(C^{*}_{r}(F_{n})) \ar[r]& 0 \ar[r] & 
}
$$

This indicates that whilst the translation algebra $C^{*}\mathcal{T}_{n}$ does not have the same K-theory as $C^{*}_{r}(S)$, the K-theory sequence is split (this relies on the results of Pimsner-Voiculescu \cite{MR670181}: $K_{*}(C^{*}\mathcal{T}_{n}))\cong K_{*}(C^{*}_{r}(F_{n-1}))$), so we are picking out the correct K-theory group as well as much more complex information that comes from the product structure on the idempotents. This is connected to regularisation of the generating set for the basis of $\E$ \cite{CEL-2}, and will be discussed later in the section.

\begin{example}(Polycyclic monoids, Strong orthogonal completions and the Cuntz extension)
Let $X$ be the set of all the positive words within the free group $F_{n}$. We consider the inverse monoid that arises from the induced translation structure. In this case the inverse monoid has a zero element and satisfies the relations: $t_{a_{i}}^{*}t_{a_{j}}=\delta_{ij}$, $t_{a_{i}}t_{a_{i}}^{*} \leq 1$. The inverse monoid satisfying this relation is called the \textit{polycyclic monoid} \cite{MR2372319}, which is a generalisation of the Bicyclic monoid. We denote this inverse monoid by $P_{n}$.


In this case we again have that $E(P_{n})$ is in bijection with $X$, which is a rooted binary tree. Hence $\E(P_{n}) \cong \X$. However, many of the domains of translation are infinite with infinite compliment and so we have nontrivial ultrafilters to consider. In general we should consider tight filters, as the closure of the ultrafilters, but by work of Lawson \cite{lawson-2011-1} $P_{n}$ is compactible for each $n$, that is the ultrafilters are closed in the subspace topology on $\E$. As a free group is exact, we can appeal to Theorem \ref{thm:PV2} to get a short exact sequence:
\begin{equation*}
0 \rightarrow C^{*}_{r}(\G_{U}) \rightarrow C^{*}_{r}(\G_{\E}) \rightarrow C^{*}_{r}(\G_{\E_{\infty}}) \rightarrow 0.
\end{equation*}
We would like to understand the algebras that appear within this sequence. Let us begin with the first term; as the left stabiliser of this subspace is trivial we can deduce from Proposition \ref{cor:C4} that $\G_{U}$ is a pair groupoid, hence $C^{*}_{r}(\G_{U})$ is the compact operators on $\ell^{2}(X)$. The middle term satisfies the relation $\sum_{i=1}^{n}t_{a_{i}}t_{a_{i}}^{*} \leq 1$, whence $C^{*}_{r}(\G_{\E}) \cong C^{*}_{r}(S)$ admits a map to $E_{n}$, the generalised Cuntz algebra. It is well known that this map is an isomorphism \cite{MR1724106,MR584266}. It now follows that $C^{*}_{r}(\G_{\E_{\infty}})$ is isomorphic to the Cuntz algebra $\mathcal{O}_{n}$.

We again appeal to Theorem \ref{Thm:Norling} to compute the K-theory. By Proposition \ref{prop:P10} we know that the action on $E(P_{n})$ translates, via the bijection onto $X$, to the translation action of $F_{n}$ on $X$. It follows that there is only a single orbit under this action as partial translation actions are transitive. The stabiliser is obviously trivial in this case, hence by Theorem \ref{Thm:Norling} we arrive at the computation $K_{*}(C^{*}_{r}(P_{n})) \cong K_{*}(\mathbb{C})\cong \mathbb{Z}$. All that remains is to compute the maps in the sequence, which are also well known..

We give a direct computation of the K-theory of the final term here, by considering Lawsons \textit{orthogonal completions} of $P_{n}$ \cite{lawson-2011-1}, denoted by $D(P_{n})$ and $C(P_{n})$ respectively.

\begin{definition}
Let $E$ be a semilattice and let $e,f \in E$. $e$ is \textit{dense} in $f$ if $e \leq f$ and there does not exist $z \in E$ such that $z \leq f$ and $ze=0$. 
\end{definition}

We remark that any tight representation of a inverse monoid cannot separate dense idempotents \cite{MR2419901}. This is particularly relevent in this example. It is clear that in $P_{n}$ the elements $\lbrace t_{a_{i}}t_{a_{i}}^{*} | i = 1,..,n \rbrace$ are pairwise orthogonal, and in the reduced $C^{*}$-algebra they have sum that is less than $1$. The idea of the orthogonal completion is to capture this $C^{*}$-algebraic behavour in an inverse monoid; in $D(P_{n})$ the sum $\bigvee_{i=1}^{n} t_{a_{i}}t_{a_{i}}^{*}$ is defined and is dense in $1$, and equal to $1$ in $C(P_{n})$, which has an underlying tight representation of $S$. 

It follows, from von Neumann equivalence of projections, that each $t_{a_{i}}t_{a_{i}}^{*}$ viewed as an operator in $C^{*}_{r}(\G_{\E_{\infty}})$ is equivalent to $1$ at the level of K-theory, in paricular using Proposition 5.3.1 and Lemma 5.3.2 from \cite{MR1222415} we observe that $C^{*}_{r}(\G_{\E_{\infty}})$ is stable and that $[1]=\sum_{i=1}^{n}[1]$. From above, we know that the K-theory group $K_{0}(C^{*}_{r}(P_{n}))$ is generated by the class $[1]$, hence we know that $[1]$ generates $K_{0}(C^{*}_{r}(\G_{\E_{\infty}})$ also. It follows that $\sum_{i=1}^{n-1}[1]=[0]$, and this gives the desired isomorphism onto $C_{n-1}$.

\end{example}

\subsection{What happens for the partial translation structure reduction in general?}

In Example \ref{ex:PV} we observed that the inverse monoid generated by the translation structure had K-theory groups that were relatively easy to calculate but much too large. This phenomenon is not uncommon; the same computation using the subspaces present in the work of Lance \cite{MR723010} also provide too rich a structure. This additional structure arises as the basis for topology on $\E$ that we are using to apply results of Norling and Cuntz-Echerhoff-Li rely on the \textit{regular} basis property. We also observe that the natural elements that contribute to the correct K-theory groups in these instances are precisely the idempotents $t_{w}^{*}t_{w}$ that arise from words $w \in F_{n}$, as opposed to their products. This essentially says that considering the generating set over the regular basis it generates appears to give the correct answer.

We also remark that the large diagram constructed in Example \ref{ex:PV} can be constructed for \textit{any} translation structure. To apply Theorem \ref{Thm:Norling} however we restrict to discrete groups with the Haagerup property. So we have the following diagram:

$$
\xymatrix@=0.7em{ 0 \ar[r]\ar[d] & K_{0}(C^{*}_{r}(\G_{U\cap \widehat{X}^{c}}))\ar[d] \ar[r]^{\cong}& K_{0}(\ker p) \ar[r]\ar[d]& 0 \ar[r]\ar[d] & \\
K_{1}(C^{*}_{r}(G)) \ar[r]\ar[d]^{\ucong} & K_{0}(C^{*}_{r}(\G_{U}))\ar[r]\ar[d]&\bigoplus_{w \in G} K_{0}(C^{*}_{r}(G_{t_{w}^{*}t_{w}})) \oplus \bigoplus_{[ef]_{\approx}} K_{*}(C^{*}_{r}(G_{ef}))\ar[r]\ar[d]^{p}& K_{0}(C^{*}_{r}(G))\ar[r]\ar[d]^{\ucong} &  \\
K_{1}(C^{*}_{r}(G)) \ar[r]& K_{0}(\mathcal{K}(\ell^{2}(X))) \ar[r]& K_{0}(C^{*}\mathcal{T}) \ar[r]& K_{0}(C^{*}_{r}(G)) \ar[r]&  
}
$$

\begin{question}
Is the middle column split in both dimensions?
\end{question}

A positive answer to that question would give us a positive answer to the following:

\begin{question}
Is $K_{*}(C^{*}\mathcal{T}) \cong \bigoplus_{w \in G} K_{0}(C^{*}_{r}(G_{t_{w}^{*}t_{w}}))$; Are the domains and ranges of the $t_{w}$ enough to get a direct computation of the K-theory?
\end{question}

\section{A counterexample to the Boundary Conjecture.}
In this section we develop the ideas of Higson, Lafforgue and Skandalis concerning the counterexamples to the coarse Baum-Connes conjecture further, to construct a metric space $Y$ that has exceptional properties at infinity. The main idea is to decompose the boundary groupoid further, giving a new short exact sequence at infinity similar to the sequences considered in Chapter 4. From this, we then construct an operator that is not a ghost operator, but is ghostly on certain parts of the boundary. A tracelike argument, similar to those of \cite{higsonpreprint, explg1} then allows us to conclude that the boundary coarse Baum-Connes conjecture fails to be surjective for the space $Y$.

\subsection{The space and its non-ghosts.}

The space we are going to consider first appeared in \cite{MR2363697}.

Let $\lbrace X_{i} \rbrace_{i \in \mathbb{N}}$ be a sequence of finite graphs. Then we construct a space similar to a space of graphs in the following manner: Let $Y_{i,j} = X_{i}$ for all $j \in \mathbb{N}$ and consider $Y:= \sqcup_{i,j} Y_{i,j}$. We metrize this space using a box metric - that is with the property that $d(Y_{i,j},Y_{k,l}) \rightarrow \infty$ as $i+j+k+l \rightarrow \infty$. 

Now let $\lbrace X_{i} \rbrace_{i}$ be an expander sequence. As discussed in Section \ref{Sect:GO} of Chapter 4, we can construct a ghost operator $p= \prod_{i} p_{i}$ on $X$, the space of graphs of $\lbrace X_{i} \rbrace_{i}$. Similarly, we can construct this operator on $Y$. In this situation we get a projection $q:=\prod_{i,j}p_{i} \in C^{*}_{u}Y$, which is a constant operator in the $j$ direction. This was precisely the operator of interest in \cite{MR2363697}, as it can be seen that $q$ is not a ghost operator, as its matrix entries do not vanish in the $j$ direction - a fact proved below in Lemma \ref{Lem:nag}.

Recall that associated to $Y$ we have a short exact sequence of $C^{*}$-algebras:
\begin{equation*}
\xymatrix{
0 \ar[r] & ker(\pi) \ar[r]& C_{r}^{*}(G(Y)) \ar[r]^{\pi} & C_{r}^{*}(G(Y)|_{\partial\beta Y}) \ar[r] & 0.
}
\end{equation*}

We remark the kernel, $ker(\pi)$ consists of all the ghost operators in $C^{*}_{u}(Y)$, that is those operators with matrix coefficients that tend to $0$ in all directions on the boundary. 

\begin{lemma}\label{Lem:nag}
The projection $\pi(q):= \prod_{i,j}p_{i} \not = 0 \in C^{*}_{r}(G(Y)|_{\partial\beta Y})$. That is $q \not\in ker(\pi)$.
\end{lemma}
\begin{proof}
We first observe that every bounded subset $B$ of $Y$ is contained in some rectangle of the form $R_{i_{B},j_{B}}:=\sqcup_{i\leq i_{B},j\leq j_{B}}Y_{i,j}$. So to prove that $q$ is not a ghost operator it suffices to show that there exists an $\epsilon>0$ such that for all rectangles $R_{i,j}$ there is a pair of points $x,y$ in the compliment of the rectangle such that  the norm $\Vert q_{x,y} \Vert \geq \epsilon$. To prove this, recall that the projection $q$ is a product of projections $p_{i}$ on each $X_{i}$ and fixing $j$, these projections form a ghost operator. 

Fix $\epsilon = \frac{1}{2}$. Then there exists an $i_{\epsilon}$ with the property that $\forall i>i_{\epsilon}$ and for every $x,y \in \sqcup_{i}X_{i}$ we know $\Vert p_{i,x,y} \Vert < \epsilon$. We remark that this $i_{\epsilon}$ can be taken to be the smallest such. So for $i \leq i_{\epsilon}-1$, we have that $\Vert p_{i,x,y} \Vert \geq \frac{1}{2}$. Now let $R_{i_{\epsilon}-1,\infty}$ be the vertical rectangle $\sqcup_{i\leq i_{\epsilon}-1,j} Y_{i,j}$. 

To finish the proof, consider an arbitrary finite rectangle $R_{i,j}$. This intersects the infinite rectangle $R_{i_{\epsilon}-1,\infty}$ in a bounded piece. Now pick any pair of points in a fixed box $x,y \in Y_{k,l} \subset R_{i_{\epsilon}-1,\infty} \setminus R_{i,j}$. Then for those points $x,y$ it is clear that $\Vert q_{x,y} \Vert = \Vert p_{k,x,y}\Vert \geq \frac{1}{2}$.
\end{proof}

We now describe the boundary $\partial\beta Y$. We are aiming at a decomposition into saturated pieces and with that in mind we construct a map to $\beta X$.

Consider the map $\beta Y \twoheadrightarrow \beta X \times \beta \mathbb{N}$ induced by the bijection of $Y$ with $X \times \mathbb{N}$ and the universal property of $\beta Y$. Now define:
\begin{equation*}
f: \beta Y \rightarrow \beta X \times \beta \mathbb{N} \rightarrow \beta X
\end{equation*}
The map $f$ is continuous, hence the preimage of $X$ under projection onto the first factor is an open subset of $\beta Y$, which intersects the boundary $\partial \beta Y$. In fact, what we can see is that $f^{-1}(X)= \sqcup f^{-1}(X_{i})$, where each $f^{-1}(X_{i})$ are closed, and therefore homeomorphic to $X_{i} \times \beta \mathbb{N}$. We can define $U = f^{-1}(X)\cap \partial\beta Y$.

\subsection{The boundary groupoid associated to the box space of a discrete group with the Haagerup property.}

Let $\Gamma$ be a finitely generated residually finite discrete group with the Haagerup property, and let $\lbrace N_{i}\rbrace$ be a family of nested finite index subgroups with trivial intersection. Let $X_{i}:=Cay(\frac{\Gamma)}{N_{i}})$. In this context, the boundary groupoid is generated by the action of the group $\Gamma$ extended to the boundary (see Proposition \ref{Prop:Crit}). In this context we can show that $U$ defined above is saturated:

\begin{lemma}
$U$ is an open, saturated subset of the boundary $\partial\beta Y$. 
\end{lemma}
\begin{proof}
We have already shown above that $U$ is open. To see it is saturated we prove that $U^{c}$ is saturated, observe that the following diagram commutes:
\begin{equation*}
\xymatrix{
\overline{g}_{Y}:\beta Y\ar[r]\ar[d]^{p} & \beta Y\ar[d]^{p}\\
\overline{g_{X} \times 1}:  \beta X \times \beta \mathbb{N} \ar[r] & \beta X \times \beta \mathbb{N}
}
\end{equation*}
where these maps extend the group action on $Y=X \times \mathbb{N}$. The projection onto $\beta X$ is also equivariant under this action. Assume for a contradiction that $U^{c}$ is not saturated; there exists $\gamma$ in $U^{c}$ such that $\overline{g}_{Y}(\gamma) \in U$. It follows that $\overline{g_{X} \times 1}(p(\gamma))$ is in $p(U)$, whilst $p(\gamma) \in p(U^{c})$, hence $\overline{g_{X}}(f(\gamma))\in U$ whilst $f(\gamma) \in U^{c}$. This is a contradiction as $f(U^{c}) = \partial\beta X$ is saturated.
\end{proof}

This gives us two natural complimentary restrictions of $G(Y)|_{\partial\beta Y}$ and a short exact sequence of function algebras as in Chapter 4:
\begin{equation*}
\xymatrix{
0 \ar[r] & C_{c}(G(Y)|_{U}) \ar[r]& C_{c}(G(Y)|_{\partial\beta Y}) \ar[r] & C_{c}(G(Y)|_{U^{c}}) \ar[r] & 0.
}
\end{equation*}

We will now show that the corresponding sequence:
\begin{equation*}
\xymatrix{
0 \ar[r] & C^{*}_{r}(G(Y)|_{U}) \ar[r]& C_{r}^{*}(G(Y)|_{\partial\beta Y}) \ar[r]^{h} & C_{r}^{*}(G(Y)|_{U^{c}}) \ar[r] & 0
}
\end{equation*}
fails to be exact in the middle. We proceed as in \cite{explg1,MR1911663} by using the element $\pi(q)$, which certainly vanishes under the quotient map from $C^{*}_{r}(G(Y)|_{\partial\beta Y}) \rightarrow C^{*}_{r}(G(Y)|_{U^{c}})$. To show the failure we will show this sequence fails to be exact in the middle at the level of K-theory and for this we will require a firm understanding of the structure of $G(Y)|_{U}$.

We observe the following facts: 
\begin{enumerate}
\item $\Gamma$ acts on the space $Y:=\sqcup_{i,j}Y_{i,j}$ built from $\lbrace X_{i} \rbrace$.
\item This action becomes free on piece of the boundary that arises as $i \rightarrow \infty$, that is $\Gamma$ acts freely on $U^{c}$.
\item The group action generates the metric coarse structure on the boundary; the finite sets associated to each $R>0$ in the decomposition are now finite rectangles. This follows from considerations of the metric on $Y$.
\end{enumerate} 

It follows from the proof of Proposition \ref{Prop:Crit} that the groupoid $G(Y)|_{U^{c}}$ is isomorphic to $U^{c}\rtimes \Gamma$ and under the assumption that $\Gamma$ has the Haagerup property we can conclude that the Baum-Connes assembly map for the groupoid $G(Y)|_{U^{c}}$ is an isomorphism (with any coefficients). We now concern ourselves with $G(Y)|_{U}$.

\begin{lemma}\label{Lem:CE3}
The groupoid $G(Y)|_{U}$ is isomorphic to $\sqcup_{i}(X_{i}\times X_{i})\times G(\mathbb{N})|_{\partial\beta \mathbb{N}}$.
\end{lemma}
\begin{proof}
The reductions to the inclusions of the preimages $f^{-1}(X_{i})$ restricted to the boundary are isomorphic to the closed subgroupoids $G(X_{i}\times \mathbb{N})|_{\partial\beta \mathbb{N}}$ of $G(Y)|_{U}$. These groupoids are disjoint by construction and therefore the inclusion $\sqcup_{i}G(X_{i}\times \mathbb{N})|_{\partial\beta \mathbb{N}}$ is an open subgroupoid of $G(Y)|_{U}$. We now prove that:
\begin{enumerate}
\item each $G(X_{i} \times \mathbb{N})|_{\partial\beta \mathbb{N}}$ is isomorphic to $(X_{i}\times X_{i})\times G(\mathbb{N})|_{\partial\beta \mathbb{N}}$;
\item the union $\sqcup_{i}G(X_{i}\times \mathbb{N})|_{\partial\beta \mathbb{N}}$ is the entire of $G(Y)|_{U}$.
\end{enumerate}
To prove (1), observe that the groupoid decomposes as $G(X_{i}\times \mathbb{N}) = \bigcup_{R>0}\overline{\Delta_{R}(X_{i}\times \mathbb{N})}$. For each $R>0$ we can find a $j_{R}$ such that $\Delta_{R}(X_{i}\times \mathbb{N}) = F_{R} \cup \bigcup_{j>j_{R}}\Delta_{R}^{j}(X_{i}\times \mathbb{N})$, hence for the boundary part of this groupoid it is enough to understand what happens in each piece $Y_{i,j}$, which is constant for each $j$. Secondly, observe that in the induced metric on a column, the pieces $Y_{i,j}$ separate as $j\rightarrow \infty$. This, coupled with the fact that for large enough $R$, we know that $ \Delta^{j}_{R}(X_{i}\times \mathbb{N}) = X_{i} \times X_{i}$ allow us to deduce that any behavour at infinity of this groupoid is a product of $X_{i} \times X_{i}$ and the boundary groupoid $G(\mathbb{N})|_{\partial\beta \mathbb{N}$ where $\mathbb{N}$ has the coarsely disconnected metric. This groupoid is isomorphic to $\partial\beta \mathbb{N}$, from which we can deduce that $G(X_{i}\times \mathbb{N})|_{\partial\beta \mathbb{N}} = (X_{i}\times X_{i})\times \partial\beta \mathbb{N}$ for each $i$.

To prove (2) we assume for a contradiciton that there is a partial translation $t$, such that $\overline{t}$ is not an element of the disjoint union. Such an element maps some $(x_{i}, \omega)$ to $(x_{k},\omega)$, where $i\not =k$. Without loss of generality assume also $t$ has translation length at most $R$. Then the domain and range of $t$ are both infinite (as the closure is defined in $G(Y)|_{U}$), and must be contained within a strip of width at most $R>0$. From the definition of the metric, there are only finitely many $Y_{i,j}$ within such a rectangle, hence $t \in F_{R}$ and hence $\overline{t}$ is not defined in $G(Y)|_{U}$, which yields a contradiction.
\end{proof}

\begin{remark}
Lemma \ref{Lem:CE3} allows us to conclude that $C^{*}_{r}(G(Y)|_{U}) \cong \bigoplus_{i}M_{\vert X_{i}\vert} \otimes C(\partial\beta Y)$
\end{remark}

To conclude that $[\pi(q)]$ is not an element of $K_{0}(C^{*}_{r}(G(Y)|_{U}))$ we construct a trace-like map.

\begin{theorem}
The element $h_{*}[\pi(q)]=0$, but does not belong to $K_{0}( C^{*}_{r}(G(Y)|_{U}))$.
\end{theorem}
\begin{proof}
We begin the proof by remarking that each $U_{i}:=f^{-1}(X_{i}) \cap U$ is a closed saturated subset of $\partial\beta Y$, hence we can consider the reduction to $U_{i}$ for each $i$. We consider the product, and the following map:
\begin{eqnarray*}
\phi : C^{*}_{r}(G(Y)|_{\partial\beta Y})& \rightarrow & \prod_{i} C^{*}_{r}(G(Y)|_{U_{i}})= \prod_{i}C^{*}_{r}(G(X_{i} \times \mathbb{N})\\
 T  &\mapsto & \prod_{i}T|_{U_{i}}
\end{eqnarray*}
Under the map $\phi$, the subalgebra $C^{*}_{r}(G(Y)|_{U})= \bigoplus_{i}M_{\vert X_{i}\vert} \otimes C(\partial\beta Y)$ maps to the ideal $\bigoplus_{i}C^{*}_{r}(G(X_{i} \times \mathbb{N}))$. So, we can define a tracelike map, in analogy to \cite[Section 6]{explg1}, by composing with the quotient map $\tau$ onto $\frac{\prod_{i}C^{*}_{r}(G(X_{i} \times \mathbb{N}))}{\bigoplus_{i}C^{*}_{r}(G(X_{i} \times \mathbb{N}))}$. This gives us a map at the level of K-theory:
\begin{equation*}
Tr_{*}=\phi \circ \tau : K_{0}(C^{*}_{r}(G(Y)|_{\partial\beta Y})) \rightarrow \frac{\prod_{i}K_{0}(C^{*}_{r}(G(X_{i} \times \mathbb{N})))}{\bigoplus_{i}K_{0}(C^{*}_{r}(G(X_{i} \times \mathbb{N})))}= \frac{\prod_{i}K_{0}(C(\partial\beta \mathbb{N}))}{\bigoplus_{i}K_{0}(C(\partial\beta \mathbb{N}))}
\end{equation*}

We now consider $[\pi(q)]$ under $Tr_{*}$. Recall that $q=\prod_{i,j}p_{i}$. We define $q_{i}=\prod_{j}p_{i}$ and observe that the operation of reducing to $G(Y)|_{U_{i}}$ can be performed in two commuting ways: restricting to $U$ then $f^{-1}(X_{i})$ or by restricting to $f^{-1}(X_{i})$ then $U$. The second tells us that $q_{i}=p_{i} \otimes 1_{\beta\mathbb{N}$ is constant in the $j$ direction and when restricted to the boundary is $\pi(q_{i})=p_{i}\otimes 1_{\partial\beta \mathbb{N}}$. Hence, $Tr_{*}([\pi(q)])=[1_{\partial\beta \mathbb{N}},1_{\partial\beta\mathbb{N}},...] \not = 0$ and so $[\pi(q)] \not \in K_{0}(C^{*}_{r}(G(Y)|_{U})$.
\end{proof}

So in this case we have the following diagram:
\begin{equation*}
\xymatrix@=0.7em{
 K_{1}(C(U^{c})\rtimes \Gamma) \ar[r] & K_{0}(\ker (\pi)) \ar[r]& K_{0}(C^{*}_{r}(G(Y)|_{\partial\beta Y})) \ar[r]& K_{0}(C(U^{c})\rtimes \Gamma)\ar[r] & K_{1}(\ker (\pi))  \\
 K_{1}^{top}(U^{c}\rtimes \Gamma) \ar[r] \ar[u]^{\ucong}& K_{0}^{top}(G(Y)|_{U}) \ar[r]\ar@{^{(}->}[u]\ar@{^{(}->}[ru]& K_{0}^{top}(G(Y)|_{\partial\beta Y}) \ar[r]\ar[u]^{\mu_{bdry}}& K_{0}^{top}(U^{c}\rtimes \Gamma) \ar[r]\ar[u]^{\ucong}& K_{1}^{top}(X \times X)\ar[u]
}
\end{equation*}

\begin{remark}
We justify the diagonal inclusion of $K_{0}^{top}(G(Y)|_{U})$ into $K_{0}(C^{*}_{r}(G(Y)|_{\partial\beta Y})$. This follows as the groupoid $G(Y)|_{U}$ is nuclear, and hence the assembly map is an isomorphism.   The algebra $C^{*}_{r}(G(Y)_{U})= \bigoplus_{i}M_{\vert X_{i}\vert} \otimes C(\partial\beta Y)$ injects into the product $\prod_{i} M_{\vert X_{i}\vert} \otimes C(\partial\beta Y)$ at the level of K-theory and this inclusion factors through in the inclusion into the kernel of $\pi$ and into $C^{*}_{r}(G(Y)|_{\partial\beta Y})$. These maps provide enough information to conclude injectivity of the assembly map $\mu_{bdry}$.
\end{remark}

A diagram chase under the assumption that the map $\mu_{bdry}$ is surjective quickly yields a contradiction, whence we have:

\begin{corollary}
The assembly map $\mu_{bdry}$ associated to $Y=\sqcup_{i,j}Cay(\frac{\Gamma}{N_{i}})$ is not surjective but is injective.\qed
\end{corollary}

\section{An application to the Exactness of Gromov Monster groups.}

It is well known \cite{MR1911663,explg1} that any group that contains a coarsely embedded expander does not have Yu's property A and admits coefficients for which the Baum-Connes conjecture fails to be a surjection, but is an injection:

\begin{theorem}\label{Thm:IT1}
Let $G$ be a Gromov monster group. Then there exists a (commutative) $G$-$C^{*}$-algebra $A$ such that the Baum-Connes assembly map:
\begin{equation*}
\mu_{r,A,*}: KK^{G}_{*}(\underline{E}G, A) \rightarrow K_{*}(A\rtimes_{r}G)
\end{equation*}
is not surjective, but is injective.
\end{theorem}

We explore this result from the point of view of the geometry that can be associated to the expander graph that it inherits from the group.

Recall the coarse Baum-Connes conjecture for any uniformly discrete bounded geometry space $X$ can be phrased as a conjecture with coefficients in a certain groupoid $G(X)$ associated to $X$ \cite{MR1905840}. This groupoid admits a transformation groupoid decomposition \cite[Lemma 3.3b)]{MR1905840}, giving an easy description of $G(X)$ when it is possible to get a handle on the generators of the metric coarse structure on $X$. When $X$ is coarsely embedded into a group, this is certainly the case; the concept of a partial translation structure \cite{MR2363428} gives any space coarsely embedded into a group a nice collection of generators, as well as a locally compact, Hausdorff, second countable \etale groupoid that impliments the transformation decomposition.

On the other hand, the question of when a groupoid admits a transformation groupoid decompostion, up to Morita equivalence, has been well studied for the class of groupoids that are constructed from suitable inverse semigroups \cite{MR1900993,Milan-Steinberg}. This is related to the problem of globalisation of a partial action of $\Gamma$ on a space $X$. The result would be a space $Y$, with a true action of $\Gamma$ such that $X \hookrightarrow Y$ is a topological embedding and the enlargement of the $\Gamma$ action by restricting the maps to $X$ induces the original partial action.

The problem of globalisation of partial actions of groups was discussed in Section \ref{Sect:S3}. We recall following definition from Chapter 2:
\begin{definition}
Let $X$ be a topological space and let $\Gamma$ be a group acting partially on $X$. Then we denote by $\Omega$ the \textit{Morita evelope} of the action of $\Gamma$ on $X$, which is constructed as follows:

Consider the space $X\times \Gamma$, equipped with the product topology. Then define $\sim$ on $X\times \Gamma$ by $(x,g)\sim (y,h)$ if there exists $\gamma \in \G$ with $x(h^{-1}g)=y$. We define $\Omega$ as the quotient of $X\times \Gamma$ by $\sim$ with the quotient topology. 
\end{definition}

\subsection{Some remarks about the coarse groupoid.}

From earlier work in Chapter 3 any space that coarsely embeds into a group admits a grouplike partial translation structure. We equip the expander sequence coarsely embedded in our Gromov monster group with this translation structure. The results of Chapter 3 then tell us that there is a groupoid $\G(\mathcal{T})$, such that the translation algebra is isomorphic to the reduced groupoid  $C^{*}$-algebra $C^{*}_{r}(\G(\mathcal{T}))$. 

Using Claim \ref{MainClaim:C1} the translations of $\mathcal{T}$ are the only elements we need to be understand when working with $\G(\mathcal{T})$. From the definition of $\mathcal{T}$ we can think of a translation structure as providing us an excellent generating set for the metric coarse structure on the space $X$; the groupoid $\G(\mathcal{T})$ acts freely on $\beta X$, and we can generate now the coarse groupoid using this data:

\begin{lemma}\label{Lem:CG}
The coarse groupoid $G(X) \cong \beta X \rtimes \G(\mathcal{T}).$
\end{lemma}
\begin{proof}
We observe that the set of $[t_{g},\widehat{D}_{t_{g}^{*}t_{g}}]$ covers $G(X)$; hence the collection $\mathcal{T}$ forms an admissible psuedogroup in the terminology of \cite{MR1905840}. The groupoid it generates is $\G(\mathcal{T})$. Then the result follows from Lemma 3.3b) \cite{MR1905840}.
\end{proof}

Now, we prove the following interesting Lemma:

\begin{lemma}\label{Lem:Strongly}
Let $X \subset \Gamma$ be a metric space with a grouplike partial translation structure $\mathcal{T}$ induced from $\Gamma$. Then the inverse submonoid $S=\langle \mathcal{T} \rangle \subset I_{b}(X)$ is strongly $0$-F-inverse.
\end{lemma}
\begin{proof}
It was proved that the monoid was $0$-F-inverse earlier in Lemma \ref{Lem:PTS}. It follows that it is strongly $0$-F-inverse as the conditions of Proposition \ref{Prop:Strongly} are satisfied for $S$.
\end{proof}

\begin{corollary}\label{Prop:Cocycle}
Let $X$ be a metric space and $G$ be a group such that $X$ is coarsely embedded into $G$. Then the translation groupoid $\G(\mathcal{T})$ admits a (T,C,F)-cocycle onto $G$.
\end{corollary}
\begin{proof}
This follows immediately from Lemma \ref{Lem:Strongly} and Corollary \ref{Thm:Trick}.
\end{proof}

\subsection{Non-exactness of a Gromov Monster.}
We begin with a definition:
\begin{definition}
A finitely generated discrete group $\Gamma$ is a \textit{Gromov monster group} if there exists a large girth expander with vertex degree uniformly bounded above $X$ and a coarse embedding $f: X \hookrightarrow \Gamma$. 
\end{definition}

These groups were shown to exist by Gromov \cite{MR1978492}, with a detailed proof by Arzhantseva, Delzant \cite{exrangrps}. The construction is technical and we require no details beyond those presented in the definition.

The rest of this chapter is deadicated to proving the following theorem:

\begin{thm}
Let $\Gamma$ be a Gromov monster group. Then there are locally compact Hausdorff topological $\Gamma$-spaces $Y_{i}$, $i \in \lbrace 1,2,3 \rbrace$ and a short exact sequence:
\begin{equation*}
0 \rightarrow C_{0}(Y_{1}) \rightarrow C_{0}(Y_{2}) \rightarrow C_{0}(Y_{3}) \rightarrow 0
\end{equation*}
such that
\begin{equation*}
0 \rightarrow C_{0}(Y_{1})\rtimes_{r} G \rightarrow C_{0}(Y_{2})\rtimes_{r} G \rightarrow C_{0}(Y_{3})\rtimes_{r} G \rightarrow 0
\end{equation*}
fails to be exact in the middle.
\end{thm}

We proceed first by anaylzing the situation from \cite[Section 8]{explg1}. The primary idea is to globalise  $C^{*}X$ in $C^{*}G$. Fix a left invariant proper metric on $G$.

Let $X_{n}:=N_{n}(X)\subset G$. Then we can form the $C^{*}$-algebras $\ell^{\infty}(X_{n}) \subseteq \ell^{\infty}(G)$. Being commutative algebras in this case, we could consider the dual picture by taking spectra, getting $C_{0}(\widehat{X_{n}}) \subset C(\beta G)$. It is clear that $X_{n} \subset X_{n+1}$, so the algebras $\ell^{\infty}(X_{n}) \subset \ell^{\infty}(X_{n+1})$. The remark here is that the inclusion of $X_{n} \subset G$ is not $G$-equivariant, but the system is $G$-equivariant; the action of $G$ on $X_{n}$ on the right by translations will send points in $X_{n}$ into a most $X_{n+l(g)}$ for each $g \in G$. Hence, the limit of the $\ell^{\infty}(X_{n})$ over $n$ is a $G$-algebra, and so we can form the semidirect product algebra $(\lim_{n}\ell^{\infty}(X_{n}))\rtimes_{r} G$. Lemma 8.4 from \cite{explg1} provides us the following isomorphisms:

\begin{lemma}\label{lem:GMG}
Let $X_{n}$ as above. Then $(\lim_{n}\ell^{\infty}(X_{n}))\rtimes G \cong \lim_{n} C^{*}_{u}(X_{n})$ and $(\lim_{n}\ell^{\infty}(X_{n},\mathcal{K}))\rtimes G \cong \lim_{n} C^{*}(X_{n})$.\qed
\end{lemma}

Let the coefficients $\lim_{n}\ell^{\infty}(X_{n},\mathcal{K})$ be denoted by $A$. We appeal to the fact that each $X_{n}$ is coarsely equivalent to $X$. As these limits are functorial in coarse maps, we conclude:

\begin{proposition}
Let $G$ be a Gromov monster group and $X$ the coarsely embedded large girth expander. Then we have $A\rtimes_{r} G \cong C^{*}X$.\qed
\end{proposition}

We develop a semigroup theoretic approach; the procedure we will follow will be a geometric analogue of this argument using translation structures and Theorem \ref{Thm:IT2} of Milan and Steinberg, which relies on the information about the coarse groupoid given above as well as the fact the the inverse semigroups associated to the coarse groupoid are strongly 0-F-inverse.

\begin{theorem}\label{Thm:GM1}
Let $G$ be a Gromov monster group. Then there are locally compact Hausdorff topological $G$-spaces $Y_{i}$, $i \in \lbrace 1,2,3 \rbrace$ and a short exact sequence:
\begin{equation*}
0 \rightarrow C_{0}(Y_{1}) \rightarrow C_{0}(Y_{2}) \rightarrow C_{0}(Y_{3}) \rightarrow 0
\end{equation*}
such that
\begin{equation*}
0 \rightarrow C_{0}(Y_{1})\rtimes_{r} G \rightarrow C_{0}(Y_{2})\rtimes_{r} G \rightarrow C_{0}(Y_{3})\rtimes_{r} G \rightarrow 0
\end{equation*}
fails to be exact in the middle.
\end{theorem}
\begin{proof}
To construct the complete sequence we use Lemma \ref{Lem:Cut} to get $Y_{1}:= (X \times G)/\sim$ and $Y_{3}:= (\partial\beta X \times G)/\sim$. We then get the short exact sequence of $G$-algebras:
\begin{equation*}
0 \rightarrow C_{0}(Y_{1}) \rightarrow C_{0}(Y_{2}) \rightarrow C_{0}(Y_{3}) \rightarrow 0.
\end{equation*}

Now we consider the crossed product algebras $C_{0}(Y_{i})\rtimes G$. Then the sequence above gives us some terms on K-theory: 
\begin{equation*}
\xymatrix@=1em{...\ar[r] & K_{0}(C_{0}(U)\rtimes G) \ar[r]& K_{0}(C_{0}(Z)\rtimes G) \ar[r]& K_{0}(C_{0}(Y)\rtimes G)\ar[r] & ...\\
...\ar[r] & K_{0}(\mathcal{K}) \ar[r]\ar[u]^{\ucong}& K_{0}(C^{*}_{r}(G(X))) \ar[r]\ar[u]^{\ucong}& K_{0}(C^{*}_{r}(G(X)|_{\partial\beta X})) \ar[r]\ar[u]^{\ucong}& ...}
\end{equation*}
And the bottom line is not exact on K-theory by the work of Chapter 4. It follows therefore that the sequence:
\begin{equation*}
0 \rightarrow C_{0}(Y_{1})\rtimes_{r} G \rightarrow C_{0}(Y_{2})\rtimes_{r} G \rightarrow C_{0}(Y_{3})\rtimes_{r} G \rightarrow 0
\end{equation*}
is not exact in the middle term.
\end{proof}

This idea can be extended to connect this proof of failure to be exact to the geometric one that is outlined above from \cite{higsonpreprint,explg1}. 

We connect this geometric approach using groupoids to the analytic approach outlined in the previous section. 

\begin{proposition}\label{prop:GMG}
Let $X=X_{0}$ and $X_{n}$ as above. Then the globalisations of $G(X_{n})$ given by $B_{n}\rtimes G$ that come from the translation groupoid action of Lemma \ref{Lem:CG} are all Morita equivalent.
\end{proposition}
\begin{proof}
As $X_{0}$ is coarsely equivalent to $X_{n}$ for all $n$, it follows that $G(X)$ is Morita equivalent to $G(X_{n})$ for all $n$. Using Lemma \ref{Lem:CG} we can see that each of the groupoids $G(X_{n})$ is isomorphic to a transformation groupoid $\beta(X_{n}) \rtimes \G_{\X_{n})$ and therefore admits a (T,C,F)-cocycle onto the monster group $G$. Using Theorem \ref{Thm:IT2-a} (or Theorem \ref{Thm:IT2}) we can conclude that each $G(X_{n})$ is also Morita equivalent to $B_{n}\rtimes G$. Subsequently $B_{n}\rtimes G$ are Morita equivalent for each $n$, induced by the natural inclusions that extend $B_{n} \rightarrow B_{n+1}$.  
\end{proof}

Lemma \ref{lem:GMG} is naturally a corollary to Proposition \ref{prop:GMG}.

\subsection{Boundary Coefficients for a Gromov Monster.}

We extend the ideas in the previous section using the results from Chapter 4. In that chapter we proved that the boundary groupoid $G(X)|_{\partial\beta X}$ of a large girth sequence with uniformly bounded vertex degree decomposes as $\partial\beta X \rtimes \G_{\widehat{X}}$, where $\G_{\widehat{X}}$ has the Haagerup property. We extend these ideas by considering the impact this has on a Gromov monster group that contains such a large girth expander. To this end we prove:

\begin{theorem}\label{Thm:GM2}
There exists a locally compact Hausdorff space $Z$ such that the groupoid $Y_{3}\rtimes G$ is Morita equivalent to $Z \rtimes F_{k}$.
\end{theorem}

This result relies on many aspects of Chapter 4.

\begin{lemma}\label{Lem:MEFree}
The boundary groupoid $G(X)|_{\partial\beta X}$ admits a (T,C,F)-cocycle onto $F_{k}$. 
\end{lemma}
\begin{proof}
We remark that this follows directly from the fact that the coarse boundary groupoid $G(X)|_{\partial\beta X}$ has a decompostion as $\partial\beta X \rtimes \G_{\widehat{X}}$, and that $S_{inf}$ is strongly $0$-F-inverse.
\end{proof}

\begin{proof}(Of Theorem \ref{Thm:GM2}).
Recall from the proof of Theorem \ref{Thm:GM1} that the groupoid $G(X)|_{\partial\beta X}$ is Morita equivalent to $Y_{3}\rtimes G$. Using Lemma \ref{Lem:MEFree} we know also that $Z:=(\partial\beta X \times F_{k} )/\sim$ is a locally compact Hausdorff space, arising from a (T,C,F)-cocycle onto $F_{k}$. This enables us to again appeal to either \cite[Theorem 6.14]{Milan-Steinberg} (Theorem \ref{Thm:IT2}) or \cite[Theorem 1.8]{MR1900993} (Theorem \ref{Thm:Trick}) to conclude that $G(X)|_{\partial\beta X}$ is Morita equivalent to $Z\rtimes F_{k}$. The Theorem then follows from transitivity of Morita equivalence.
\end{proof}

Theorem \ref{Thm:GM2} has an important Corollary, as the Baum-Connes conjecture with coefficients is a Morita invariant:

\begin{corollary}
The Baum-Connes conjecture for $G$ with coefficients in any $(Y_{3}\rtimes G)$-$C^{*}$-algebra is an isomorphism.\qed
\end{corollary}