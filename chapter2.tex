%\setstretch{1.6}
\chapter{The Basics.}
As outlined in the introduction inverse semigroups and groupoids play a large role in the development of many aspects of combinatorics, graph theory and analysis. In this section we provide some basics concerning these areas and the connections between them, developing the notion of Paterson and Exel of a \textit{universal groupoid} associated to an inverse semigroup. We then give a brief introduction to operator algebras, focusing on the situation that arises naturally from both inverse semigroups and groupoids. Lastly, we consider topological K-theory of $C^{*}$-algebras and we outline all the tools we need for the later chapters.

\section{Semigroup and Groupoid Theory.}

A \textit{semigroup} is a set $S$, together with an associative binary operation. If additionally it has a unit element, then we say it is a \textit{monoid}.

\begin{definition}\label{Def:invsemi}
Let $S$ be a semigroup. We say $S$ is $inverse$ if there exists a unary operation $*:S \rightarrow S$ satisfying the following identities:
\begin{enumerate}
\item $(s^{*})^{*}=s$
\item $ss^{*}s=s$ and $s^{*}ss^{*}=s^{*}$ for all $s \in S$
\item $ef=fe$ for all idempotents $e,f \in S$ 
\end{enumerate}
\end{definition}

A very fundamental example of such an object is the \textit{symmetric inverse monoid} on any set $X$. Consider the collection of all partial bijections of $X$ to itself equipped with the  natural composition law associated to functions, which is to find the largest possible domain for the composition makes sense. This is shown below in Figure \ref{Fig:Comp}.


\begin{figure}[h]
\begin{center}

%Def of Circles needed
\def\firstcircle{(-0.25,-1.25) circle (1.0cm)}
\def\secondcircle{(-0.25,0) circle (1.0cm)}
\def\thirdcircle{(-4.75,0) circle (1.0cm)}
\def\forthcircle{(-4.75,-2.5) circle (1.0cm)}
\def\fifthcircle{(-4.75,-1.25) circle (1.0cm)}

\tikzset{filled/.style={fill=circle area, draw=circle edge, thick},
    outline/.style={draw=circle edge, thick}}
    
\setlength{\parskip}{5mm}
% Set A and B
\begin{tikzpicture}
    \begin{scope}[fill opacity=0.5]
        \clip \firstcircle
              \fifthcircle;
        \fill[filled] \secondcircle
                      \thirdcircle
                      \forthcircle;
        \end{scope}
               
    %\draw[outline]    
    \draw[outline] \firstcircle node [below] {$dom(f_{2})$};
    \draw[outline] \secondcircle node [above] {$im(f_{1})$};
    \draw[outline] \thirdcircle node [above] {$dom(f_{1})$};
    \draw[outline] \forthcircle node [below] {$im(f_{1})$};
        %\node[anchor=south] at (current bounding box.north) {$A \cap B$};
    \draw[>=stealth,->,thick] (-3.25,0) -- node [above] {$f_{1}$} (-1.5,0);
    \draw[>=stealth,->,thick] (-1.5,-1.5) -- node [above] {$f_{2}$} (-3.5,-2.5);
    \draw (-6.75,-0.85) node {$dom(f_{2}\circ f_{1})$};
    \draw (-6.75,-1.75) node {$im(f_{2}\circ f_{1})$};    
    \draw[outline] (-8.5,-4) rectangle (2.25,1.5);
    \draw (1.75,-3.25) node {$X$};
\end{tikzpicture}

\caption{The multiplication of partial bijections}
\label{Fig:Comp}
\end{center}
\end{figure}
Explicitly:
\begin{equation*}
f_{2}\circ f_{1}: f_{1}^{-1}(im(f_{1})\cap dom(f_{2})) \rightarrow f_{2}(im(f_{1})\cap dom(f_{2})).
\end{equation*}

A key observation that makes connects the combinatorial aspects of inverse semigroup theory to the geometry of partial bijections is the following representation theorem of Wagner and Preston \cite{MR1455373}:

\begin{theorem}\label{Thm:WP}
Let $S$ be an inverse semigroup. Then there exists a set $X$ such that $S \hookrightarrow I(X)$.\qed
\end{theorem}

When $X$ is a metric space we can consider a inverse submonoid of $I(X)$ in which every partial bijection that maps elements only a finite distance. We call these partial translations and we denote the submonoid of these by $I_{b}(X)$.

\begin{definition}
Let $S$ be an inverse monoid. We denote by $E(S)$ the semilattice of idempotents (just by $E$ if the context is clear). This is a meet semilattice, where the meet is given by the product of $S$ restricted to $E$. In this situation we can define the following partial order:
\begin{equation*}
e \leq f \Leftrightarrow ef=e
\end{equation*}
In the situation that $E$ consists of subsets of some set $X$, the meet is intersection and this order corresponds to subset inclusion.
\end{definition}

We remark that for a metric space $X$ every idempotent element in $I(X)$ moves elements no distance, and hence $E(I(X))=E(I_{b}(X))$. An inverse submonoid with this property is often called \textit{full}.

We want to consider quotient structures of an inverse monoid, and unlike in group theory where we have the concept of a \textit{normal} subgroup our subsemigroups will not in general contain enough structure. One possible choice is to consider \textit{ideals} in $S$. 

\begin{definition}
Let $I$ be a subset of $S$. $I$ is an ideal of $S$ if $SI \cup IS \subset I$.
\end{definition}

From an ideal we can define a quotient at the cost of a \textit{zero element}, that is an element $0 \in S$ such that $0s=s0=0$ for all $s \in S$.

\begin{definition}
Let $S$ be an inverse monoid and let $I$ be an ideal of $S$. Then we can define $\frac{S}{I}$ to be the set $(S \setminus I) \cup \lbrace 0 \rbrace$, equipped with the following product:
\begin{eqnarray*}
s \ast t = \left\lbrace \begin{array}[c] $st$ \mbox{if} s \mbox{and} t \not\in I \\ 0 \mbox{ if } s \mbox{ or } t \in I \end{array}
\end{eqnarray*}
This is an inverse monoid with 0 called the \textit{Rees quotient} of $S$ by $I$.
\end{definition}

General quotients are given by equivalence relations and in order to get an inverse monoid structure on the equivalence classes it is enough to impose a closure condition on the relation. A relation of this type is called a \textit{congruence} on $S$.

\begin{definition}
An equivalence relation $\sim$ on $S$ is called a \textit{congruence} if for every $u,v,s,t \in S$ such that $s \sim t$, we know that $su\sim tu$ and $vs \sim vt$. This allows us to equip the quotient $\frac{S}{\sim}$ with a product, making it into an inverse monoid.
\end{definition}

We will be considering a specific congruence on $S$ called the \textit{minimum group congruence}.  This congruence, denoted by $\sigma$, is given by:
\begin{equation*}
s \sigma t \Leftrightarrow (\exists e \in E) es = et
\end{equation*}
A congruence is \textit{idempotent pure} if $e\in E$ and $e \sim s$ then $s \in E$. This collects all idempotents into classes when quotienting out. 
\begin{definition}
An inverse monoid $S$ is called E-unitary if for all $e \in E$ and $s \in S$ if $e \leq s$ then $s \in E$. $S$ is F-inverse if the preimage of each $g \in \frac{S}{\sigma}$ has a maximum element in the order on $S$.
\end{definition}
The minimum group congruence, on the class of E-unitary inverse monoids, is an example of an idempotent pure congruence. Additionally it is the smallest group congruence on $S$ \cite{MR1694900}.

We denote these maximal elements by $Max(S)$, and we remark that this is equivalent to asking that for every element $s \in S$ there exists a \textit{unique} maximal element $t \in S$ such that $s \leq t$.

For an F-inverse monoid it is possible to study the minimum group congruence by considering all the maximal elements with a new product:
\begin{equation*}
(\forall s,t \in Max(S)) s \Xst t = u \mbox{ for } !u \in Max(S) \mbox{ with } st \leq u.
\end{equation*}

In general the inverse monoids we will construct will not have this property because they will have a zero element. However we can make similar definitions in this case:

\begin{definition}
We say $S$ is 0-E-unitary if $\forall e \in E\setminus 0, s \in S$ $e \leq s$ implies $s \in E$. We say it is 0-F-inverse if there exists a subset $T \subset S$ such that for every $s \in S$ there exists a unique $t \in T$ such that $s \leq t$ and if $s \leq u$ then $u \leq t$.
\end{definition}

As mentioned before, the minimum group congruence on such monoids will return the trivial group. However by working in a category with a more relaxed type of morphism we can still build useful maps to groups. We develop this in Section \ref{Sect:S3} of this Chapter.

\section{Groupoids.}

\begin{definition}\label{def:grpoid2}
A \textit{groupoid} is a set $\G$ equipped with the following information:
\begin{enumerate}
\item A subset $\G^{(0)}$ consisting of the objects of $\G$, denote the inclusion map by $i: \G^{(0)}\hookrightarrow \G$. 
\item Two maps, $r$ and $s: \G  \rightarrow \G^{(0)}$ such that $r\circ i = s \circ i = Id$ 
\item An involution map $^{-1}:\G \rightarrow \G$ such that $s(g)=r(g^{-1})$
\item A partial product $\G^{(2)} \rightarrow \G$ denoted $(g,h) \mapsto gh$, with $\G^{(2)}=\lbrace (g,h) \in \G \times \G | s(g)=r(h) \rbrace\subseteq \G\times \G$ being the set of composable pairs.
\end{enumerate}
Moreover we ask the following:
\begin{itemize}
\item The product is associative where it is defined in the sense that for any pairs: 
\begin{equation*}
(g,h),(h,k)\in \G^{(2)} \mbox{ we have }(gh)k \mbox{ and } g(hk) \mbox{ defined and equal}.
\end{equation*}
\item For all $g \in \G$ we have $r(g)g=gs(g)=g$.
\end{itemize}
\end{definition}

A groupoid is \textit{principal} if $(r,s): \G \rightarrow \G^{(0)} \times \G^{(0)}$ is injective and \textit{transitive} if $(r,s)$ is surjective. A groupoid $\G$ is a \textit{topological groupoid} if both $\G$ and $\G^{(0)}$ are topological spaces, and the maps $r,s, ^{-1}$ and the composition are all continuous. A Hausdorff, locally compact topological groupoid $\G$ is \textit{proper} if $(r,s)$ is a proper map and \textit{\'etale} or \textit{r-discrete} if the map $r$ is a local homeomorphism. When $\G$ is \'etale, $s$ and the product are also local homeomorphisms, and $\G^{(0)}$ is an open subset of $\G$.

\begin{definition}
Let $\G$ be a groupoid and let $A,B \subset \G^{(0)}$. Set:
\begin{enumerate}
\item $\G_{A}=s^{-1}(A)$
\item $\G^{A}=r^{-1}(A)$
\item $\G^{B}_{A}=\G^{B} \cap \G_{A}$
\end{enumerate}

\begin{definition}
A subset of $F\subseteq G^{(0)}$ is said to be \textit{saturated} if for every element of $\gamma \in G$ with $s(\gamma) \in F$ we have $r(\gamma) \in F$.
\end{definition}

Let $A$ be a saturated set. Denote by $\G|_{A}$ the subgroupoid $\G_{A}^{A}$, called the \textit{reduction} of $\G$ to $A$. In particular it is worth noting that the groupoids $\G|_{\lbrace x \rbrace}$ are in fact groups, and we say that for a given $x \in \G^{(0)}$ that the group $\G^{x}_{x}$ is the \textit{isotropy} group at $x$.
\end{definition}

\begin{definition}
Let $\G$ be a locally compact groupoid and let $Z$ be a locally compact space. $\G$ acts on $Z$ (or $Z$ is a $\G$-space) if there is a continuous, open map $r_{Z}: Z \rightarrow \G^{(0)}$ and a continous map $(\gamma, z) \mapsto \gamma .z$ from $\G \ast Z:= \lbrace (\gamma, z) \in \G \times Z | s_{\G}(\gamma)=r_{Z}(z)\rbrace$ to $Z$ such that $r_{Z}(z).z=z$ for all $z$ and $(\eta \gamma).z= \eta.(\gamma. z)$ for all $\gamma, \eta \in \G^{(2)}$ with $s_{\G}(\gamma)=r_{Z}(z)$.
\end{definition}

When it is clear we drop the subscripts on each map. Right actions are dealt with similarly, replacing each incidence of $r_{Z}$ with $s_{Z}$.

\begin{definition}
Let $\G$ act on $Z$. The action is said to be \textit{free} if $\gamma.z=z$ implies that $\gamma = r_{Z}(z)$.
\end{definition}
We end this section with some useful examples.

\begin{example}\label{Ex:TransGrp}
Let $X$ be a topological $\Gamma$-space. Then the \textit{transformation groupoid} associated to this action is given by the data $X \times G \rightrightarrows X$ with $s(x,g)=x$ and $r(x,g)=g.x$. We denote this by $X \rtimes G$. A basis $\lbrace U_{i} \rbrace$ for the topology of $X$ lifts to a basis for the topology of $X \rtimes G$, given by sets $[U_{i},g]:=\lbrace (u,g) | u \in U_{i} \rbrace$. 
\end{example}

\begin{example}
The construction in the example above can be generalized to actions of \'etale groupoids. We are concerned with the topology here: Given an \'etale groupoid $\G$ and a $\G$-space $X$ as well as a with a basis $\lbrace U_{i} \rbrace$ for $\G^{(0)}$. We can pull this basis back to a basis for $X \rtimes \G$ given by $[r_{z}^{-1}(U_{i}),\gamma]$, where $U_{i} \subseteq s(\gamma)$.
\end{example}

\subsection{Groupoids from inverse monoids}\label{sect:semitogrpoid}
In this section we outline the machine of \cite{MR1724106,MR2419901} for producing a groupoid $\G_{\E}$ from an inverse semigroup $S$. The way we proceed involves studying the actions of $S$ on its semilattice $E$. Working with semilattices, being generalisations of Boolean algebras, we still have access to a version of Stone duality; there exists many compactifications of $E$, built from its order structure, that extends the natural conjugation action of $S$. 

We outline the steps in the construction.
\begin{enumerate}
\item Build an action of $S$ on $E$.
\item Build a dual space $\E$ to $E$, which is locally compact and Hausdorff. Construct an action of $S$ on $\E$.
\item Build the groupoid $\G_{\E}$ from this data.
\end{enumerate}
After the construction, we make some remarks about more general groupoids of germs built from representations of $S$.

\begin{definition}

\begin{enumerate}
\item Let $D_{e}=\lbrace f \in E | f \leq e \rbrace$. For $ss^{*} \in E$, we can define a map $\rho_{s}(ss^{*})=s^{*}s$, extending to $D_{ss^{*}}$ by $\rho_{s}(e) = s^{*}es$. This defines a partial bijection on $E$ from $D_{ss^{*}}$ to $D_{s^{*}s}$. 

\item We consider a subspace of $\textbf{2}^{E}$ given by the functions $\phi$ such that $\phi(0)=0$ if $S$ has a zero and $\phi(ef)=\phi(e)\phi(f)$. We can topologise this as a subspace of $\textbf{2}^{E}$, where it is closed. This makes it compact Hausdorff, with a base of topology given by $\widehat{D}_{e}= \lbrace \phi \in \E | \phi(e)=1 \rbrace$. This admits a dual action induced from the action of $S$ on $E$. This is given by the pointwise equation for every $\phi \in \widehat{D}_{s^{*}s}$:
\begin{equation*}
\widehat{\rho}_{s}(\phi)(e)=\phi(s^{*}es)
\end{equation*}
The use of $\widehat{D}_{e}$ to denote these sets is not a coincidence, as we have the following map $D_{e} \rightarrow \widehat{D}_{e}$:
\begin{equation*}
e \mapsto \phi_{e}, \phi_{e}(f)=1 \mbox{ if } e \leq f \mbox{ and } 0 \mbox{ otherwise }.
\end{equation*}
\begin{remark}
These character maps $\phi: E \rightarrow \lbrace 0,1 \rbrace$ have an alternative interpretation, they can be considered as \textit{filters} on $E$. A filter on $E$ is given by a set $F \subset E$ with the following properties:
\begin{itemize}
\item for all $e,f \in F$ we have that $e\wedgef=ef \in F$
\item for $e\in F$ with $e \leq f$ we have that $f \in F$ and
\item $0 \not\in F$ if $E$ has a zero.
\end{itemize}
the relationship between characters and filters can be summarised as: To each character $\psi$ there is a filter:
\begin{equation*}
F_{\psi}= \lbrace e \in E | \psi(e)=1 \rbrace.
\end{equation*}
And every filter $F$ provides a character by considering $\chi_{F}$, its characteristic function. This implements a 1-1 correspondence.
\end{remark}

\item We take the set $S \times \E$, topologise it as a product and consider subset $\Omega:= \lbrace (s, \phi) | \phi \in D_{s^{*}s} \rbrace$ in the subspace topology. We then quotient this space by the relation:
\begin{equation*}
(s, \phi) \sim (t, \phi^{'}) \Leftrightarrow \phi=\phi^{'} \mbox{ and } (\exists e \in E) \mbox{ with } \phi \in \widehat{D}_{e} \mbox{ such that } es=et
\end{equation*}
We can give the quotient $\G_{\E}$ a groupoid structure with the product set, unit space and range and source maps:
\begin{eqnarray*}
\G_{\E}^{(2)}:=\lbrace ([s,\phi],[t,\phi^{'}]) | \phi=\widehat{\rho}_{t}(\phi^{'}) \rbrace \\
\G_{\E}^{(0)}:= \lbrace [e,\phi] | e \in E \rbrace \cong \E \\
s([t,\phi])=[t^{*}t,\phi], r([t,\phi])=[tt^{*},\phi], 
\end{eqnarray*}
and product and inverse:
\begin{eqnarray*}
[s,\phi].[t,\phi^{'}]= [st,\phi^{'}] \mbox{ if } ([s,\phi],[t,\phi^{'}]) \in \G_{\E}^{(2)}, [s,\phi]^{-1} = [s^{*},\widehat{\rho}_{s}(\phi)] 
\end{eqnarray*}
For all the details of the above, we refer to \cite[Section 4]{MR2419901}. This groupoid is the \textit{universal groupoid} associated to $S$. We collect some information about this groupoid from \cite{MR2419901,MR1724106} in Theorem \ref{Thm:Info}.
\item Lastly we consider certain subspaces of $\E$ that are closed and saturated and we outline their construction and some associated technicalities below. This subspace arises from the question: what are the ultrafilters on $E$?

The answer to this question and the technical obstructions that arise form a large part of the papers \cite{MR2419901,MR2672179} of Exel and Lawson. We denote the subspace of ultrafilters $\E_{\infty}$. The main technical point is that, unlike the Boolean algebra case, $\E_{\infty}$ need not be a closed subset of $\E$ when $E$ is a semilattice. This leads Exel to consider what he calls \textit{tight} filters, which we denote by $\E_{tight}$. In \cite{MR2419901} it is shown that tight filters are the closure of the ultrafilters inside $\E$. 

\end{enumerate}
\end{definition}

We will regularly make use of the following result that arises from the presence of maximal elements:

\begin{claim}\label{MainClaim:C1}
Let $S$ be 0-F-inverse. Then every element $[s,\phi] \in \G_{\E}$ has a representative $[t,\phi]$ where $t$ is a maximal element.
\end{claim}
\begin{proof}
Take $t=t_{s}$ the unique maximal element above $s$. Then we know 
\begin{equation*}
s = t_{s}s^{*}s \mbox{ and } s^{*}s \leq t_{s}^{*}t_{s}
\end{equation*} 
The second equation tells us that $t_{s}^{*}t_{s} \in F_{\phi}$ as filters are upwardly closed, thus $(t_{s},\phi)$ is a valid element. Now to see $[t_{s},\phi]=[s,\phi]$ we need to find an $e \in E$ such that $e \in F_{\phi}$ and $se=t_{s}e$. Take $e=s^{*}s$ and then use the first equation to see that $s(s^{*}s)=t_{s}(s^{*}s)$.
\end{proof}
Using Claim \ref{MainClaim:C1} will be able to forget the non-maximal elements in the monoid $S$ when working with $\G_{\E}$. This technique will be prevalent throughout this document as it allows many natural geometric considerations to enter into what would otherwise be purely combinatorial calculations.

Lastly for this section we make remarks about the more general notion associated to a representation of $S \rightarrow I(X)$ called a \textit{groupoid of germs}.

\begin{remark}
If we define a topological action of $S$ on a locally compact Hausdorff space $X$ to be a representation of $\pi:S \rightarrow I(X)$ such that each $s \in S$ is continuous and has a open domain, where these domains satisfy $\bigcup_{s \in S}D_{s^{*}s} = X$.

From this we can construct a groupoid the recipe of for the universal groupoid. We do this by considering the subset of $S \times X$ given by $K:=\lbrace (s,x) | x \in D_{s^{*}s}\rbrace$. We then quotient by the relation outlined in the construction of the universal groupoid and give it the same product and inverse. This turns the quotient into a groupoid; called the groupoid of germs, denoted by $X \rtimes S$.

Putting $\E$ into this construction provides $\G_{\E}$ and every other suitable representation gives us a restriction of $\G_{\E}$. This follows from the work of \cite{MR2419901}.
 
\end{remark}

\section{Prehomomorphisms of inverse monoids and general partial actions of discrete groups.}\label{Sect:S3} 
In this section we outline some basic properties of partial actions of discrete groups on topological spaces, paying particular attention to the types of inverse monoid $S$ these generate. We then use analytic information associated to the group together with properties of inverse monoid to understand analytic properties of the universal \'etale groupoid $\G_{\E}$ that is built from the inverse monoid $S$. We begin with a definition.

\begin{definition}
Let $\rho: S \rightarrow T$ be a map between inverse semigroups. This map is called a \textit{prehomomorphism} if for every $s,t \in S$, $\rho(st) \leq \rho(s)\rho(t)$ and a \textit{dual prehomomorphism} if for every $s,t \in S$ $\rho(s)\rho(t) \leq \rho(st)$.
\end{definition}

We recall that a congruence is said to be \textit{idempotent pure} if the preimage of any idempotent is an idempotent. We extend this definition to general maps in the natural way. In addition we call a map $S \rightarrow T$ \textit{0-restricted} if the preimage of $0 \in T$ is $0 \in S$.

\begin{definition}
Let $S$ be a 0-E-unitary inverse monoid. We say $S$ is \textit{strongly 0-E-unitary} if there exists an idempotent pure, 0-restricted prehomomorphism, $\Phi$ to a group $G$ with a zero element adjoined, that is: $\Phi:S \rightarrow G^{0}$. In this instance the prehomomorphism property translates into: if $s,t \in S$ with $st \not = 0$, we have $\Phi(st)=\Phi(s)\Phi(t)$, as the order structure on $G^{0}$ is simply given by $g \leq h \Leftrightarrow g = 0$ or $g=h$.

We say it is \textit{strongly 0-F-inverse} if it is 0-F-inverse and strongly 0-E-unitary. This is equivalent to the fact that the preimage of each group element under $\Phi$ that is not $0$ contains a maximum element.
\end{definition}

This class of inverse monoids is particularly important: the idempotent pure, 0-restricted prehomomorphism onto a group (with 0) can be thought of as a generalisation of the minimum group congruence in the larger category of inverse monoids with prehomomorphisms. We will utilise this technology later to regain some of the information from a group when we cannot quotient out in any meaningful way due to the presence of a zero element.

\begin{example}
In \cite{MR745358,MR2221438} the authors introduce an inverse monoid that is universal for dual prehomomorphisms from a general inverse semigroup. In the context of a group $G$ this is called the \textit{prefix expansion}; its elements are given by pairs: $(X,g)$ for $\lbrace 1,g\rbrace \subset X$, where $X$ is a finite subset of $G$. The set of such $(X,g)$ is then equipped with a product and inverse:
\begin{equation*}
(X,g)(Y,h) = (X\cup gY,gh)\mbox{ , } (X,g)^{-1}=(g^{-1}X,g^{-1})
\end{equation*}
This has maximal group homomorphic image $G$, and it has the universal property that it is the largest such inverse monoid. We denote this by $G^{Pr}$. The partial order on $G^{Pr}$ can be described by reverse inclusion, induced from reverse inclusion on finite subsets of $G$. It is F-inverse, with maximal elements: $\lbrace(\lbrace 1,g \rbrace, g):g \in G \rbrace$.
\end{example}

\begin{definition}
Let $G$ be a finitely generated discrete group and let $X$ be a (locally compact Hausdorff) topological space. A \textit{partial action} of $G$ on $X$ is a dual prehomomorphism $\theta$ from $G$ to the symmetric inverse monoid $I(X)$ that has the following properties:
\begin{enumerate}
\item The domain $D_{\theta_{g}^{*}\theta_{g}}$ is an open set for every $g$.
\item $\theta_{g}$ is a continuous map.
\item The union: $\bigcup_{g \in G}D_{\theta_{g}^{*}\theta_{g}}$ is $X$.
\end{enumerate}
\end{definition}

Given this data we can generate an inverse monoid $S$ using the set of $\theta_{g}$. This would then give a representation of $S$ into $I(X)$. If the space $X$ is a coarse space, then it makes sense to ask if each $\theta_{g}$ is a close to the identity. In this case, we would get a representation into the bounded symmetric inverse monoid $I_{b}(X)$. We call such a $\theta$ a \textit{bounded partial action} of $G$.

We are going to be interested in turning a partial group action into a full group action; the process of globalisation has been considered in a variety of settings \cite{MR0160848, MR1798993, MR2041539, MR2419858, MR1900993, Milan-Steinberg}, each using the same central theme.

\begin{definition}
A \textit{globalisation} of a partial action $\theta: G \rightarrow I(X)$ is a space $Y$ with an injection $X \hookrightarrow Y$ and action $\tilde{\theta}$ of $G$ such that the partial action obtained from restricting the action $\tilde{\theta}$ to $X$ is equal to $\theta$. 
\end{definition}

A globalisation is minimal if it injects into any other globalisation. In \cite{MR2041539} the authors proved that for any partial action of a group $G$ there is a unique globalisation (up to equivalence of partial actions). This is defined as follows:

\begin{definition}
Let $X$ be a topological space and let $G$ be a group acting partially on $X$. Then we denote by $\Omega$ the \textit{Morita evelope} of the action of $G$ on $X$, which is constructed as follows:

Consider the space $X\times G$, equipped with the product topology. Then define $\sim$ on $X\times G$ by $(x,g)\sim (y,h)$ if and only if $x(h^{-1}g)=y$. We define $\Omega$ as the quotient of $X\times G$ by $\sim$ with the quotient topology. 

$G$ acts on $\Omega$ using right multiplication by inverses on the group factor of the equivalence classes. Clearly the map that sends $x \in X$ to $[1,x] \in \Omega$ is a topological injection. The main result of \cite{MR2041539} is that this new topological space is minimal amongst globalisations of $X$.
\end{definition}

This notion will be developed further in Section \ref{Sect:GVC} and will also play an important role in certain examples in Chapter 5.

\section{\texorpdfstring{$C^{*}$}{C*}-algebras of groupoids and inverse semigroups.}
Now we change directions slightly by introducing the analytic counterparts to topological spaces; $C^{*}$-algebras play an important role in generalising the notions of topology into a noncommutative setting. The work we initially outline below is the duality theorem of Geifand, Neimark and Segal that connects topology with analysis. Then we develop some purely noncommutative ideas by outlining the construction of natural $C^{*}$-algebras associated to both groupoids and inverse semigroups.

\subsection{Topological Spaces and Commutative $C^{*}$-algebras}

We define an abstract $C^{*}$-algebra, then consider some examples.

\begin{definition}
A Banach *-algebra is an algebra $A$, equipped with an involution $^{*}$ and a norm $\Vert . \Vert$ such that the algebra is complete in this norm.
\end{definition}

\begin{definition}
An abstract $C^{*}$-algebra is a Banach *-algebra $A$ such that $\Vert a^{*}a \Vert = \Vert a \Vert^{2}$.
\end{definition}

The fundamental example of this is bounded operators on Hilbert Space.

\begin{example}
Let $H$ be a Hilbert Space; then we consider the algebra $\mathcal{B}(H)$ of bounded linear operators on $H$. This has a native involution sending each $T \in \mathcal{B}(H)$ to its adjoint: $T^{*} \in \mathcal{B}(H)$ and a native norm arising from the inner product and this satisfies the identity above.
\end{example}

It is possible to connect this example to abstract $C^{*}$-algebras via the Gelfand-Naimark-Segal theorem:

\begin{theorem}
Every abstract $C^{*}$-algebra is a isometrically *-isomorphic to a $C^{*}$-subalgebra of $\mathcal{B}(H)$ for some Hilbert space $H$.\qed
\end{theorem}

The second example links these objects to topological spaces:

\begin{example}
Let $X$ be a Hausdorff, locally compact topological space. Consider the algebra of continuous, complex valued functions that vanish at infinity $C_{0}(X)$ with pointwise operations:
\begin{equation*}
(f+g)(x)=f(x)+g(x), (f\circ g)(x)=f(x)g(x)
\end{equation*}
We can add an involution to this algebra in the following way:
\begin{equation*}
f^{*}(x)=\overline{f(x)}
\end{equation*}
This turns this algebra into a *-algebra. We can also define a norm in the following way:
\begin{equation*}
\Vert f \Vert = \sup_{x \in X} \vert f(x) \vert
\end{equation*}
This is complete (Banach) algebra in this norm. Observe it satisfies the following identity:
\begin{equation*}
\Vert f^{*}f \Vert = \Vert f \Vert^{2}
\end{equation*}
So it is a $C^{*}$-algebra. Observe also has a commutative product.
\end{example}

This example will allow us classify \textit{all} commutative $C^{*}$-algebras using the following Theorem of Gelfand:

\begin{theorem}
The category of all commutative $C^{*}$-algebras and *-homomorphisms is equivalent to the opposite of the category of locally compact, Hausdorff topological spaces with proper maps.
\end{theorem}

So the study of commutative algebras is parallel to the study of locally compact Hausdorff topological spaces. The benefit of dealing with the $C^{*}$-algebras is that we can consider \textit{noncommutative} algebras. This concept forms the central backbone of the noncommutative geometry program of Connes \cite{MR1826266}.

\subsection{Hilbert $C^{*}$-modules}
To consider groupoid $C^{*}$-algebras we want to consider representations that fiber over the unit space; in particular we need to consider \textit{fields of Hilbert Spaces} - Hilbert Modules \cite{MR1325694}.

\begin{definition}
Let $A$ be a $C^{*}$-algebra. A \textit{Hilbert A-module} \mathscr{E} is a right $A$-module equipped with an $A$-valued form $\langle , \rangle: \mathscr{E} \times \mathscr{E} \rightarrow A$ which satisfies the following axioms:
\begin{enumerate}
\item $\langle \eta ,\zeta_{1} + \zeta_{2} \rangle = \langle \eta , \zeta_{1}\rangle + \langle \eta ,\zeta_{2} \rangle$;
\item $\langle \eta , \zeta a \rangle = \langle \eta ,\zeta \rangle a$;
\item $\langle \eta , \zeta \rangle^{*} = \langle \zeta ,\eta \rangle$;
\item $\langle \eta, \eta \rangle \geq 0$;
\item $\langle \eta ,\eta  \rangle = 0 $ if and only if $ \eta = 0$ and
\item $\mathscr{E}$ is complete with respect to $\Vert \eta \Vert = \Vert \langle \eta , \eta \rangle \Vert_{A}^{1/2}$
\end{enumerate}
\end{definition}

\begin{remark}
The axioms above imply a generalization of the Cauchy-Schwartz inequality and so Hilbert $A$-modules also satisfy the triangle inequality.
\end{remark}

\begin{remark}
If you put $A= \mathbb{C}$ then the above definition reduces to that of a Hilbert space. A $C^{*}$-algebra $A$ can be thought of as a Hilbert $A$-module over itself using the inner product: $\langle a, a^{'} \rangle = a^{*}a^{'}$. Also we remark that not all the basic facts that apply to Hilbert spaces apply to Hilbert modules - in general the Riesz Representation Theorem fails for Hilbert Modules \cite{MR1077390}.
\end{remark}

\subsection{Constructions of Groupoid \texorpdfstring{$C^{*}$}{C*}-algebras.}\label{Sect:GVC} The standard technique for a locally compact group $G$ involves considering norm completions associated to representations of the ring of compactly supported functions on the group. We can associate a very natural representation on the space $L^{2}(G,\mu)$, where $\mu$ is the Haar measure on $G$. To extend these ideas to a locally compact groupoid we will need an analogue of this measure in a suitably fibred manner.

\begin{definition}
A \textit{left Haar system} for a locally compact groupoid $\G$ is a family $\lbrace \lambda^{u} \rbrace$, where each $\lambda^{u}$ is a positive regular Borel measure on the locally compact Hausdorff $\G^{u}$, such that the following hold:
\begin{enumerate}
\item the support of each $\lambda^{u}$ is the whole of $\G^{u}$;
\item for any $g \in C_{c}(\G)$, the function $g^{0}$, where:
\begin{equation*}
g^{0}(u)=\int_{\G^{u}}g d\lambda^{u}
\end{equation*}
belongs to $C_{c}(\G^{(0)})$.
\item for any $x \in \G$ and $f \in C_{c}(\G)$,
\begin{equation*}
\int_{\G^{d(x)}}f(xz)d\lambda^{d(x)}(z) = \int_{\G^{r(x)}}f(y)d\lambda^{r(x)}(y)
\end{equation*}
\end{enumerate}
\end{definition}

We now observe that when the groupoid is \'etale it is possible to take as a Haar system the counting measure (this fact is a consequence of Proposition 2.2.5  in \cite{MR1724106}). This eases the passage through calculations significantly and so we make the assumption that $G$ is locally compact and \'etale from now on. We now give explicit formulae for the convolution product and adjoint on $C_{c}(G)$. This is taken from \cite{MR2419901}. For every $f,g \in C_{c}(\G)$:
\begin{eqnarray*}
(f \ast g)(\gamma) & = & \sum_{\substack{(\sigma,\tau) \in \G^{(2)}\\ \sigma\tau=\gamma}}f(\sigma)g(\tau)\\
f^{*}(\gamma) & = & \overline{f(\gamma^{-1})} 
\end{eqnarray*}

We outline two methods generalising the standard views from the theory of locally compact groups. First is the technique of inducing a representation of the group from a representation of the functions defined on the identity and the second involves Hilbert modules, which is outlined at the end of this section.

What follows is given in full generality and taken from \cite[Appendix D]{MR1724106}
\begin{definition}
A dense *-subalgebra of a $C^{*}$-algebra is called a \textit{pre-$C^{*}$-algebra}.
\end{definition}

Let $A$ and $B$ be pre-$C^{*}$-algebras such that $B$ acts an algebra of right multipliers on $A$. The action of $B$ will be denoted by: $(a,b) \rightarrow a.b$. This is assumed to be continuous. 

\begin{definition}
Let $P$ be a linear, self-adjoint positive map from $A \rightarrow B$. We say that $P$ is a \textit{generalised conditional expection} if:
\begin{enumerate}
\item $P(a.b)=P(a)b$
\item for all $c \in A$ the linear map $a \mapsto P(c^{*}ac)$ from $A$ to $B$ is bounded
\item for every $a\in A$ and every $\epsilon > 0 $ there exists a $c$ in the span $A^{2}$ of elements $a_{1}a_{2}, a_{i} \in A$ such that:
\begin{equation*}
\Vert P((a-c)^{*}(a-c))\Vert < \epsilon
\end{equation*}
\item $P(A)$ generates a dense subalgebra of $B$.
\end{enumerate}
\end{definition}

This map projects functions from one algebra onto another in a way that compliments the multiplier action. Take $A$ and $B$ to be pre-$C^{*}$-algebras with a conditional expectation $P: A \rightarrow B$.

Let $H$ be a Hilbert space and define $\pi:B \rightarrow H$ by treating $H$ as a left Hilbert $B$-module by defining:
\begin{equation*}
b\xi = \pi(b)\xi
\end{equation}
$A$ is a right Hilbert $B$-module as $B$ acts on $A$ by right multipliers. Now we can form the tensor product $A \otimes_{B} H$, and this becomes a (pre)-Hilbert space using the inner product:
\begin{equation}
\langle a\otimes \xi, a^{'}\otimes \eta \rangle = \langle P((a^{'})^{*}a)\xi, \eta \rangle_{H}.
\end{equation}
Quotenting by the zero vectors in this gives a Hilbert space, and we denote this by $K$. We can now represent $A$ on $K$ as follows:
\begin{equation*}
Ind(\pi):A \rightarrow \mathcal{B}(K), Ind(\pi)(a)(a^{'} \otimes \xi)=aa^{'}\otimes \xi
\end{equation}
Then the map $Ind(\pi)$ is called the \textit{induced representation of A} associated with $\pi$.

So consider the application of this process to the following pre-$C^{*}$-algebras:

\begin{example}
Let $\mathcal{G}$ be a r-discrete topological groupoid. Let $A:=C_{c}(\mathcal{G})$ and $B:=C_{0}(\mathcal{G}^{(0)})$. $P$ in this case is the restriction map. We can represent $B$ on $L(\mathcal{G}^{(0)}, \mu)$ where $\mu$ is a measure on the unit space. So for a given unit $v \in \mathcal{G}^{(0)}$ we can construct a Hilbert space and a representation as follows:

Let $\pi:B \rightarrow B(H)$ given by diagonal multiplication as above. Then we have a map $P_{v}:f \in A \rightarrow P(f)=f(v)$ where $P:A \rightarrow B$ is given as a sum of these maps in the following way:

\begin{equation*}
P(f)=(f(v_{1}),f(v_{2}),...)=\oplus_{i \in \mathbb{N}} P_{v_{i}}(f)
\end{equation}
We can define a Hilbert space: $K_{v}=A \otimes_{\lbrace v \rbrace} H$ with the inner product: $\langle a\otimes \xi, a^{'}\otimes \eta \rangle = \langle \pi(P_{v}(a^{'}^{*}a))\xi, \eta \rangle_{H}$ and a Hilbert $C_{0}(\G^{(0)})$-module $K=\bigoplus_{v\in \mathcal{G}^{(0)}} K_{v}$. Now the induced representation is the sequence of multiplier operators:
\begin{eqnarray*}
Ind(v)(f)(a \otimes \xi)=P_{v}(f)a \otimes \xi\\
Ind(\pi)=\bigoplus_{v \in \mathcal{G}^{(0)}}Ind(v)\\
\end{eqnarray*}
\end{example}

We can define a norm on $C_{c}(\mathcal{G})$ coming from this representation:
\begin{definition}
Let $f \in C_{c}(\mathcal{G})$ Then $\Vert f \Vert_{r}$=$sup \lbrace \Vert Ind(v) \Vert_{K_{v}} : v\in \mathcal{G}^{(0)} \rbrace$. We call this norm \textit{the reduced groupoid norm}
\end{definition}
Completing $C_{c}(\mathcal{G})$ in this norm on $\mathcal{B}(K)$ gives the reduced groupoid $C^{*}$-algebra $C^{*}_{r}(\mathcal{G})$.

We observe that this completion arises by considering a field of Hilbert spaces over $C_{0}(\G^{(0)})$. We can also come up with an identification between this module structure and the natural field of Hilbert spaces structure outlined above.  We begin by putting a natural pre-Hilbert $C_{0}(\G^{(0)})$-module structure on this function algebra by defining the inner product:
\begin{equation*}
\langle \zeta, \eta \rangle = (\zeta^{*}\ast\eta)|_{\G^{(0)}}.
\end{equation*}
We observe that for any function $f \in C_{0}(\G^{(0)})$ we can define a right action on $C_{c}(\G)$ by: $(\eta.f)(\gamma) = \eta(\gamma)f(s(\gamma)) $. We can then complete this as a Hilbert module, and we denote this by $L^{2}(\G)$. The algebra $C_{c}(\G)$ represents naturally on this algebra using the representation: $\lambda(f)(\eta)=f \ast \eta$.

It is well known that any Hilbert $C_{0}(\G^{(0)})$-module $\mathcal{M}$ is the space of sections of a continuous field of Hilbert spaces $\lbrace \mathcal{M}_{x} \rbrace_{x \in \G^{(0)}}$, with any bounded adjointable operator $T$ on $\mathcal{M}$ decomposing as a strongly $*$-continuous field $(T_{x})_{x \in \G^{(0)}}$ with, $\Vert T \Vert = \sup_{x \in \G^{(0)}} \Vert T_{x} \Vert$. We use this to get easier access to the norm by explicitly constructing each $\mathcal{M}_{x}$. To do this, we construct an inner product for each $x \in \G^{(0)}$:
\begin{equation*}
\langle \zeta, \eta \rangle_{x} = (\zeta^{*}\ast\eta)(x). 
\end{equation*}
This defines an inner product on $C_{c}(\G_{x})$, which we can use to complete. We denote this completion by $L^{2}(\G_{x})$. This gives us the natural field of Hilbert spaces we were looking for, namely: $\lbrace L^{2}(\G_{x}) \rbrace_{x \in \G^{(0)}}$. It also gives us a natural representation of $C_{c}(\G)$ given by $\lambda_{x}(f)\eta = (f \ast \eta)(x)$. Hence we can conclude that $\Vert f \Vert = \sup_{x \in \G^{(0)}} \Vert \lambda(f) \Vert = \Vert \lambda(f) \Vert$. From this we can complete $C_{c}(\G)$ in either the norm on $L^{2}(\G)$ or the family of norms $\lbrace L^{2}(\G_{x}) \rbrace_{x \in \G^{(0)}}$, getting the same completion, denoted by $C^{*}_{r}(\G)$. 

These ideas clearly agree with the more formal construction given at the beginning of this section as the $K_{v}$ are isometrically isomorphic to $L^{2}(\G|_{v})$. 

The last point of this section is to collect some information concerning the universal groupoid $\G_{\E}$ of a inverse monoid $S$.

\begin{theorem}\label{Thm:Info}
Let $S$ be a countable 0-E-unitary inverse monoid, $E$ its semilattice of idempotents and $\G_{\E}$ its universal groupoid. Then the following hold for $\G_{\E}$:
\begin{itemize}
\item $\E$ is a compact, Hausdorff and second countable space.
\item $\G_{\E}$ is a Hausdorff groupoid.
\item Every representation of $S$ on Hilbert space gives rise to a covariant representation of $\G_{\E}$ and vice versa.
\item We have $C^{*}_{r}(S) \cong C^{*}_{r}(\G_{\E})$.
\end{itemize}
\end{theorem}
\begin{proof}
The first point is a consequence of the fact that $E$ is countable, in this situation we know precisely that $\textbf{2}^{E}$ is metrizable, hence as a closed subset we know that $\E$ is second countable. It is compact and Hausdorff as it is a closed subset of a compact, Hausdorff space.

The second point follows from Corollary 10.9 \cite{MR2419901}, the third point is Corollary 10.16 \cite{MR2419901} and the fourth point follows from \cite{MR1724106}, but a more elementary proof is given in \cite{MR1900993}.
\end{proof}

\section{Semigroup Valued Cocycles and a Theorem of Milan and Steinberg.}

In this section we consider the question of when a groupoid admits a transformation groupoid decompostion up to Morita equivalence. This question connects to the idea of globalising partial actions discussed in Section \ref{Sect:S3} and this has been well studied for the class of groupoids constructed from suitable inverse semigroups \cite{MR1900993,Milan-Steinberg}.

\begin{definition}
Let $\G$ be a locally compact groupoid. Then we call a continuous homomorphism from $\G$ to a inverse semigroup $S$ an inverse semigroup valued cocycle (or just cocycle).
\end{definition}

\begin{definition}
Let $\rho: \G \rightarrow S$ be a cocycle. We say it is:
\begin{enumerate}
\item \textit{transverse} if the map $S \times \G \rightarrow S \times X$, $(s, \gamma) \mapsto (s\rho(\gamma),s(\gamma))$ is open.
\item \textit{closed} if the map $\gamma \mapsto ((r(\gamma),\rho(\gamma),s(\gamma))$ is closed.
\item \textit{faithful} if the map $\gamma \mapsto ((r(\gamma),\rho(\gamma),s(\gamma))$ is injective.
\end{enumerate}
We call a cocycle $\rho$ with all these properties a \textit{(T,C,F)-cocycle}.
\end{definition}

Below is the main result of \cite{Milan-Steinberg}, a generalisation of the main results of \cite{MR1900993}:

\begin{theorem}\label{Thm:IT2}
Let $\rho: \G \rightarrow S$ be a continuous, faithful closed transverse cocycle where $\G$ is a locally compact groupoid and $S$ is a countable inverse semigroup. Then there is a locally compact Hausdorff space $X$ equipped with an action of $S$ so that $\G$ is Morita equivalent to the groupoid of germs $X \rtimes S$. Consequently $C^{*}_{max}\G$ is strongly Morita equivalent to $C_{0}(X)\rtimes S$. If $S$ is a group, then the analogous result holds for reduced $C^{*}$-algebras.
\end{theorem}

From an F-inverse monoid $S$ it is possible to construct a (T,C,F)-cocycle onto the maximal group homomorphic image of $G$ \cite{MR1900993}. To prove Theorem \ref{Thm:IT2} in the case that the monoid is $F$-inverse then makes use of the Morita envelope of the partial action that the maximal group homomorphic image $G$ has on the unit space of the universal groupoid $\G_{\E}$. IThe closed condition on the cocycle makes this space Hausdorff.

What follows from here can be found as a Corollary to Theorem \ref{Thm:IT2} from \cite{Milan-Steinberg}. We provide a direct proof of a special case using the original methods of \cite{MR1900993}. This is possible by considering the construction of the groupoid $\G_{\E}$ for a strongly 0-E-unitary inverse monoid $S$. It is clear that the only danger is mapping elements to $0$ in $\Gamma^{0}$; this is overcome by the observation that the element $[0,f]$ would be defined if and only if $f \in D_{0}$. However, $f \in D_{0}$ implies that $f(0)=1$ and hence $f \not\in \E$, so the $0$ element of $S$ contributes nothing to the groupoid $\G_{\E}$, either in objects or arrows.

We are interested in proving that if $S$ is a strongly $0$-F-inverse monoid  then we can apply some analogue of Theorem \ref{Thm:IT2}. This is Corollary 6.17 from a \cite{Milan-Steinberg}, however we give a direct proof here for completeness just in the special case in which we are interested, by adapting the original techniques of \cite{MR1900993}. 

\begin{definition}
We say that S satisfies the finite cover property with respect to $\phi$,, if for every $p,q \in S$, $p,q\not = 0$ there exists a finite set $U \subset S_{g}$ such that:
\begin{equation*}
pS_{g}q=\lbrace s \in S | \exists u \in U | s \leq u \rbrace.
\end{equation*}
Where $S_{g}$ is the preimage $\phi^{-1}(g)$.
\end{definition}

\begin{theorem}\label{Thm:IT2-a}
Let $S$ be an inverse monoid. If $S$ is strongly 0-E-unitary with non-trivial universal group $U(S)=\Gamma$ such that the prehomomorphism has the finite cover property. Then the groupoid $\G_{\E}$ admits a transverse and faithful cocycle to a group $\Gamma$.
\end{theorem}
\begin{proof}
Let $\Phi$ be the 0-restricted, idempotent pure prehomomorphism onto $\Gamma^{0}$. We build an induced map on the groupoid $\G_{\E}$ by considering a new map $\Psi:$
\begin{equation*}
\Psi([s,x])=\Phi(s)
\end{equation*}
This map is well-defined as any non-zero idempotent in $S$ is mapped to the identity in $\Gamma$, and so for any pair $(s,f) \sim (t,f)$ there is an $e \in E \cap D_{f}$, in particular not $0$, such that $es=et$ and hence $\Phi(s)=\Phi(es)=\Phi(et)=\Phi(t)$. This is clearly a groupoid homomorphism to $\Gamma$. To check it is continuous observe that as $\Gamma$ is a discrete group so all subsets are open. The preimage of a singleton is given by the union:
\begin{equation*}
\Psi^{-1}(\lbrace g \rbrace)=\bigcup_{\Phi(u)=g}[u,D_{u^{*}u}] 
\end{equation*}
which is certainly open in $\G_{\E}$. The map is proper, because the preimage of any finite set in $\Gamma$ is equal to a finite union of $[u,D_{u^{*}u}]$ by the finite cover property and these are compact by construction.

It remains to check it is a (T,C,F)-cocycle, and from the remarks prior to the Theorem the proof of this follows exactly from the proof \cite[Proposition 3.6]{MR1900993} modified suitably.

To prove this is transverse, it is enough to prove that $\lbrace (\Psi(\gamma),s(\gamma)):\gamma \in \G_{\E}\rbrace$ is open in $\Gamma \times \E$, and this in turn reduces to studying this problem for all $g \in G$, that is if $\lbrace s(\gamma) :\Psi(\gamma)=g \rbrace$. is open in $\E$. This set is equal to $\bigcup_{\Psi(\gamma)=g}D_{s(\gamma)}$, which is certainly open in $\E$ as each piece is.

To see that this is faithful, let $[u,f], [v,f^{'}] \in \G_{\E}$ such that $(f,\Phi(u),\theta_{u}(f))=(f^{'},\Phi(v),\theta_{v}(f^{'}))$. Then it is clear that $f=s([u,f])=s([v,f^{'}])=f^{'}$, so it is enough to prove now that $\Phi(v)=\Phi(u)$ implies $[u,f]=[v,f]$. Observe that $\Phi(u)\Phi(v)^{-1}=1$ in $\Gamma$ and $\Phi(v)^{-1}=\Phi(v^{*})$, so $\Phi(uv^{*})=1$. This map is idempotent pure, so $uv^{*} \in E(S)$. So $[u,f][v,f]^{-1}=[uv^{*},\theta_{v}(f)]$ is a unit in $\G_{\E}$. From here it is clear that $[u,f]$ is an inverse to $[v^{*},\theta_{v}(f)]$ and so $[u,f]=[v,f]$.\end{proof}

We still need to check the fact that the cocycles are closed. Again this follows from the work of \cite{Milan-Steinberg} or \cite{MR1900993}, but we give the proof in this setting:

\begin{lemma}
If $S$ is an inverse monoid and has the finite cover property with respect to $\phi$, then the induced cocycle $\rho$ is closed.
\end{lemma}
\begin{proof}
As $\Gamma$ is discrete, it is enough to prove that the graph $Gr(g)$ over $g$ in $\E \times \E$ is closed. We remark also that this product space is covered by the set of $D_{e} \times D_{f}$, where $e,f$ run though the idempotents $E(S)$, and is compact; thus only finitely many pairs $D_{e}\times D_{f}$ are necessary. The intersection $Gr(g) \cap D_{e}\times D_{f}$ is covered by $\bigcup_{u \in eS_{g}f} [u,\widehat{D}_{u^{*}u}]$ and so are compact if and only if:
\begin{equation*}
Gr(g) \cap D_{e}\times D_{f} = \bigcup_{u \in U}[u,\widehat{D}_{u^{*}u}]
\end{equation*}
for some finite $U \subset S_{g}$. However, this is precisely implied by the finite cover property.
\end{proof}

\begin{corollary}\label{Thm:Trick}
If $S$ is a 0-E-unitary monoid with the finite cover property and non-trivial universal group then the groupoid $\G_{\E}$ is Morita equivalent a transformation groupoid $Y \rtimes \Gamma$.\qed
\end{corollary}
\begin{proof}
This follows from Theorem 1.8 from \cite{MR1900993}.
\end{proof}

\begin{remark}
If, in addition the inverse monoid $S$ is 0-F-inverse, then it satisfies the finite cover property with $\vert U \vert=1$ as each non-empty $S_{g}$ will contain a unique maximal element.
\end{remark}


\section{K-theory of \texorpdfstring{$C^{*}$}{C*}-algebras}
In this section we give the basic definitions of operator K-theory, following the exposition of \cite{MR1222415}. There are many alternative texts that could be followed instead, such as: \cite{MR2340673,MR1656031} but the explicit calculations make \cite{MR1222415} exceptionally clear. In this section we will consider only unital $C^{*}$-algebras. It is possible to perform the calculations required for the proofs of these facts via the unitisation. This is similar to working with locally compact spaces in topological K-theory.

\begin{definition}
A projection in a $C^{*}$-algebra is a self-adjoint idempotent operator. That is $p=p^{*}=p^{2}$ (i.e $p^{*}p=p$). A pair of projections $p$ and $q$ are orthogonal if $pq=0$. 
\end{definition}

\begin{lemma}
The sum $p+q$ of two projections $p$ and $q$ is a projection if and only if $p$ and $q$ are orthogonal.\qed
\end{lemma}

\begin{definition}
Let $p,q \in A$ be projections. 
\begin{enumerate}
\item $p$ is said to be \textit{equivalent} to $q$, $p \sim q$, if there exists a partial isometry $v \in A$ such that $p=v^{*}v$ and $q=vv^{*}$.
\item $p$ is said to be \textit{unitarily equivalent} to $q$, $p \sim_{u} q$, if there exists a unitary $u \in A$ such that $p=u^{*}qu$.
\item $p$ is said to be \textit{homotopic} to $q$, $p \sim_{h} q$, is there exists a norm continous path of projections $p_{t}$ such that $p_{0}=p$ and $p_{1}=q$.
\end{enumerate}
\end{definition}

\begin{proposition}
If $p$ and $q$ are projections in $A$, then: $p \sim_{h} q \Rightarrow p \sim_{u} q \Rightarrow p \sim q$.\qed
\end{proposition}

These relations are not in general reversible. If one is willing to work with matrix algebras $M_{n}(A)$ over $A$, then they are however.

\begin{lemma}
Let $p,q$ be projections in $A$. Then $p \sim q \Rightarrow diag(p,0) \sim_{u} diag(q,0)$ and $p \sim_{u} q \Rightarrow diag(p,0) \sim_{h} diag(q,0)$.
\end{lemma}

In defining K-theory, similar to both topological K-theory and algebraic K-theory, we will actually be considering a \textit{stabilisation} of $A$, i.e working with $M_{\infty}(A)$. Lastly, need a result that allows a "sum" of projections to be well defined up to equivalence. The trick, again, is to rely on passing to a matrix algebra: given $p,q \in A$ projections, we observe that $diag(0,p)$ is unitarily equivalent to $diag(p,0)$, and orthogonal to $diag(q,0)$. We define: $[p] + [q] = [diag(p,q)]$. 

\begin{definition}
Let $A$ be a $C^{*}$-algebra. We denote by $Proj(A)$ to be the set of equivalence classes of projections in $M_{\infty}(A)$. With the sum, $+$ defined above, this is a commutative monoid. We denote by $K_{0}(A)$ the Grothendieck group of $Proj(A)$.
\end{definition}

A morphism $A \rightarrow B$ naturally extends entrywise to $M_{\infty}(A) \rightarrow M_{\infty}(B)$; These morphisms induce maps $Proj(A) \rightarrow Proj(B)$ which then pass to the Grothendieck group $K_{0}(A) \rightarrow K_{0}(B)$

\begin{lemma}\label{lem:ses}
To any short exact sequence $0 \rightarrow C \rightarrow B \rightarrow \frac{B}{C} \rightarrow 0$ we get an induced half-exact sequence, using the entrywise induced maps discussed above:
\begin{equation*}
K_{0}(C) \rightarrow K_{0}(B) \rightarrow K_{0}(\frac{B}{C})
\end{equation*}
\end{lemma}

The definition of $K_{1}$ is more technical and is constructed from unitaries instead of projections. We present it tersely here.

\begin{definition}
$K_{1}(A):= \frac{GL(A)}{GL(A)_{0}}$.
\end{definition}

In particular, for a finite invertible matrix $u \in GL_{n}(A)$, the class $[u] \in K_{1}(A)$ is the connected component containing $diag(u,1_{\infty}) \in GL(A)$. Similar to Lemma \ref{lem:ses}:

\begin{lemma}\label{lem:ses2}
To any short exact sequence $0 \rightarrow C \rightarrow B \rightarrow \frac{B}{C} \rightarrow 0$ we get an induced half-exact sequence:
\begin{equation*}
K_{1}(C) \rightarrow K_{1}(B) \rightarrow K_{1}(\frac{B}{C})
\end{equation*}
\end{lemma}

\begin{definition}(Boundary map)\cite[Def. 8.1.1]{MR1222415}
Let $J \triangleleft A$ and let $x \in K_{1}(\frac{A}{J})$. Then we can find a $u \in \mathcal{U}_{n}^{+}(\frac{A}{J})$ such that $x=[u]$ and $v \in \mathcal{U}_{k}^{+}(\frac{A}{J})$ such that $diag(u,v)$ is homotopic to $1_{n+k}$ in $\mathcal{U}_{n+k}^{+}(\frac{A}{J})$. Let $w zin \mathcal{U}_{n+k}^{+}(A)$ be a lift of $diag(u,v)$. Then the \textit{boundary map} $\delta: K_{1}(\frac{A}{J}) \rightarrow K_{0}(J)$ is defined by:
\begin{equation*}
\delta(x):= [wp_{n}w^{*}]-[p_{n}].
\end{equation*}
This is a well-defined group homomorphism.
\end{definition}

\begin{lemma}
This gives us a long exact sequence:
\begin{equation*}
K_{1}(J) \rightarrow K_{1}(A) \rightarrow K_{1}(\frac{A}{J})\rightarrow^{\delta} K_{0}(J) \rightarrow K_{0}(A) \rightarrow K_{0}(\frac{A}{J})
\end{equation*}
\end{lemma}

Alternatively, we could have proceeded as we would have in topological K-theory, that is via \textit{cones} and \textit{suspensions}.

\begin{definition}
Let $A$ be a $C^{*}$-algebra. The Cone of $A$, denoted $CA$, is the set of functions: $\lbrace f \in C([0,1],A) | f(0)=0 \rbrace$. The suspension of $A$, denoted $SA$, is a subalgebra of the cone given by functions: $\lbrace f \in CA | f(1)=0 \rbrace$. We define the higher K-groups via suspensions: $K_{n}(A):=K_{0}(S^{n}A)$.
\end{definition}

We remark that these definitions are equivalent for $K_{1}$.

\begin{theorem}(Bott Periodicity)
There is an isomorphism $K_{0}(A)\cong K_{0}(S^{2}A)$.
\end{theorem}

The proof of this result relies on constructing the \textit{Bott map} $\beta$, which converts the long exact sequence into a \textit{cyclic} exact sequence of length $6$:

\begin{theorem}
Given a short exact sequence $0 \rightarrow J \rightarrow A \rightarrow \frac{A}{J} \rightarrow 0$ there is a cyclic long exact sequence:
\begin{equation*}
\xymatrix@=1em{
K_{0}(J) \ar[r] & K_{0}(A) \ar[r]& K_{0}(\frac{A}{J})\ar[d] & \\
K_{1}(\frac{A}{J})\ar[u]^{\mu}& K_{1}(A) \ar[l]& K_{1}(J)\ar[l] &
}
\end{equation*}
\end{theorem}

This six term sequence is a key tool in computations concerning K-theory of $C^{*}$-algebras related in an extension.

The last ideas that are outlined in this section are those of Pimsner and Voiculescu on actions of groups on $C^{*}$-algebras via automorphisms.

\begin{definition}
Let $A$ be a $C^{*}$-algebra represented on a Hilbert space $\mathcal{H}$ and let $\rho: G \rightarrow Aut(A)$ be a representation of a group $G$. Then we can naturally form the $C^{*}$-algebra $A\rtimes_{r}G$; it is the completion of $C_{c}(G,A)$ equipped with a twisted convolution and completed in the norm that arises from the represention in $\mathcal{B}(\mathcal{H}\oplus \ell^{2}(G))$.
\end{definition}

The following is original result of Pimsner-Voiculescu concerning the structure of infinite cyclic group actions \cite{MR670181}:

\begin{theorem}\cite[Theorem 10.2.1]{MR1222415}
Let $A$ be a $C^{*}$-algebra and let $\alpha \in Aut(A)$. Then there is a cyclic $6$-term exact sequence:
\begin{equation*}
\xymatrix@=1em{
K_{0}(A) \ar[r]^{1-\alpha_{*}} & K_{0}(A) \ar[r]^{i_{*}}& K_{0}(A\rtimes \mathbb{Z})\ar[d] & \\
K_{1}(A \rtimes \mathbb{Z})\ar[u]& K_{1}(A) \ar[l]^{i_{*}}& K_{1}(A)\ar[l]^{1-\alpha_{*}} &
}
\end{equation*}
\end{theorem}

The main idea of this result is that it gives an understanding of a crossed product structure by looking at only the induced action on the K-theory groups of the coefficient algebra. The main issue is that in general these can be as bad as one wants, so computations of the action could be particularly difficult.

These ideas naturally extend to free group actions by automorphisms \cite{MR670181}:

\begin{theorem}\cite[Theorem 10.8.1]{MR1222415}\label{thm:BPV2}
Let $\alpha_{i}$, $i \in \lbrace 1,...,k \rbrace$ be elements of $Aut(A)$ that give a representation of $F_{k}$. Then there is a cyclic $6$-term sequence:
\begin{equation*}
\xymatrix@=1em{
\bigoplus_{i=1}^{k}K_{0}(A) \ar[r]^{\rho} & K_{0}(A) \ar[r]^{i_{*}}& K_{0}(A\rtimes_{r} F_{k})\ar[d] & \\
K_{1}(A \rtimes_{r} F_{k})\ar[u]& K_{1}(A) \ar[l]^{i_{*}}& \bigoplus_{i=1}^{k} K_{1}(A)\ar[l]^{\rho} &
}
\end{equation*}
with $\rho:= \sum_{i=1}^{k}(1-\alpha_{i,*})$.
\end{theorem}

The major Corollary of this result gives a computation of the K-theory of a Free group $C^{*}$-algebra inductively by considering the action on $\mathbb{C}$.

Generalisations of this situation to F-inverse monoids play the main role in the next chapter.