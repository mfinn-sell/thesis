%\setstretch{1.6}
\chapter{Partial Translations and Inverse Semigroups.}

In this chapter we outline the construction of a short exact sequence of $C^{*}$-algebras associated to an F-inverse monoid $S$. This relates the reduced $C^{*}$-algebra of $S$ to the reduced $C^{*}$-algebra of its maximal group homomorphic image, generalising some of the ideas present in the proof of the Pimsner-Voiculescu short exact sequence from \cite{MR670181}. We then make use of this result in a metric context; to any subspace $X$ of a given finitely generated discrete group $G$ we associate an object called a \textit{partial translation structure} to $X$. This has a naturally associated inverse monoid and we investigate precisely when this inverse monoid is F-inverse. In this case, we construct an analogue of the short exact sequence for these inverse monoids. This has applications within K-theory, which we discuss at the beginning of the Chapter 5.

\section{A Pimsner-Voiculescu short exact sequence for an F-inverse monoid.}\label{sect:S1}
In this section we construct a sequence of $C^{*}$-algebras that can naturally be associated to an F-inverse monoid and prove that it is exact. The process of doing this requires careful analysis of the representation theory of the universal groupoid associated to the inverse monoid in question. Later in Section \ref{sect:S1-a} of this chapter we utilise this machinery to prove the following result, which is Theorem \ref{thm:PV1} in the text.

Let $G$ be the maximal group homomorphic image of $S$.

\begin{thm}
Let $S$ be an F-inverse monoid and let $\G_{\E}$ be its universal groupoid. Then there is a distinguished element, denoted by $\infty$, of $\E$. We denote the compliment of $\infty$ by $U$, which is open and saturated and let $A=C_{c}(\G_{U})$. Then we have the following short exact sequence of $C^{*}$-algebras:
\begin{equation*}
0 \rightarrow \overline{A} \rightarrow C^{*}_{r}(\mathcal{G}) \rightarrow C^{*}_{r}(G) \rightarrow 0.
\end{equation*}
\end{thm}

The first step in this is to understand the representation theory of the universal groupoid.

\begin{lemma}\label{lem:L2}
Let $S$ be a 0-F-inverse monoid, let $\G=\G_{\E}$ be the universal groupoid and let $\lbrace L^{2}(\mathcal{G}_{x}) \rbrace_{x \in \E}$ be the field of Hilbert spaces associated with $\mathcal{G}$. Let $x,y \in \E}$ such that $x \subset y$ Then there exists a projection $Q_{y,x}: L^{2}(\mathcal{G}_{y}) \rightarrow L^{2}(\mathcal{G}_{x})$ such that $\lambda_{x}(1_{tt^{*}}\delta_{t}) = Q_{y,x}\lambda_{y}(1_{tt^{*}}\delta_{t})Q_{y,x}^{*}$.
\end{lemma}

\begin{proof}
A basis for $L^{2}(\mathcal{G}_{x})$ is given by Dirac functions of its elements, i.e $\lbrace \delta_{[s,x]} : [s,x] \in \mathcal{G}_{x} \rbrace$. Claim \ref{MainClaim:C1}improves this by considering the maximal element in each equivalence class, $\lbrace \delta_{[t_{s},x]} : [s,x] \in \mathcal{G}_{x} \rbrace$. Let $L_{x} = \lbrace t \in \text{Max(S)} : [t,x] \in \mathcal{G}_{x} \rbrace$. As $x \subset y$ we know that $L_{x} \subset L_{y}$ and this  allows us to construct the projection from $L^{2}(\mathcal{G}_{y})$ on the basis in the following way:
\begin{equation}
Q_{y,x}(\delta_{[t,y]})= \begin{cases} \delta_{[t,x]} \mbox{ if } t \in L_{x} \\ 0 \mbox{ else} \end{cases}  
\end{equation}

This function is clearly surjective; we extended this linearly. To see that this is a bounded operator we observe that truncation of a Hilbert space element to a subset is norm decreasing.

Now to see the last part of the lemma we appeal to the definition of the convolution. Let $v = \sum_{u \in L_{x}} a_{u}\delta_{[u,x]} \in L^{2}(\mathcal{G}_{x})$. Then
\begin{equation*}
\lambda_{x}(1_{tt^{*}}\delta_{t})(v)([m,x])=\sum_{\substack{[n,y][u,x]\\=[m,x]}} 1_{tt^{*}}([n,y])v([u,x])=\sum_{\substack{[t,y][u,x]\\=[m,x]}}v([u,x])=v([u,x])
\end{equation*}
Where $[m,x]=[m_{tu},x]$ is the maximal representative of the element $[tu,x]$ using Claim \ref{MainClaim:C1}. We see that:
\begin{equation*}
\lambda_{x}(1_{tt^{*}}\delta_{t})(\delta_{[u,x]})= \delta_{[m_{tu},x]} \mbox{ if } u\in L_{x} \mbox{ and } m_{tu} \in L_{x}
\end{equation*}
So $\lambda_{x}(1_{tt^{*}}\delta_{t})$ acts on those elements $[u,x]$ for whom there exists a maximal element $m$ and a $y \in \widehat{E}$ such that $[m,x]=[tu,x]$. 

\begin{figure}\label{fig:F1}


\xymatrix@=100{
& x \ar@/_/[rr]_{[m_{tu},x]} \ar@/^/[r]^{[u,x]} & {y} \ar@/^/[r]^{[t,y]} & {z}
}



\caption{The action of $\lambda_{x}(1_{tt^{*}}\delta_{t})$}

\end{figure}

Now consider what happens for a general element $v = \sum_{u \in L_{x}} a_{u}\delta_{[u,x]} \in L^{2}(\mathcal{G}_{x})$. We get the following:
\begin{equation}\label{eqn:LE3}
Q_{y,x}(\lambda_{y}(1_{tt^{*}}\delta_{t}))Q_{y,x}^{*}(v)=Q_{y,x}(\lambda_{y}(1_{tt^{*}}\delta_{t}))(v^{'})
\end{equation}
where $v^{'}= \sum_{u \in L_{y}} a_{u}\delta_{[u,y]} \in L^{2}(\mathcal{G}_{y})$ with $a_{u}=0$ if $u \not \in L_{x}$. Then 
\begin{eqnarray*}
\mbox{(\ref{eqn:LE3})} = Q_{y,x}(\sum_{\substack{ m_{tu}\\ u \in L_{x}}} a_{u}\delta_{[m_{tu},y]})
=\sum_{\substack{ m_{tu}\\  u \in L_{x},m_{tu} \in L_{x}}}a_{u}\delta_{[m_{tu},x]}=\lambda_{x}(1_{tt^{*}}\delta_{t})(v)
\end{eqnarray*}
\end{proof}

We specialise this result in the case that $S$ is F-inverse. As such a monoid has no zero element the function $1_{E}: E \rightarrow \lbrace 0,1 \rbrace$ that assigns $1$ to every idempotent is a valid character. We denote the ultrafilter that corresponds to that character by $\infty$.

\begin{corollary}\label{cor:C1}
Let $S$ be an F-inverse monoid and let $\mathcal{G}=\mathcal{G}_{\widehat{E}}$ be its universal groupoid and let $\lbrace L^{2}(\mathcal{G}_{x}) \rbrace_{x \in \widehat{E}}$ be the field of Hilbert spaces associated with $\mathcal{G}$. Then for each $x \in \E \setminus \lbrace 1_{E} \rbrace$ there exists a projection map $Q_{x}: L^{2}(\mathcal{G}_{\infty}) \rightarrow L^{2}(\mathcal{G}_{x})$ such that $\lambda_{x}(1_{tt^{*}}\delta_{t}) = Q_{x}\lambda_{\infty}(1_{tt^{*}}\delta_{t})Q_{x}^{*}$.
\end{lemma}
\begin{proof}
The ultrafilter $\infty$ contains all filters of $E(S)$. We apply Lemma \ref{lem:L2} to construct maps $Q_{x}=Q_{\infty,x}$ for each $x \in \E \setminus \lbrace 1_{E} \rbrace$ .
\end{proof}

\subsection{Representations of $C_{c}(\G_{\E})$ for an F-inverse monoid}
We discuss representations of an F-inverse monoid. We make use of the following result from \cite{MR1900993}:

\begin{proposition}\label{prop:P3} \mbox{ \cite[Cor 2.4]{MR1900993} }
For a dense subset $D \subset \widehat{E}$ we have $\Vert f \Vert_{r} = \Vert \lambda(f) \Vert = \sup \lbrace \Vert \lambda_{x}(f) \Vert : x \in D \rbrace$.
\end{proposition}
This is useful as the idempotents $E$ are dense in $\E$ as $\E$ is a compactification of $E$. Additionally, recall that $\E_{tight}$ is the closure of $\E_{\infty}$ in $\E$.

\begin{definition}
An idempotent $e \in E$ is \textit{primitive} if $e$ is minimal amongst the elements of $E(S)\setminus\lbrace 0 \rbrace$.
\end{definition}

We denote by $\G_{tight}$ the restriction of $\G_{\E}$ to $\E_{tight}$. We can truncate to build a quotient from $C^{*}_{r}(\G)$ onto $C^{*}_{r}(\G_{tight})$:

\begin{proposition}\label{prop:P4}
Let $S$ be an 0-F-inverse monoid with no primitive idempotents and let $\G=\G_{\E}$ be its universal groupoid. Then we have a surjective *-homomorphism from $C^{*}_{r}(\G)$ onto $C^{*}_{r}(\G_{tight})$.
\end{proposition}

\begin{proof}
We construct the map $q$ using truncation of functions:
\begin{equation*}
q: \sum_{t \in \text{Max(S)}} f_{t} \delta_{t} \mapsto \sum_{t \in \text{Max(S)}} f_{t}|_{\E_{tight}} \delta_{t} 
\end{equation*}
We need to show that
\begin{enumerate}
\item The map $q$ is contractive
\item The map $q$ is a *-homomorphism.
\end{enumerate}
To tackle (1) we consider the regular representations of $f = \sum_{t \in \text{Max(S)}} f_{t} \delta_{t}$ and $qf$ respectively. Using Remark \ref{rem:Rep} and the following (commuting) diagram

\begin{center}
\begin{tikzpicture}
\matrix(m)[matrix of math nodes,row sep=3em, column sep=3em, text height=1.5ex, text depth = 0.25ex]
{C_{c}(\G_{\E})&C_{c}(\G_{tight})\\
\mathcal{B}(L^{2}(\G_{\E}))&\mathcal{B}(L^{2}(\G_{\E_{tight}}))\\};
\path[->,font=\scriptsize]
(m-1-1) edge node[auto] {$q$} (m-1-2)
(m-2-1) edge node[auto] {$p$} (m-2-2)
(m-1-1) edge node[auto] {$\lambda_{tight}$} (m-2-2);
\path[right hook->]
(m-1-1) edge node[auto] {$\lambda$} (m-2-1)
(m-1-2) edge node[auto] {$\lambda_{R}$} (m-2-2);

\end{tikzpicture}
\end{center}

where $\lambda_{R}$ is the left regular representation of $C_{c}(\G_{tight})$. It follows from the definition of $p$ that the bottom triangle commutes and the top triangle commutes as:
\begin{equation*}
\lambda_{x}(f) =\sum_{t \in \text{Max(S)}} f_{t}(x)\lambda_{x}(1_{tt^{*}}\delta_{t}) = \lambda_{x}(qf)
\end{equation*}
For each $x \in \E_{tight}$. Hence:
\begin{equation*}
\Vert qf \Vert_{r} = \sup_{x \in \E_{tight}} \lbrace \Vert \lambda_{x}(qf) \Vert \rbrace = \sup_{x \in \E_{tight}} \lbrace \Vert \lambda_{x}(f) \Vert \rbrace \leq \Vert \lambda(f) \Vert = \Vert f \Vert_{r}
\end{equation*}
and so $q$ is contractive (and therefore continuous).
 
Now to consider (2). It is enough to compute the result of products of elements of the form $f_{s}\delta_{s}$ for some $s \in \text{Max(S)}$. We then check the following two identities:


\begin{enumerate}[I]
\item $q(f_{s}\delta_{s} \Xst f_{t}\delta_{t}) = q(f_{s}\delta_{s})\Xst q(f_{t}\delta_{t})$
\item $(q(f_{s}\delta_{s}))^{*}=q((f_{s}\delta_{s})^{*})$
\end{enumerate}


To see (I) compute on a single element:
\begin{eqnarray*}
(f_{s}\delta_{s} \Xst f_{t}\delta_{t})([st,\phi])=f_{s}(\theta_{t}(\phi))f_{t}(\phi)
\end{eqnarray*}
Apply $q$:
\begin{eqnarray*}
q(f_{s}\delta_{s} \Xst f_{t}\delta_{t})([st,\phi])= (f_{s}\delta_{s} \Xst f_{t}\delta_{t})|_{\E_{tight}}([st,\phi]) =f_{s}(\theta_{t}(\phi))f_{t}(\phi)
\end{eqnarray*}
For all $[st,\phi] \in \G_{\E_{tight}}$. Then compute the right hand side: 
\begin{eqnarray*}
(q(f_{s}\delta_{s}) \Xst q(f_{t}\delta_{t}))([st,\phi])=f_{s}|_{\E_{tight}}(\theta_{t}(\phi))f_{t}|_{\E_{tight}}(\phi)
\end{eqnarray*}
Which matches for each $[st,\phi] \in \G_{\E_{tight}}$. 

To prove (II) we need to compute on a single element, where $(f_{s}\delta_{s})^{*}=\alpha_{s^{*}}(\overline{f_{s}})\delta_{s^{*}}$:

\begin{eqnarray*}
q((f_{s}\delta_{s})^{*})([s^{*},x])& = &\alpha_{s^{*}}(\overline{f_{s}})_{\E_{tight}}(x)\\
& = & \overline{f}(\theta_{s}(x)) \\ & = & \overline{f(\theta_{s}(x))} \\ & = & \overline{q(f)}(\theta_{s}(x)) \\ & = &  \alpha_{s^{*}}(\overline{q(f)})(x)
\end{eqnarray*}
Where the above holds for all $x \in \E_{tight}$ where the function $f_{s}$ is defined at $\theta_{s}(x)$ as required.

As $q$ is a continuous *-homomorphism, it extends to the completions.
\end{proof}

\subsection{Applying the machinery}\label{sect:S1-a}
We encode the norm estimations required for the proof of Theorem \ref{thm:PV1} in the following Lemma:
\begin{lemma}\label{lem:L3}
Let $S$ be F-inverse and let $K \subset \text{Max(S)}$ such that $K$ is finite and $T=\sum_{t \in K} a_{t}\lambda(1_{tt^{*}}\delta_{t})$ such that $a_{t}$ is the constant function that has value $a_{t}$ on $D_{tt^{*}}$. Then $\Vert T \Vert = \Vert qT \Vert$
\end{lemma}
\begin{proof}
It is immediate that $\Vert T \Vert \geq \Vert qT \Vert$ as $q$ is contractive. We arrive at the other inequality by applying Corollary \ref{cor:C1}.
\begin{eqnarray*}
\Vert T \Vert_{L^{2}(\mathcal{G}_{x})} = \Vert \sum_{t \in K} a_{t}\lambda_{x}(1_{tt^{*}}\delta_{t}) \Vert = \Vert \sum_{t \in K} a_{t}Q_{x}\lambda_{\infty}(1_{tt^{*}}\delta_{t})Q_{x}^{*} \Vert \\
= \Vert Q_{x}(\sum_{t \in K} a_{t}\lambda_{\infty}(1_{tt^{*}}\delta_{t}))Q_{x}^{*} \Vert = \Vert Q_{x}(qT)Q_{x}^{*} \Vert \leq \Vert qT \Vert.
\end{eqnarray*}
This holds for every $x \in E$ and so by  $\Vert T \Vert = \Vert \lambda(T) \Vert = \sup \lbrace \Vert \lambda_{x}(T) \Vert : x \in E \rbrace \leq \Vert qT \Vert$. This gives the desired equality.  
\end{proof}

We remark here that if $S$ is F-inverse then minimal elements do not play a role in the ultrafilters, which was the reason for removing them when $S$ had a zero. Additionally, in this instance the groupoid $\G_{tight}$ just the maximal group homomorphic image $G$. 

\begin{theorem}\label{thm:PV1}
Let $S$ be an F-inverse monoid, let $\G_{\E}$ be its universal groupoid and let $A=C_{c}(\G_{U})$. Then we have the following short exact sequence of $C^{*}$-algebras:
\begin{equation*}
0 \rightarrow \overline{A} \rightarrow C^{*}_{r}(\mathcal{G}) \rightarrow C^{*}_{r}(G) \rightarrow 0
\end{equation*}
\end{theorem}
\begin{proof}

We know that we have a surjective *-homomorphism $q$ from $C^{*}_{r}(\G)$ to $C^{*}_{r}(G)$, we just need to see that the kernel of this map is $\overline{A}$. The set $\overline{A}$ is contained in the kernel as elements in $A$ are sums of functions with value at $1_{E}=\infty \in \widehat{E}$ of zero and projection onto this value (i.e applying q) will send the entire element to $0 \in C^{*}_{r}(G)$. So it is enough to show that $A$ is dense in the kernel.

First consider finite sums. Let $f \in C_{c}(\mathcal{G})$. We need to show that if $qf=0$ then $ f \in A$.

$f$ has the form:
\begin{equation*}
f=\sum_{s \in S} f_{s}\delta_{s} \mbox{ where } f_{s}\in C(D_{ss^{*}})
\end{equation*}
With only finitely many non-zero terms. This can be viewed concretely on $L^{2}(\mathcal{G})$ using
\begin{equation*}
\lambda(f)=\sum_{s \in S} f_{s}\lambda(1_{ss^{*}}\delta_{s})
\end{equation*}
As $S$ is F-inverse we can reduce this sum using the observation that for each $s \in S$ we can write the term $f_{s}\delta_{s}$ as $f_{s}\chi_{s}\delta_{t_{s}}$ where $t_{s}$ is the maximal element above $s$. So for each $t \in \text{Max(S)}$ we can define $f^{'}_{t}=\sum_{s \leq t}f(s)\chi_{s}$ and then
\begin{equation}\label{Eq1}
\lambda(f)=\sum_{t \in \text{Max(S)}} f^{'}_{t}\lambda(1_{tt^{*}}\delta_{t})
\end{equation}
(\ref{Eq1}) is in the kernel of $q$ if and only if each $f^{'}_{t}(\infty)=0$ for every $t \in \text{Max(S)}$ that is if and only if $\lambda(f) \in A$.

Now let $T$ be an element of $C^{*}_{r}(\G})$ such that $qT=0$. Then we need to show $T$ can be approximated by finite sums that lie in $A$. Let $T_{n}$ be finite sums in $C_{c}(\mathcal{G})$ with $T_{n} \rightarrow T$. Without loss of generality, these $T_{n}$ have the following form for some finite $K_{n} \subset \text{Max(S)}$:
\begin{equation*}
T_{n}=\sum_{t \in K_{n}} f^{n}_{t}\lambda(1_{tt^{*}}\delta_{t})
\end{equation*}
then $qT_{n} = \sum_{t \in K_{n}} f^{n}_{t}(\infty)\lambda_{\infty}(1_{tt*}\delta_{t})$. Define a pullback of $qT_{n}$:
\begin{equation}
S_{n} = \sum_{t \in K_{n}} a^{n}_{t}\lambda(1_{tt*}\delta_{t}) \in C_{c}(\mathcal{G})
\end{equation}
Where $a^{n}_{t}$ is the constant function on $D_{tt^{*}}$ with value $f_{t}^{n}(\infty)$. It is clear that $qS_{n}=qT_{n}$ and using Lemma \ref{lem:L3} we have that $\Vert S_{n} \Vert = \Vert qS_{n} \Vert = \Vert qT_{n} \Vert$ so $\Vert S_{n} \Vert \rightarrow 0$\\
\\
Let $U_{n}=(T_{n}-S_{n})$. Then $U_{n} \in A$ and $U_{n}=(T_{n}-S_{n}) \rightarrow (T-0)=T$, whence $A$ is dense in $ker(q)$.
\end{proof} 

\section{A similar sequence for 0-F-inverse monoids}\label{sect:S2}
In this section we consider a generalisation of Theorem \ref{thm:PV1} to strongly 0-F-inverse monoids with non-trivial K-exact universal group. We consider saturated subsets of the unit space as defined in Chapter 2. Clearly, subsets that are invariant under the action of $S$ are also saturated. The following Lemma outlines the connections between saturation and Morita enveloping actions.

\begin{lemma}\label{Lem:Cut}
Let $\G$ be an \etale locally compact Hausdorff groupoid with a (T,C,F)-cocycle $\rho$ to $\Gamma$. Then relation $\sim$ on $\G^{(0)} \times \Gamma$ preserves saturated subsets of $\G^{(0)}$
\end{lemma}
\begin{proof}
Let $U$ be a saturated subset of $\G^{(0)}$ and let $x \in U$, $y \in U^{c}$. Assume for a contradiction that $(x,g) \sim (y,h)$ in $\G^{(0)} \times \Gamma$. Then there exists a $\gamma$ $\in \G$ such that $s(\gamma)=x$, $r(\gamma)=y$ and $\rho(\gamma)=g^{-1}h$, but as $U$ is saturated no such $\gamma$ exists. 
\end{proof}

\begin{theorem}\label{thm:PV2}
Let $S$ be a strongly $0$-F-inverse monoid with universal group $G:=U(S)$. If $G$ is infinite and $K$-exact then the sequence:
\begin{equation*}
0 \rightarrow C^{*}_{r}(\G_{\U}) \rightarrow C^{*}_{r}(\G_{\E}) \rightarrow C^{*}_{r}(\G_{\E_{tight}}) \rightarrow 0
\end{equation*}
is exact at the level of K-theory.  
\end{theorem}
\begin{proof}
We begin by using either Theorem \ref{Thm:IT2-a} or \ref{Thm:IT2} to construct a transformation groupoid $Y_{\E}\rtimes G$ and a Morita equivalence between $\G_{\E}$ and $Y_{\E}\rtimes G$. We can repeat this process for both $\E_{tight}$ and $U:=\E_{tight}^{c}$, and by Lemma \ref{Lem:Cut} and the fact that $\E_{tight}$ is closed in $\E$ we can conclude that we have a natural sequence of commutative $C^{*}$-algebras:
\begin{equation*}
0 \rightarrow C_{0}(Y_{U}) \rightarrow C_{0}(Y_{\E}) \rightarrow C_{0}(Y_{\E_{tight}}) \rightarrow 0
\end{equation*}
each of which is a $G$-algebra. We now act by the reduced cross product, which produces a sequence of $C^{*}$-algebras:
\begin{equation*}
0 \rightarrow C_{0}(Y_{U})\rtimes_{r}G \rightarrow C_{0}(Y_{\E})\rtimes_{r}G \rightarrow C_{0}(Y_{\E_{tight}})\rtimes_{r}G \rightarrow 0.
\end{equation*}
This may not be exact in the middle term. However, it is exact at the level of K-theory, so consider the K-theory long exact sequence:
\begin{equation*}
\xymatrix@=1em{...\ar[r] & K_{0}(C_{0}(Y_{U})\rtimes_{r} G) \ar[r]& K_{0}(C_{0}(Y_{\E})\rtimes_{r} G) \ar[r]& K_{0}(C_{0}(Y_{\E_{tight}})\rtimes_{r} G)\ar[r] & ...\\
...\ar[r] & K_{0}(C^{*}_{r}(\G_{\U})) \ar[r]\ar[u]^{\ucong}& K_{0}(C^{*}_{r}(\G_{\E})) \ar[r]\ar[u]^{\ucong}& K_{0}(C^{*}_{r}(\G_{\E_{tight}})) \ar[r]\ar[u]^{\ucong}& ...}
\end{equation*}
where the isomorphisms are induced by the Morita equivalences given by Theorems \ref{Thm:IT2-a} and \ref{Thm:IT2}. This concludes the proof.
\end{proof}


\section{Translation Structures to Inverse Monoids}
In this section we outline the definition of a partial translation structure and describe some of the results concerning them from the literature. Focusing on a special case, which we call \textit{grouplike partial translation structures}, we connect uniform embeddability into groups for metric spaces to translation structures.  We then outline an inverse monoid approach to understanding the translation algebra that can be naturally associated to a partial translation structure.

The concept of a partial translation structure was first introduced in \cite{MR2363428}. By associating to a metric space this additional information, namely a collection of partial bijections that form entourages in the metric coarse structure, it is possible to use the local symmetries of the space to control the metric. 


\begin{definition}\label{PT2}
A partial translation structure on $X$ is a collection $\mathcal{T}$ of partial translations of $X$ such that for all $R>0$ there exists a finite subset $\mathcal{T}_{R}$ of disjoint partial translations in $\mathcal{T}$ and a collection $\Sigma_{R}$ of partial cotranslations of $\mathcal{T}_{R}$ satisfying the following axioms:
\begin{enumerate}
\item The union of the partial translations  $t \in \mathcal{T}_{R}$ contains the R-neighbourhood of the diagonal.
\item There exists $k$ such that for each $x,x^{'} \in X$ there are at most $k$ elements $\sigma \in \Sigma_{R}$ such that $\sigma x=x^{'}$.
\item For each $t \in \mathcal{T}_{R}$ and for all $(x,y),(x^{'},y^{'}) \in t$ there exists $\sigma \in \Sigma_{R}$ such that $\sigma x=x^{'}$ and $\sigma y=y^{'}$.
\end{enumerate}
\end{definition}

\begin{definition}(Freeness, Global control)
A partial translation structure on X is said to be \textit{free} if in Definition \ref{PT2} ii) $k=1$; i.e there is a unique cotranslation such that for each pair $(x,y)\in X \times X$ we have $\sigma x = y$.\\
A partial translation structure on X is said to be \textit{globally controlled} if the partial cotranslation orbits are partial translations. 
\end{definition}

The following is Theorem 19 from \cite{MR2363428}.

\begin{lemma}\label{lem:L10}
Let $G$ be a group equipped with a proper left invariant metric and let $X \subseteq G$ equipped with the induced metric. Then the restriction of the action of $G$ on itself by right multiplication to $X$ is a partial translation structure that is free and globally controlled.
\end{lemma}

The intuition for the Definition \ref{PT2} is a metric version of a group action for spaces, with freeness and global control giving conditions that are similar to a free and transitive action of group. 

\begin{definition}\label{def:ZD1}
Let $\mathcal{T}$ be a partial translation structure. Then we say $\mathcal{T}$ has \textit{zero divisors} if there exists a product of disjoint translations $t_{1},t_{2},...,t_{n} \in \mathcal{T}$ such that $t_{1}t_{2}t_{3}...t_{n}$ is empty (i.e has empty domain). We say $\mathcal{T}$ has no zero divisors if no such product is empty.
\end{definition}
We specialize our definition slightly in light of the following proposition, the proof of which can be found in \cite[Proposition 8.1]{rosiesthesis}

\begin{proposition}\label{prop:TFAE} Let $G$ be a countable discrete group and let $X \subseteq G$
The following are equivalent:
\begin{enumerate}
\item $X^{c}$ is not coarsely dense in $G$.
\item For every $R>0$ there exists $g \in G$ such that $B_{R}(g) \subseteq X$.
\item The monoid generated by $\mathcal{T}_{G}|_{X}$ has no zero element.
\end{enumerate}
\end{proposition}
The definition provided below is stronger than the definition provided in \cite{MR2363428}, however this better emulates the situation that arises when you consider a space that is uniformly embedded into a group.

\begin{definition}\label{Def:PTS}
Let $X$ be a countable discrete metric space. A collection of partial bijections $\mathcal{T}$ is a called a \textit{grouplike partial translation structure} for $X$ if:
\begin{enumerate}
\item $\mathcal{T}$ partitions $X\times X$.
\item $\forall t_{i}, t_{j} \in \mathcal{T}$ $\exists t_{k} \in \mathcal{T}$ we have $t_{i}t_{j} \subseteq t_{k}$ (i.e $\mathcal{T}$ is subclosed).
\item $\forall t \in \mathcal{T}$ we have $t^{*} \in \mathcal{T}$.
\item $\mathcal{T}$ has a global identity, denote this $t_{0}$.
\end{definition}

As a consequence of the Wagner-Preston Theorem \cite{MR1455373} partial bijections move us toward inverse semigroup theory.
\begin{proposition}\label{prop:P6}
Let $X$ be a metric space equipped with a group-like partial translation structure $\mathcal{T}$. Then $\mathcal{T}$ generates an inverse submonoid of $I_{b}(X)$
\end{proposition}
\begin{proof}
The axioms for a grouplike structure tell us that for every translation we have the adjoint translation - that acts an inverse. As $\mathcal{T}$ partitions $X \times X$ $t^{*}$ is unique for every $t \in \mathcal{T}$. These elements are partial bijections on $X$, and so are elements of $I_{b}(X)$ and we can consider the subsemigroup they generate. As each element of this subsemigroup has a unique inverse we get that it must be an inverse subsemigroup. The presence of the global identity map in $\mathcal{T}$ will give a global identity in the subsemigroup it generates. Hence the subsemigroup is a submonoid, as required.
\end{proof}

We can characterise the monoids generated by partial translation structures:

\begin{lemma}\label{Lem:PTS}
The inverse monoid $S$ generated by a partial translation structure $\mathcal{T}$ is 0-F-inverse, with maximal element set $\lbrace t_{i} : t_{i} \in \mathcal{T} \rbrace$. 
\end{lemma}
\begin{proof}
First we prove maximality of the translations. We prove that for any $s\in S \setminus \lbrace 0 \rbrace$ there exists a unique $t \in \mathcal{T}$ such that $s \leq t$. Property (2) from the Definition \ref{Def:PTS} implies that $\mathcal{T}$ generates the partial order on $S$. As $\mathcal{T}$ partitions $X \times X$ we have that for any pair $t_{i},t_{j}\in \mathcal{T}$ $et_{i}=et_{j} \Leftrightarrow t_{i}=t_{j}$. 

Now we prove that $S$ is $0$-E-unitary. Let $e\in E(S)\setminus \lbrace 0 \rbrace$ and $s\in S\setminus \lbrace 0 \rbrace$. Without loss of generality we can treat $s$ as maximal in what follows. Assume that $e \leq s$. This gives us two equations: $es=e \leq s$. As the natural order is preserved by taking inverses we see that $s^{*}e \leq s^{*}$. These imply that $s=s^{*}$. This tells us that $s^{2}$ is idempotent, but we want $s$ idempotent. To show this we will aim for $s^{2}=s^{3}$. Observe that $es=es^{2}\leq s^{2} \leq s$. This implies $s^{2}=fs$ for some $f\in E(S)$. $s^{3}=s^{2}s=(fs)s=f^{2}s=fs=s^{2}$. As $s=s^{3}$ we get $s \in E(S)$ as required.
\end{proof}

\subsection{An Embeddability Theorem for Metric Spaces with Grouplike Partial Translation Structures.}

The precise nature of the relationship between partial translation structures in the sense of Definition \ref{Def:PTS} and uniform embeddings is understood. It follows from Theorem 19 of \cite{MR2363428} that given any space that admits a uniform embedding into a group, we can equip it with a translation structure given by the Definition \ref{Def:PTS}. The inverse monoid generated by this translation structure is also understood from Lemma \ref{Lem:PTS}.

In this section we provide a partial converse to Theorem 19 of \cite{MR2363428}:

\begin{theorem}\label{thm:T2}
Let $X$ be a countable discrete metric space equipped with a grouplike partial translation structure $\mathcal{T}$, where $\mathcal{T}$ has no zero divisors. Then there exists a countable discrete group $G$ and an embedding $X \hookrightarrow G$ such that the translation structure provided by $G$ restricted to $X$ denoted $\mathcal{T}_{G}|_{X}$ is equal to $\mathcal{T}$.
\end{theorem}

\begin{proof}
Consider the inverse monoid $S = \langle \mathcal{T} \rangle$. $\mathcal{T}$ has no zero divisors implies that $S/\sigma$ is a non-trivial group. Denote that group by $G$. The aim now is to embed $X$ into $G$. The maximal elements in $\mathcal{T}$ generate this group, and $\sigma$ induces an inverse semigroup homomorphism from $S$ into $G$, which is a bijection between the maximal elements and $G$. Denote by $T_{x_{0}}$ the following:
\begin{equation}
T_{x_{0}} := \lbrace t \in \mathcal{T} : tx=x_{0} \rbrace
\end{equation}
where $x_{0}$ is a basepoint in $X$. Observe that because $\mathcal{T}$ partitions $X \times X$ we can construct a bijection between $X$ and $T_{x_{0}}$. Restricting to the image of $T_{x_{0}}$ under $\sigma$ we get a subspace of the group that is in bijection with $X$, i.e we can view $X$ as a subset of the group $G$. To finish the proof, we need the translation structure $\mathcal{T}$ to come from the group. We can construct this as follows. Take a translation $t_{j} \in \mathcal{T}$. For every $x \in Dom(t_{j})$ there exists a unique $t_{x} \in T_{x_{0}}$ such that $t_{x}x=x_{0}$. For each $x \in Dom(t_{j})$ there exists a unique $y \in X$ such that $t_{j}x=y$ Taking adjoints: $t_{j}^{*}y=x$. This gives a map: $t_{x}t_{j}^{*}y=x_{0}$ and $y$ corresponds to some element in $T_{x_{0}}$, denote this $t_{y}$. This gives the following situation:
\begin{equation}
t_{x}t_{j}^{*} \subseteq t_{y}.
\end{equation} 
Under $\sigma$ we have:
\begin{equation}
g_{x}g_{j}^{-1}=g_{y}
\end{equation}
This action on the right by inverses agrees with the typical translation structure of a group restricted to $X$, as we can define:
\begin{equation}
t_{g_{j}}:g_{x} \mapsto g_{y}, \mbox{ using $\sigma$; } x \mapsto y
\end{equation}
And this construction holds for all $x\in Dom(t_{j})$. This tells us that $Dom(t_{j}) \subseteq Dom(t_{g_{j}})$. All that remains is to show the reverse inclusion. Let $h \in Dom(t_{g_{j}})$ Then $h \in X \cap Xg_{j}$ so $h=h^{'}g_{j}$ and:
\begin{equation}
t_{g_{j}}:h \mapsto h^{'}
\end{equation}
Pulling $h$ and $h^{'}$ back into $X$ using the original bijection, we get a pair $(x,y) \in X \times X$. As $\mathcal{T}$ partitions $X\times X$ we have a unique $t_{p} \in \mathcal{T}$ such that $t_{p}x=y$. Via $\sigma$ we get the following situation:
\begin{equation}
h=h^{'}g_{p}=h^{'}g_{j} \Leftrightarrow g_{p}=g_{j}
\end{equation}
And pulling back this gives us $t_{p}=t_{j}$. So for every point $x\in Dom(t_{g_{j}})$ we have that $x \in Dom(t_{j})$.

Hence for each map in $\mathcal{T}$ we have a corresponding map in $\mathcal{T}_{G}|_{X}$ which is defined in the same places and is equal where it is defined. This implies $\mathcal{T}=\mathcal{T}_{G}|_{X}$ s required.
\end{proof}

The following is a direct corollary of Theorem \ref{thm:T2} and Proposition \ref{prop:TFAE}:

\begin{corollary}
The compliment of $\sigma (T_{x_{0}})$ is not coarsely dense in $G$.\qed
\end{corollary}


In summary, Theorem \ref{thm:T2} provides us a wealth of examples of F-inverse monoids with the added information of a concrete representation on an interesting metric space. It turns out that this provides a simplification to Theorem \ref{thm:PV1} when dealing with such representations.

\subsection{Translation Algebras}
Let $X$ be a uniformly discrete bounded geometry metric space.

\begin{definition}
The translation algebra associated with a partial translation structure $\mathcal{T}$ on $X$, denoted by $C^{*}\mathcal{T}$, is the completion as a *-subalgebra of $\mathcal{T}$ viewed as bounded operators on $\ell^{2}(X)$
\end{definition}

The aim of this section is to give a description of the partial translation algebra associated to a grouplike partial translation structure $\mathcal{T}$ with no zero divisors as the $C^{*}$-algebra of a groupoid, where the groupoid is related to the inverse monoid generated by the partial translations. We then recast Theorem 8.3 of Brodzki, Niblo, Putwain and Wright \cite{rosiesthesis} outlining a short exact sequence of $C^{*}$-algebras arising from such translation structures and compute some examples in certain cases.

Given the information of Lemma \ref{Lem:PTS} we have a inverse monoid that we can associate to a grouplike translation structure. This has a natural $C^{*}$-algebra, as outlined in Chapter 2. However, we have not used the geometric representation of this inverse monoid on $I(X)$, which determines a representation on $\ell^{2}(X)$ in the standard way. This representation will be the focus of this section. The following is Proposition 10.6 \cite{MR2419901}

\begin{proposition}\label{prop:P7} 
Let $\mu$ be a representation of $S$ on a Hilbert Space $H$. Then there exists a unique *-representation $\pi_{\mu}$ of $C_{0}(\E)$ on $H$ such that $\pi_{\mu}(1_{e})=\mu(e)$ for every $e \in E$ In addition $(\pi_{\mu} \times \mu)$ is a covariant representation for $\G_{\E}$. 
\end{proposition}

The proof of the above result relies on the spectrum of the commutative $C^{*}$-subalgebra $A=C^{*}_{\pi_{\mu}}(E)$ of $C^{*}_{\pi_{\mu}}(S)$. We denote the spectrum by $\X$. The key aspect of the proof of Proposition \ref{prop:P7} is the natural injective continuous map $j$ defined by: 
\begin{equation*}
j: \psi \in \X \mapsto \phi = \psi \circ \sigma \in \E
\end{equation*}

So given $X$ equipped with a grouplike partial translation structure with no zero divisors $\mathcal{T}$, we get an inverse monoid $S=\langle \mathcal{T} \rangle$ and a representation $\mu: S \hookrightarrow I_{b}(X)$ from Proposition \ref{prop:P6}. So applying Proposition \ref{prop:P7} we arrive at a representation $\pi_{\mu}$ of $C(\E)$ on $\ell^{2}(X)$. 

\begin{proposition}\label{prop:P9}
Let $S$ be a 0-F-inverse monoid and let $\mu: S \hookrightarrow I_{b}(X)$ be a geometric representation. Then the following hold for $\X$: \begin{enumerate}
\item $\X \hookrightarrow \E$ is a topological embedding
\item $\beta X \twoheadrightarrow \X$ is a quotient map.
\end{itemize}
Moreover the topologies are all compatable with the topology endowed as the spectrum of $A=C^{*}_{\pi_{\mu}}(E)$.
\end{proposition}
\begin{proof}
We give a concrete proof when $S$ has no 0: First we show (1) using the map $j$ defined above. $j(\X)$ is compact as j is continous and closed because $\E$ is Hausdorff.

For (2) we observe that the quotient map is given by the equivalence relation 
\begin{equation*}
\phi \sim \phi^{'} \leftrightarrow \phi \cap E(S) = \phi^{'} \cap E(S)
\end{equation*}
This map is surjective as given any $\psi \in \X$ we can view this as a filter on $X$ be considering the set:
\begin{equation*}
F_{\psi} = \lbrace e \in E(S) | \psi(\sigma(e))=1 \rbrace
\end{equation*}
We can complete this to an ultrafilter in $\beta X$ in many ways using Zorn's Lemma, however it is enough to show we can do it such that $F_{\psi ,UF}\cap E(S) = \psi$. So it is enough to pick subsets according to the following rules. Let $M,M^{c} \in \lbrace 0,1 \rbrace^{X}$ and
\begin{itemize}
\item If $M \in E(S)$ then add $M^{c}$ to $F_{\psi}$
\item If $M \not\in E(S)$ then add $M$ to $F_{\psi}$
\item If $M,M^{c} \not \in E(S)$ add either to $F_{\psi}$
\end{itemize}
With the case in which both $M$ and $M^{c}$ are contained in $E(S)$ is impossible as $E(S)$ has no zero element.

Now $F_{\psi}$ has the correct property and is an ultrafilter of $\beta X$ that maps onto $\psi$. Observe that the image of $\beta X$ is again compact, and thus closed, hence the map is a quotient.

In the case that $S$ has a zero element, we appeal to universal properties and another result of Exel \cite{MR2419901}. By Proposition 10.10 \cite{MR2419901} the space $\widehat{X}$ is closed and invariant.

Recall each $t \in \mathcal{T}$ is an element of $I_{b}(X)$ implies that the algebra $C^{*}_{\pi}(S)$ is a subalgebra of the uniform Roe algebra $C^{*}_{u}X$. We now remark that the representation $\pi_{X}$, when restricted to $C^{*}E$ assigns each idempotent a projection in $C^{*}_{u}X$, that is $C^{*}_{\pi_{\mu}}(E)=\pi_{}(C^{*}E) \subset \ell^{\infty}(X)$. Taking the spectra associated to this inclusion then gives us a map:
\begin{equation*}
r_{\beta X}: \beta X \twoheadrightarrow \widehat{X}
\end{equation*}
which is continuous. In particular as both $\beta X$ and $\widehat{X}$ are compact Hausdorff spaces, this map is closed (and open) and hence a quotient.
\end{proof}

\begin{corollary}\label{cor:C3}
Let $\psi_{x} = \lbrace x \rbrace^{\uparrow} \cap E(S)$. Then the set $\lbrace \psi_{x} | x \in X \rbrace$ is dense in $\X$.
\end{corollary}

In the most general situation the subspace $X$ may be stabilized under the right or left action of the group; we denote the left stablizer $LStab_{G}(X)$ by $H$. 

\begin{proposition}\label{prop:P9a}
$x_{1}x_{2}^{-1} \in H \Leftrightarrow \forall t \in \mathcal{T} (x_{1} \in Dom(t) \Leftrightarrow x_{2} \in Dom(t))$
\end{proposition}
\begin{proof}
($\Rightarrow$) $x_{1}x_{2}^{-1} \in H$ is equivalent to $x_{1}, x_{2} \in Hx$ for some $x \in X$. This implies that $(x_{1} \in Dom(t) \Leftrightarrow x_{2} \in Dom(t))$ as the elements of $H$ are contranslations of $X$. 

In fact we can say that the elements of $H$ are precisely the cotranslations that are bijections of $X$. This is key in proving the converse:

($\Leftarrow$) Consider $h=x_{1}x_{2}^{-1}$. We want to see that for all $x \in X$ $hx \in X$. Observe that by the first property of translation structures there exists a unique translation $t$ such that $t(x_{2})=x$ Then $hx=ht(x_{2})=t(hx_{2})=t(x_{1}) \in X$ and this chain of equalities holds precisely when $(x_{1} \in Dom(t) \Leftrightarrow x_{2} \in Dom(t))$

\end{proof}

\begin{corollary}\label{cor:C5}
$B$ is in bijection with $H \backslash X$
\end{corollary}
\begin{proof}
The righthand side of Proposition \ref{prop:P9a} is equivalent to the condition that $\psi_{x_{1}}=\psi_{x_{2}} \in B$.
\end{proof}

It is immediate (using \cite[Prop 10.10]{MR2419901}) that the set $\X$ is invariant under the action of $S$. To compute the groupoid and groupoid $C^{*}$-algebras associated to $\X$ we would like to know a little more about the Hilbert spaces associated to the fibers and general connectness of the set $B:=\lbrace \psi_{x} | x \in X \rbrace$ (which using Corollary \ref{cor:C3} is dense in $\X$).

\begin{proposition}\label{prop:P10}
Let $t \in \mathcal{T}$. Then $\theta_{t}(\psi_{x})=\psi_{t(x)}$ for all $x \in Dom(t)$.
\end{proposition}
\begin{proof}
First some observations:
\begin{enumerate}
\item $\theta_{t}(\psi_{x})$ is defined as $\theta_{t}(\psi_{x}) \in D_{tt^{*}} \Leftrightarrow \psi_{x} \in D_{t^{*}t} \Leftrightarrow t^{*}t \in \psi_{x} \Leftrightarrow x \in t^{*}t = Dom(t)$.
\item $(\theta_{t}(\psi_{x}))(tet^{*})=\psi_{x}(t^{*}(tet^{*})t)=\psi_{x}(e)$. Hence $e \in \psi_{x} \Leftrightarrow tet^{*}  \in \theta_{t}(\psi_{x})$.
\item $\psi_{x}=\psi_{y} \Leftrightarrow \psi_{t(x)}=\psi_{t(y)}$, in fact more is true as: $\psi_{t(x)}=\psi_{t^{'}(y)} \Leftrightarrow Dom(t^{'})=Dom(t)$.
\end{enumerate}
We prove inclusions. First $\theta_{t}(\psi_{x}) \subset \psi_{t(x)}$. Without loss of generality, we can take $tet^{*}$ to be the general form of an element of $\theta_{t}(\psi_{x})$ and then: $tet^{*} \in \psi_{t(x)} \Leftrightarrow t(x) \in tet^{*} \Leftrightarrow tet^{*}(t(x))=t(x)$, which is the case if and only if $e \in \psi_{x}$.

To see the reverse inclusion let $f \in \psi_{t(x)}$. Then $f \in \theta_{t}(\psi_{x}) \Leftrightarrow t^{*}ft \in \psi_{x} \Leftrightarrow t(x) \in f \Leftrightarrow f \in \psi_{t(x)}$. 

To conclude; (3) controls the behavior of the action when the stabilizer is non-trivial, the first part shows the action behaves with respect to the quotient and the second part shows that given any pair $(x^{'},y^{'})\in X \times X$ such that $x^{'}\in Hx, y^{'}\in Ht(x)$ the unique translation $t_{x^{'}y^{'}} \in \mathcal{T}$ that sends $x^{'}$ to $y^{'}$ defines an arrow between $\psi_{x}$ and $\psi_{t(x)}$. 
\end{proof}

\begin{proposition}\label{cor:C4}
$B$ is invariant and $\G_{B}$ is connected. Moreover if $H$ is trivial, then $B$ is invariant and $\G_{B}$ is uniquely connected.
\end{proposition}
\begin{proof}
$B$ is invariant as a consequence of Proposition \ref{prop:P10}, and connected by the first property of grouplike partial translation structures - $\mathcal{T}$ partitions $X \times X$. This tells us that when we pass to the quotient $H \backslash X$ we get a collection of arrows between each pair of points that are indexed by $H$. 

In the situation that $H$ is trivial we get a unique arrow between any two points in $B$ and $B$ is in bijection with $X$, hence the groupoid $\G_{B}$ is precisely the pair groupoid $X \times X$, with the norm coming from the stalks, each of which have the form of $\ell^{2}(X)$ by the uniquely connected property of $B$.
\end{proof}

\begin{remark}
In the situation that $H$ is non-trivial, we have a unit space for $\G_{B}$ that is $B\times B$, with arrows between each pair indexed by $H$. The Hilbert space associated to each fibre $L^{2}(\G_{\X})|\psi_{x}$ is exactly the Hilbert space with basis indexed by the \textit{arrows} $[t_{h},\psi_{x}]$ - the set of arrows based at $\psi_{x}$ is in bijection with $X$, construct a map using the first property of translation structures.

For each point $hx \in Hx$ there exists a unique translation $t_{y,h}$ to each other point $y \in X$. We then define the map $[t_{y,h}, \psi_{x}] \mapsto t_{y,h}(hx)=y$.

This provides a unitary isomorphism between these spaces, denote this map at the level of Hilbert spaces by $U_{x}$ for each $x \in H \backslash X$.
\end{remark}

\begin{proposition}\label{prop:P11}
$\Vert \lambda(1_{tt^{*}}\delta_{t}) \Vert = \Vert \mu(t) \Vert_{\ell^{2}(X)}$ for all $t \in \mathcal{T}$. 
\end{propositon}
\begin{proof}
The proof of this fact follows from a computation on the basis of $\ell^{2}(X)$ using the unitary isomorphism $U_{x}$. We compute $U_{x}\lambda_{\psi_{x}}(1_{tt^{*}}\delta_{t})U_{x}^{-1}$ evaluated on a basis element $\delta_{y} \in \ell^{2}(X)$.
\begin{enumerate}
\item $U_{x}^{-1}(\delta_{y})=\delta_{[t_{y,h},\psi_{x}]}$
\item $\lambda_{\psi_{x}}(1_{tt^{*}}\delta_{t})(\delta_{[t_{y,h},\psi_{x}]})(\delta_{[s,\psi_{x}]})=\sum_{\substack{[n,\psi_{z}][u,\psi_{x}]\\=[s,\psi_{x}]}} 1_{tt^{*}}([n, \psi_{z}]\delta_{[t_{y,h},\psi_{x}]}([u,\psi_{x}])=\delta_{[t_{y,h},\psi_{x}]}$

Hence $\lambda_{\psi_{x}}(1_{tt^{*}}\delta_{t})$ moves the basis element $\delta_{[t_{y,h},\psi_{x}]}$ to the basis element $\delta_{[s,\psi_{x}]}$, where $s$ is the unique translation above $tt_{y,h}$ in $\mathcal{T}$.
\item $U_{x}(\delta_{[tt_{y,h},\psi_{x}]})=\delta_{t(t_{y,h}(hx))}=\delta_{t(y)}=\mu(t)(\delta_{y})$.
\end{enumerate}
This holds for all $y$ in the domain of $t$, as the multiplication in the groupoid is defined for only that situation.

As we have this equality for each $\psi_{x} \in B$; we get that $\Vert \lambda(1_{tt^{*}}\delta_{t}) \Vert_{r} = \sup \lbrace \Vert \lambda_{\psi_{x}}(1_{tt^{*}}\delta_{t}) \Vert : \psi_{x} \in B \rbrace = \Vert \mu(t) \Vert_{\ell^{2}(X)}$.
\end{proof}
\end{proposition}

This extends to finite sums:

\begin{lemma}\label{lem:L7}
Let $K \subset \X$ be a finite subset and let $a_{t}$ be the constant function valued $a_{t}$ on $D_{tt^{*}}$. Then \Vert \sum_{t \in K} a_{t}\delta_{t} \Vert_{r} = \Vert \sum_{t \in K} a_{t} \mu(t) \Vert_{\ell^{2}(X)}  
\end{lemma}
\begin{proof}

First we show that $\sum_{t \in K} a_{t}\delta_{t}$ represents as $\sum_{t \in K} a_{t} \mu(t)$ on the basis of $\ell^{2}(X)$. We proceed as in proposition \ref{prop:P10}. 

First compute $U_{x}^{-1}(\delta_{y})$:
\begin{equation*}
U_{x}^{-1}(\delta_{y})=\delta_{[t_{y,h},\psi_{x}]}
\end{equation*}
Then compute: 
\begin{eqnarray*}
(\sum_{t \in K}\lambda_{\psi_{x}}(a_{t}\delta_{t}))(\delta_{[t_{y,h},\psi_{x}]})& = &\sum_{\left\lbrace t\in K:\substack{[t,\psi_{y}][t_{y,h},\psi_{x}]\\=[tt_{y,h},\psi_{x}] }\right\rbrace} a_{t}([t, \psi_{y}])\delta_{[t_{y,h},\psi_{x}]}([t_{y,h},\psi_{x}])\\
& = &\sum_{\lbrace t \in K, y \in Dom(t)\rbrace }a_{t}([t,t_{y,h}(\psi_{x})])\delta_{[tt_{y,h}, \psi_{x}]}\\ & = &\sum_{t \in K,y \in Dom(t)}a_{t}\delta_{[tt_{y,h}, \psi_{x}]}
\end{eqnarray*}
Lastly move back to $\ell^{2}(X)$ via $U_{x}$:
\begin{eqnarray*}
U_{x}(\sum_{t \in K,y \in Dom(t)}a_{t}\delta_{[tt_{y,h}, \psi_{x}]}) & = &\sum_{t \in K, y \in Dom(t)}a_{t}\delta_{t(y)}\\ & = &(\sum_{t \in K}a_{t}\mu(t))(\delta_{y})
\end{eqnarray*}

\end{enumerate}

So both finite sums transform the basis in the same way. This equality holds for each $\psi_{x}$ in $B$, so we can conclude that $\Vert \sum_{t \in K} a_{t}\delta_{t} \Vert_{r} = \sup \lbrace \Vert \lambda_{\psi_{x}}(\sum_{t \in K} a_{t}\delta_{t} ) \Vert : \psi_{x} \in B \rbrace = \Vert \sum_{t \in K} a_{t} \mu(t) \Vert_{\ell^{2}(X)}$.
\end{proof}

This lets us define a map: 
\begin{equation*}
\mathcal{Q}: \lambda(1_{tt^{*}}\delta_{t}) \mapsto \mu(t)
\end{equation*}

Now we can state and prove the main result of this section:
\begin{theorem}\label{thm:T5}
Let $X \subset G$, giving us a translation structure $\mathcal{T}_{G}|_{X}$ with no zero divisors and a representation $\mu: S = \langle \mathcal{T}_{G}|_{X} \rangle$. Then we have an isomorphism C^{*}_{r}(\G_{\X}) \cong C^{*}_{\mu}(S) = C^{*}\mathcal{T}. 
\end{theorem}

\begin{proof}
The map $\mathcal{Q}$ is surjective onto $\mathbb{C}S$ (mapping to the generators of $S$), so it remains to see that it passes to the completion and is injective. 
To show this, we appeal to Lemma \ref{lem:L7} to show that the norms match under the map $\mathcal{Q}$ up to finite sums - making the map on the incomplete algebras uniformly continous. Ideally, we would now complete - however we need to be careful as the incomplete *-algebra $M$ generated by finite sums of $1_{e}\delta_{t}$ may not be dense (and is the source of the map $\mathcal{Q}$).

However we observe that by the Stone-Weierstrass Theorem every element in $C(\X)$ can be approximated by elements of the form $1_{e}$, i.e for all $f_{t} \in C(\X)$:
\begin{equation*}
f_{t} = \lim_{n} (\sum_{e \in E} a^{n}_{e}1_{e})
\end{equation*}
So for a particular element  $f=\sum_{t \in K}f_{t}\delta_{t} \in C_{c}(\G_{\X})$ we can approximate each $f_{t}$ in turn by limits of $\sum_{e \in E} a_{e}1_{e}$ giving us an approximation by finite sums of elements in $M$. Hence $\overline{M} = C^{*}_{r}(\G_{\X})$, allowing $\mathcal{Q}$ to pass to completions (by uniform continuity).

After passing to the completion, the map is isometric; hence injective. This gives us the first isomorphism. To see the equality, observe that by definition the translation algebra is the algebra generated in $\mathcal{B}(\ell^{2}(X))$ by the set of operators $\lbrace \mu(t) | t \in \mathcal{T} \rbrace$. This is precisely $C^{*}_{\mu}(S)$.
\end{proof}

\subsection{A Short Exact Sequence of Translation Algebras}

In this situation we still have access to all the tools available in the general inverse monoid case, as well as all the geometric properties arising from the representation $\mu$.

\begin{theorem}\label{thm:T4}
Let $X \subset G$, $\mathcal{T}=\mathcal{T}_{G}|_{X}$ be a grouplike partial translation structure on $X$ with no zero divisors and $S=\langle \mathcal{T} \rangle \hookrightarrow_{\mu} I(X)$ be the associated F-inverse monoid. Then we have the following short exact sequence of $C^{*}$-algebras:
\begin{equation}\label{eqn1}
0 \rightarrow C^{*}_{r}(\G_{U}|_{\X}) \rightarrow C^{*}_{r}(\G_{\X}) \rightarrow C^{*}_{r}(G) \rightarrow 0
\end{equation}
Where the middle term is the translation algebra associated to $X$ arising from $\mathcal{T}$
\end{theorem}
\begin{proof}
The proof follows the same lines as the proof of Theorem \ref{thm:PV1}.
\begin{enumerate}
\item The map defined on finite sums still has the desired properties, i.e a finite sum is in the kernel if and only if all its components are 0.
\item We still have the same pullbacks of constant functions to the entire space; This enables the same construction of approximating elements; who each have the same norm control property provided by Corollary \ref{cor:C1}.
\item We can then conclude that the kernel is the desired algebra with a density argument.
\end{enumerate}
\end{proof}
