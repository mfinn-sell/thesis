\begin{center}
UNIVERSITY OF SOUTHAMPTON\\[0.75cm]

\underline{ABSTRACT}\\[0.55cm]

FACULTY OF SOCIAL AND HUMAN SCIENCES\\
MATHEMATICAL SCIENCES\\[0.75cm]

\underline{Doctor of Philosophy}\\[0.55cm]

INVERSE SEMIGROUPS IN COARSE GEOMETRY\\[0.6cm]

By Martin Finn-Sell\\[1.5cm]
\end{center}


\addcontentsline{toc}{chapter}{Abstract.}
Inverse semigroups provide a natural way to encode combinatorial data from geometric settings. Examples of this occur in both geometry and topology, where the data comes in the form of partial bijections that preserve the topology, and operator algebras, where the partial bijections encode $*$-subsemigroups of partial isometries of Hilbert space. In this thesis we explore the connections between these two pictures within the backdrop of coarse geometry.

The first collection of results is concerned primarily with inverse semigroups and their $C^{*}$-algebras. We give a construction of a six term sequence of $C^{*}$-algebras connecting the semigroup $C^{*}$-algebra to that of a naturally associated group $C^{*}$-algebra. This result is a generalisation of the ideas of Pimsner and Voiculescu, who were concerned with computing K-theory groups associated to actions of groups. We outline how to connect this picture, via groupoids, to that of a partial translation algebra of Brodzki, Niblo and Wright, and further consider applications of these sequences to computations of certain K-groups associated with group and semigroup $C^{*}$-algebras.

Secondly, we give an account of the coarse Baum-Connes conjecture associated to a uniformly discrete bounded geometry metric space and rephrase the conjecture in terms of groupoids and their $C^{*}$-algebras that can naturally be associated to a metric space. We then consider the well-known counterexamples to this conjecture, giving a unifying framework for their study in terms of groupoids and a new conjecture for metric spaces that we call the boundary coarse Baum-Connes conjecture. Generalising a result of Willett and Yu we prove this conjecture for certain classes of expanders including those of large girth by constructing a partial action of a discrete group on such spaces.