%\setstretch{1.6}
\chapter{Introduction.}

Groups have played a role in geometry, topology and analysis throughout the last century. The overall theme is that to recognise objects one must focus on their symmetries; the set of structure preserving automorphisms of an object form a group that describes the ways to permute the object whilst preserving what makes it interesting \textit{globally}. This idea works well within both geometry and topology; the fundamental group of a manifold plays an important role in both areas, as well as entering the realms of physics through representation theory. This makes group theory a natural candidate for study.

Another possible model with similar behaviour is that of a \textit{groupoid}. An example is the fundamental groupoid that is constructed in algebraic topology similar to the path groupoid of a graph. These objects arise much more generally than groups and are able to play much more subtle roles in classifying structures such as equivalence relations (dealing with non-Hausdorff topological quotient spaces \cite{MR1826266}), group actions and encoding coarse information about metric and topological spaces \cite{MR1905840}. In each instance these groupoids capture local transformational data about the underlying object space.

A natural question one could ask is ``What happens with the local structure?" This is where \textit{semigroups} enter the picture; local symmetries can be captured by \textit{partial bijections}, which form an \textit{inverse semigroup} under composition. This point of view is less publicised than the corresponding groupoid theory, but work on these ideas enters into both geometry and topology in many places \cite{MR0160848,MR1694900,MR1798993}, even breaking into physical questions concerning aperiodic tilings \cite{MR1798993,MR2041539}.

We have a dictionary between these two views; inverse semigroups admit a groupoidification \cite{MR1724106,MR2419901}. There exists for each inverse semigroup $S$ a universal groupoid with the same linear representation theory. These connections make inverse semigroups and groupoids very useful to study from the point of view of answering questions that require some analysis or topology. Each object has natural advantages, the combinatorial theory of semigroups is much more developed than that of groupoids, but the topological aspects of groupoids often play an important role within applications.

To illustrate this view we will consider the following very simple but general example:
\begin{example}\label{Ex:Int1}
Let $\Gamma$ be a finitely generated discrete group and let $X$ be a subset of the Cayley graph of $\Gamma$. Fix a left invariant metric on $\Gamma$ and equip $X$ with the subspace metric. The right action of $\Gamma$ on itself given by multiplication by inverses gives us a set of maps:
\begin{equation*}
t_{g}: \Gamma \rightarrow \Gamma , x \mapsto xg^{-1}
\end{equation*}
We can now restrict these maps to $X$, where they may not be defined everywhere. Denote the set of points in $X$ with image in $X$ by $D_{g}$. Then we have:
\begin{equation*}
t_{g}^{X}: D_{g} \rightarrow D_{g^{-1}} , x \mapsto xg^{-1}
\end{equation*}
These are partial bijections of $X$, that is maps that are bijections between subsets of $X$, that move points of $X$ bounded distances. We can then generate subsemigroup of all the partial bijections on $X$ using this collection. This monoid belongs to the class of semigroups known as \textit{inverse} semigroups, and it captures both the metric of $X$, as the group action determines the metric on $\Gamma$ via the group action coarse structure. Additionally, it gives information about the local structure of $X$ (as the $D_{g}$ need not be all of $X$, or even connected, however the collection of partial translations does provide a partition of $X\times X$). This is an example of a \textit{partial} action of $\Gamma$ and the inverse monoid it generates belongs to a very nice combinatorial class known as \textit{strongly 0-F-inverse} monoids that have been well-studied in the literature \cite{MR1721768,MR1798993,Nor-2012}.
\end{example}

The work presented in this document develops this connection between group theory and inverse semigroup theory on the one hand and topology and geometry on the other by considering the universal groupoid associated to the partial action defined above. Via this groupoid we get access to the much more well developed analytic tools of noncommutative geometry; groupoid $C^{*}$-algebras are very well studied in comparison to those of an inverse monoid. In particular, we will be interested in constructing a $C^{*}$-algebraic analogue to Example \ref{Ex:Int1}. With that in mind we consider a special case of Example \ref{Ex:Int1}, but add the operator algebra view.

\begin{example}\label{Ex:Int2}
Let $\Gamma=\mathbb{Z}$ and let $X=\mathbb{N}$. The maps defined above turn into:
\begin{eqnarray*}
& t_{n}: \mathbb{N} \rightarrow \mathbb{N}\setminus \lbrace 0,1,...,n-1 \rbrace, x \mapsto x+n \\
&t_{-n}: \mathbb{N}\setminus \lbrace 0,1,...,n-1 \rbrace \rightarrow \mathbb{N} , x \mapsto x-n
\end{eqnarray*} 
These partial bijections generate an inverse monoid, given by the presentation:
\begin{equation*}
S=\langle t_{1},t_{-1} | t_{-1}t_{1}=1 \rangle.
\end{equation*}
This is a well-known object in inverse semigroup theory: \textit{the bicyclic monoid}. We can also consider these maps as partial isometry operators inside $\mathcal{B}(\ell^{2}(X))$ in a very natural way and then consider the $C^{*}$-algebra they generate. This algebra is called the translation algebra $C^{*}T$ associated to the set of maps $T$. In this instance it coincides the the inverse semigroup $C^{*}$-algebra $C^{*}_{r}(S)$, which is defined using the natural multiplication of $S$ as partial isometries on $\ell^{2}(S)$. This $C^{*}$-algebra also satisfies the presentation defined above.

This is well-known to operator algebraists: $C^{*}(T)\cong C^{*}_{r}(S) \cong \mathcal{T}$: the Toeplitz algebra. This fits into the short exact sequence:
\begin{equation*}
0 \rightarrow \mathcal{K}(\ell^{2}(\mathbb{N})) \rightarrow \mathcal{T} \rightarrow C(S^{1}) \rightarrow 0.
\end{equation*}
This can be translated into semigroup language:
\begin{equation*}
0 \rightarrow \mathcal{K}(\ell^{2}(\mathbb{N})) \rightarrow C^{*}_{r}(S) \rightarrow C^{*}_{r}(\mathbb{Z}) \rightarrow 0.
\end{equation*}
The last isomorphism arises from the Fourier transform, but is recorded combinatorially by the fact that $\mathbb{Z}$ is the maximal group homomorphic image of $S$, i.e is given by $S$ after quotienting by a congruence. 
\end{example}

As we saw in the construction above there is an inverse semigroup that is $0$-F-inverse capturing the partial action that underlies the translation algebra. Studying representations of inverse monoids within this class one might then wonder how much of this result is true in general. A study of this is performed in Chapter 3 and this is one of the main results:

\begin{thm}\label{Ex:Int3}
Let $S$ be an F-inverse monoid and let $\G_{\E}$ be its universal groupoid and let $A=C_{c}(\G_{U})$. Then we have the following  short exact sequence of $C^{*}$-algebras:
\begin{equation*}
0 \rightarrow \overline{A} \rightarrow C^{*}_{r}(\mathcal{G}) \rightarrow C^{*}_{r}(G) \rightarrow 0.
\end{equation*}
\end{thm}

The above theorem also captures the work of Pimsner and Voiculescu \cite{MR670181} concerning the action of free groups on $C^{*}$-algebras, which was a fundamental development in noncommutative geometric techniques in operator K-theory.

We discuss the computations of the K-theory of the translation inverse monoid in Chapter 5, which are simple as there is a large machinery in the literature to compute these K-groups \cite{Nor-2012,CEL-2}. We see, unlike in the example above, that the inverse semigroup reduced $C^{*}$-algebra is not the correct choice to reconstruct the calculations of \cite{MR670181} completely from the Theorem above; We prove a similar result about the representation connected to the translation structure:

\begin{thm}
Let $X \subset G$, $\mathcal{T}=\mathcal{T}_{G}|_{X}$ be a grouplike partial translation structure on $X$ with no zero divisors and $S=\langle \mathcal{T} \rangle \hookrightarrow_{\mu} I(X)$ be the associated F-inverse monoid. Then we have the following short exact sequence of $C^{*}$-algebras:
\begin{equation}\label{eqn1}
0 \rightarrow C^{*}_{r}(\G_{U}|_{\X}) \rightarrow C^{*}_{r}(\G_{\X}) \rightarrow C^{*}_{r}(G) \rightarrow 0.
\end{equation}
Where the middle term is the translation algebra associated to $X$ arising from $\mathcal{T}$
\end{thm}

This does produce the correct short exact sequences for the algebras that arise from Example \ref{Ex:Int1} and \ref{Ex:Int2}. In general the K-theory is much harder to compute but is connected to the easier computations for the inverse semigroup $C^{*}$-algebra provided by the work of \cite{Nor-2012,CEL-2} via a complex of short exact sequences.

In general the inverse monoid generated by the translation structure by the construction of Example \ref{Ex:Int1} will not be as well behaved as in these examples and will not satisfy the Theorems above.
Through machinery generalising work of Khoshkam and Skandalis \cite{MR1900993} captured by Milan and Steinberg \cite{Milan-Steinberg} and suitably weak hypothesis on the group we prove:

\begin{thm}
Let $X \subset \Gamma$ where $\Gamma$ is K-exact and let $\mathcal{T}=\mathcal{T}_{\Gamma}|_{X}$ be a grouplike partial translation structure on $X$. Consider $S=\langle \mathcal{T} \rangle \hookrightarrow_{\mu} I(X)$ the associated strongly 0-F-inverse monoid. Then we have the following short exact sequence of $C^{*}$-algebras:
\begin{equation*}
0 \rightarrow C^{*}_{r}(\G_{U}|_{\X}) \rightarrow C^{*}_{r}(\G_{\X}) \rightarrow C^{*}_{r}(\G_{\E}|_{\X\cap\E_{tight}}) \rightarrow 0.
\end{equation*}
\end{thm}

We explore in chapter 5 the K-theory of the Cuntz algebras $\mathcal{O}_{n}$ from this perspective as well as applying this idea to give an alternative proof that Gromov's monster groups \cite{MR1978492,exrangrps} are not K-exact.

Another way to use the ideas of Example \ref{Ex:Int1} is to see what can be said about the coarse geometry of a metric space $X$ given that it admits a partial action by a discrete group $\Gamma$. The coarse information we are interested in is captured by the coarse Baum-Connes conjecture; recall that the coarse Baum-Connes conjecture \cite{MR1388312} asks if a certain assembly map:
\begin{equation*}
\mu_{X,red}:KX_{*}(X) \longrightarrow K_{*}(C^{*}(X))
\end{equation*}
is an isomorphism for $X$ a uniformly discrete bounded geometry metric space. This conjecture is a geometric intrepretation of the well-known \textit{Baum-Connes conjecture} \cite{MR1292018}, and connects to it via a \textit{descent principle} \cite{MR1399087,MR1817560}; for a finitely generated group $\Gamma$ the associated Cayley graph will be a uniformly discrete space with bounded geometry and a positive result for the coarse Baum-Connes conjecture in such situations has strong implications such as the Strong Novikov conjecture concerning the homotopy invariance of the higher signatures of a smooth manifold \cite{MR866507} or the existence of metrics with positive scalar curvature for manifolds $M$ that have $\pi_{1}(M)\cong \Gamma$ \cite{MR1817560}.

The Baum-Connes conjecture can be developed in other directions, particularly into the realm of topological groupoids \cite{MR1798599}. It is a well known result from \cite{MR1905840} that the above statement of the coarse Baum-Connes conjecture can be replaced with a conjecture with coefficients for some groupoid $G(X)$ that we can associate to any uniformly discrete bounded geometry metric space $X$.

In this context, the coarse Baum-Connes conjecture asks if the map:
\begin{equation*}
\mu_{r}:K_{*}^{top}(G(X), \ell^{\infty}(X,\mathcal{K})) \rightarrow K_{*}(\ell^{\infty}(X,\mathcal{K})\rtimes_{r}G(X))
\end{equation*}
is an isomorphism.

The beginning of Chapter 4 develops these ideas from considering the basics of coarse geometry through to the groupoid definition of the coarse assembly map. There are two main objectives within the chapter: first to outline the constructions of counterexamples to the conjecture \cite{higsonpreprint,MR1911663,explg1,explg2,MR2568691} and give simplifications via single unified method: the boundary coarse Baum-Connes conjecture. This conjecture, defined via groupoids, tackles the space at infinity:
\begin{conjecture}
Let $X$ be a uniformly discrete bounded geometry metric space, let $G(X)$ be the associated coarse groupoid on $X$ and let $A_{\partial}= l^{\infty}(X,\mathcal{K})/C_{0}(X,\mathcal{K})$. Then:
\begin{equation*}
\mu_{bdry}:K_{*}^{top}(G(X)|_{\partial\beta X}, A_{\partial}) \rightarrow K_{*}(A_{\partial}\rtimes_{r}G(X)|_{\partial\beta X})
\end{equation*}
is an isomorphism.
\end{conjecture}
This conjecture, if true, provides information about the coarse Baum-Connes conjecture via homological methods. We outline these methods in Chapter 4.

The class of spaces this conjecture is designed to study are expander graphs \cite{MR2247919}; these play a large role in the counterexample arguments in the literature. In particular the main result of the Chapter, generalising work of Willett and Yu \cite{explg1}, is the following:
\begin{thm}
The boundary coarse Baum-Connes conjecture is true for sequences of finite graphs with large girth and uniformly bounded vertex degree.
\end{thm}

The process to prove this associates to each such sequence a partial action of some finitely generated free group. This partial action does not generate the metric as in Example \ref{Ex:Int1} but does control how the metric behaves at infinity. 

Finally, in Chapter 5 we tie these ideas together. Firstly, certain examples of the short exact sequences of Chapter 3 and their K-theory are considered. Secondly, a counterexample to the boundary coarse Baum-Connes conjecture is constructed and lastly we show that Gromov monster groups, the groups that coarsely contain large girth expanders, fail to satisfy Baum-Connes with coefficients \cite{MR1978492,exrangrps}, and show that there are coefficients where the conjecture holds.

In summary, in Chapter 2 we make precise the definitions and properties of inverse semigroups and groupoids that we will use within this thesis, as well as outlining the connections between them that are present in the literature. Following this we define partial actions of groups, which become the primary objects of study in later chapters. Lastly, we give the definition of a $C^{*}$-algebra and develop ideas concerned with $C^{*}$-algebras of groupoids and inverse semigroups as well as introducing the methods of topological K-theory.

In Chapter 3 the results concerning short exact sequences associated to F-inverse and strongly 0-F-inverse monoids that were outlined above are proved. The connections to coarse geometry are introduced; we introduce the concept of a partial translation structure and then use this to construct a short exact sequence associated to any sufficiently good subset of a finitely generated group. 

In Chapter 4 the focus changes to metric spaces and coarse geometry. The coarse Baum-Connes conjecture is defined via two different approaches, one analytic and one via a groupoid construction from the literature. In this instance we focus on the groupoid version and explain how counterexamples to the conjecture are constructed. We then develop a new conjecture, the boundary coarse Baum-Connes conjecture, and prove it for certain sequences of finite graphs.

Lastly, Chapter 5 is devoted to giving examples and connections between the ideas of the previous chapters. We compute some K-theory groups associated to both translation algebras and inverse monoid $C^{*}$-algebras, give a counterexample to the boundary coarse Baum-Connes conjecture and use translation structure ideas and the results of Chapter 4 to prove that Gromov monster groups are not exact.
