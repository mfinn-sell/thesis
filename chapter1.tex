\setstretch{1.6}
\chapter{Introduction.}

General fluff about inverse semigroups and groupoids being important, particular examples include the partial action of a discrete group in both cases. Connect in the work on partial symmetries, remark that this connects work on tilings as well as work of Macbeth on decidablity problems for certain groups.

Outline the general approach we're going to use in the thesis. Move to some specific examples.

\subsection{The work of Pimsner and Voiculescu.}

This leads to certain generalisations associated to F-inverse monoids...

State the main results on this, and outline certain K-theory discussions. This is part 1; Using partial actions of a group to understand the group.

\subsection{Counterexamples to the coarse Baum-Connes conjecture.}

This arises from considering certain bad graphs. Explain why this conjecture is relevant, and why on earth understanding the graphs is a good idea. Explain that natural examples come with a group action, and explain that we generalise this idea to get bad sequences with a partial free group action. .

Outline the main theorems.

This is ii), understanding a space by understanding that it admits a partial group action.

\subsection{Further connections.}

Explain that the ideas arising from chapter 3 will connect with those of chapter 4. And outline how this connection runs through a partial translation structure / algebra. This is interplay between i) and ii)

State some results;

Explain the outgoing ideas and open problems this thesis leaves behind.

Smile. It's only 6-10 pages!

